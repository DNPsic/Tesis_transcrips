\section{Sesión del de del 2024}

\noindent
Sesión del 09 de agosto de 2024.\\
ID: AH-0908-2024.\\
Participantes:\\
AH: Mujer de 65 años, informante.\\
D: Daniela Salinas, entrevistadora. \\
DT: Dante Nava, entrevistador.\\
AC: Hombre, hijo de AH.\\
\\
\noindent
001 \# Segmento: inicio de sesión.\\
002 \# Condición: lectura.\\
003 \# Material: cuaderno de AH, pizarrón blanco y marcadores.\\
004 \# AH lee el contenido escrito en su cuaderno previamente.\\
005 AH: yo estaba en soriana a las doce y media con mi hija\\
006 D: muy bien\\
007 AH: en el refigrerador la ca la crema, pan español, yo yo, y aceite y dos leches y los atunes\\
008 D: ahora explíqueme sus dibujos\\
009 AH: leche atún crema aceite \# sus dibujos son figuras simples con las palabras escritas dentro, señala cada y las lee.\\
010 D: de acuerdo y ¿dibujó cuando estaba usted en soriana?\\
011 AH: sí\\
012 D: ¿dónde?\\
013 AH: aquí \# señala la oración que leyó al inicio.\\
014 D: pero el dibujo\\
015 AH: \# ve a D sin dar respuesta.\\
016 D: dibújeme un pez \# le da el marcador rojo de pizarrón.\\
017 AH: ¿un pez?\\
018 D: un pez\\
019 AC: un pececito\\
020 AH: un pez \# dibuja un óvalo con una línea pequeña en la parte inferior.\\
021 D: ¿le falta algo a ese pez?\\
022 AH: las \# dibuja dos lineas más desde los lados del óvalo, una del lado izquierdo y otra en el derecho.\\
023 D: sus aletitas ¿y su colita?\\
024 AH: colita \# dibuja dos puntos dentro del óvalo.\\
025 D: muy bien a parte de esto me dijo que hizo más ¿no?\\
026 AH: a ver\\
027 D: ¿esta no? \# le muestra otra hoja de su cuaderno.\\
028 AH: a ver\\
029 \# Escrito: la hoja muestra en la parte superior «8 de agosto fecha 2024», a continuación se lee (primer párrafo) «los regalos de los sicologos y las playeras del doctor alejandro» (segundo párrafo) «la doctora Daniela el doctor Guillermo», el uso de minúsculas y mayúsculas es inconsistente a diferencia de lo representado aquí.\\
030 \# Escrito: la hoja muestra en la parte superior «viernes 9  de Agosto 2024», a continuación (primer párrafo) «ayer fui a soriana a comprar sandia y tortillas», (segundo párrafo) «yo vi la televicion», finalmente se observan 3 dibujos y debajo de cada uno las palabras «television, sandia, tortillas», el primer dibujo es un rectángulo con 4 puntos en su parte superior, el segundo es una forma ovalada con 6 puntos en su interior, el trecero son 6 puntos alineados verticalmente y en dos columnas.\\
031 D: muy bien, léame esta primera oración \# señala la primera oración.\\
032 \# Condición: lectura.\\
033 AH: fui a a soriana a comprar sandía y tortillas\\
034 D: muy bien\\
035 AH: yo yo va yo vi vi la televisión\\
036 D: muy bien dígamnumero: .\\
037 AH: fui\\
038 D: ¿qué otro verbo hay?\\
039 AH: comprar\\
040 D: ¿y en esta otra oración cuál es el verbo? \# señala la segunda oración.\\
041 AH: yo\\
042 D: no, acuérdese que los verbos son acciones\\
043 AH: yo vi, vi vi\\
044 D: muy bien, por ejemplo en la primera oración usted ya me dijo que el verbo es fui\\
045 AH: ajá sí\\
046 D: y también comprar mientras que en la segunda es vi\\
047 AH: ah sí\\
048 D: pero por ejemplo dígame cosas que podemos encontrar en esa oración, cosas no verbos\\
049 AH: fui\\
050 D: no ese quedamos que era el verbo, quedamos que las cosas no sienten, por ejemplo si yo le pego a esto \# da dos golpes al borrador del pizarrón. no siente\\
051 AH: ajá\\
052 D: sólo si alguien más los mueve, generalmente no tienen intenciones, o no se mueve solo\\
053 AH: ajá\\
054 D: por ejemplo si yo dibujo \# dibuja una rebanada de una sandía en el pizarrón.\\
055 AH: ah sandía\\
056 D: ajá muy bien ¿esto qué es una cosa o es un verbo?\\
057 AH: este cosa\\
058 D: ¿la sandía siente?\\
059 AH: no no\\
060 D: muy bien y por ejemplo si yo dibujo \# dibuja un rectángulo y unas figuras humanas dentro, al lado un óvalo con círculos de diferentes tamaños. imagínese que esto es una televisión ¿esto sería una cosa o una acción?\\
061 AH: no, una acción\\
062 D: ¿una acción? por qué\\
063 AH: la televisión estaba bien\\
064 D: pero usted el ver, que usted esté por ejemplo  dibuja una figura humana al lado del dibujo anterior. imagínese que aquí está usted, y está observando la televisión  dibuja una flecha desde la cabeza de la figura humana hasta el dibujo de la televisión. pero usted está viéndolo, usted experimenta usted sí siente\\
065 AH: sí\\
066 D: usted percibe las imágenes ¿de acuerdo?\\
067 AH: cosa las cosas\\
068 DT: ajá muy bien\\
069 D: sí, usted percibe las cosas, usted por ejemplo puede ver las hojas\\
070 AH: sí sí\\
071 D: la bolsa, mis bibujos\\
072 AH: sí\\
073 D: pero quien siente es usted\\
074 AH: ajá\\
075 D: y usted lo que está haciendo es ver\\
076 AH: ver\\
077 D: ese sí es el verbo, que usted experimente todo eso es el verbo\\
078 AH: ah\\
079 D: pero esto \# señala el dibujo de la televisión. sólo es una cosa sólo existe, y no hace nada más ni siente nada más\\
080 AH: sí\\
081 D: entonces la televisión ¿es una cosa o es un verbo?\\
082 AH: una cosa\\
083 D: muy bien, y por ejemplo ¿ese reloj?  señala el reloj de pared.\\
084 AH: una cosa\\
085 D: ¿y sus lentes?\\
086 AH: mis lentes\\
087 D: ¿son una cosa o son una acción?\\
088 AH: una cosa una cosa\\
089 D: y por ejemplo cuando sus niñas están jugando, jugando ¿jugar es una acción o es una cosa?\\
090 AH: acción acción\\
091 D: y por ejemplo cuando sus hijos están trabajando\\
092 AH: acción\\
093 D: ¿y la puerta, la puerta qué es?\\
094 AH: cosa cosa\\
095 D: muy bien a ver un último ejemplo, la silla ¿es una cosa o una acción?\\
096 AH: cosa\\
097 D: ahora yo le voy a decir unas palabras y usted las va a escribir\\
098 AH: sí\\
099 \\
100 \# Segmento: clasificación de cosas y acciones al dictado.\\
101 \# Material: pizarrón y marcador rojo. en el pizarrón se observan dos columnas, en la izquierda dice acción y en la derecha cosa.\\
102 D: el papel de baño ¿qué es una cosa o una acción?\\
103 AH: cosa\\
104 D: a ver escriba ahí papel de baño\\
105 AH: ¿pa?\\
106 D: papel\\
107 \# Condición: escritura.\\
108 AH: pa \# escribe «Pa». pa \# escribe la letra P.\\
109 D: pape\\
110 AH: ¿ele?\\
111 D: no no aquí leame qué dice \# señala lo que AH acaba de escribir.\\
112 AH: pa\\
113 D: pap hasta acá dice pap, pape \# hace énfasis en el sonido de la e.\\
114 AH: ¿ele?\\
115 D: no, antes de la ele, falta un sonido pe /p-e/ \# hace énfases en los dos sonidos.\\
116 AH: ah pe\\
117 D: pero la letra pe ya está aquí ¿cuál es siguiente sonido? /p/ es este sonido \# señala la última letra P escrita. y el siguiente /e/\\
118 AH: ah ((enrique)) \# escribe la letra E.\\
119 D: ahora sí ¿qué nos falta?\\
120 AH: ele\\
121 D: ajá muy bien, ¿papel de?\\
122 AH: higiénico\\
123 D: de\\
124 AH: de \# escribe «de» con letra cursiva.\\
125 D: muy bien, papel de baño\\
126 AH: ba \# escribe «va».\\
127 D: con esa la podemos dejar de momento, baño\\
128 AH: ¿eñe?\\
129 D: sí muy bien\\
130 AH: \# escribe la letra Ñ.\\
131 D: ¿qué nos falta?\\
132 AH: /o/ \# escribe la letra o.\\
133 \# Condición: conversacion.\\
134 DT: muy bien\\
135 D: ahora por ejemplo, caminar ¿caminar es una cosa o una acción?\\
136 AH: acción acción\\
137 DT: muy bien\\
138 D: sí muy bien, por ejemplo cuando caminamos hay movimiento, escriba caminar\\
139 \# Condición: escritura.\\
140 AH: caminar \# escribe «camiNar» en la columa de acción, la escribe mientras dice en voz alta la palabra separando por sílabas ca-mi-nar.\\
141 \# Condición: conversacion.\\
142 D: muy bien ahora por ejemplo licuar ¿licuar qué es una acción o es una cosa?\\
143 AH: cosa\\
144 D: ¿licuar?\\
145 AH: acción\\
146 D: sí pero a ver, le voy a dibujar una licuadora \# dibuja una licuadora en el pizarrón con algunos ingredientes dentro. cuando usted licúa hay movimiento ¿verdad?\\
147 AH: sí\\
148 D: cuando usted es quien lo hace, cuando usted prende la licadora, se mueve ¿verdad?\\
149 AH: sí cierto\\
150 D: pero sola no hace nada en sí misma ¿verdad?\\
151 AH: no no\\
152 D: la licuadora sí es una cosa, sólo la licuadora, esto \# dibuja un círculo encerrando la licuadora. iría aquí en cosa, pero licuar, el que pueda usted llegar y tocar el botón y empeice a funcionar, esto que es un jitomate un chile y un ajo, se haga un caldo, todo eso ya es licuar pero todo el proceso ¿cierto?\\
153 AH: sí\\
154 D: entonces ¿para licuar qué hace usted? a ver cuénteme\\
155 AH: este acción\\
156 D: sí pero ¿qué hace primero?\\
157 AH: una licuadora\\
158 D: pero usted llega ¿y qué hace con la licuadora?\\
159 AH: este licuando\\
160 D: ajá usted licúa pero\\
161 AH: jitomates\\
162 D: llega y presiona un botón ¿verdad?\\
163 AH: sí cierto\\
164 D: y luego cuando presiona el botón ¿qué pasa?\\
165 AH: ...\\
166 D: ¿qué pasa con el jitomate, en qué se convierte?\\
167 AH: una licuadora licuadora\\
168 D: pero se convierte en una salsa ¿verdad?\\
169 AH: una salsa sí cierto\\
170 D: entonces ¿licuar es una acción o es una cosa?\\
171 AH: acción acción\\
172 D: porque hay todo un proceso que usted hace ¿verdad?\\
173 AH: sí sí\\
174 D: ¿y la licuadora qué es?\\
175 AH: cosa cosa\\
176 D: muy bien muy bien, a ver entonces ¿licuar? \# le da el marcador rojo para que escriba.\\
177 \# Condición: escritura.\\
178 AH: licuar \# escribe «licur». \\
179 D: no ¿qué nos falta?\\
180 AH: la a\\
181 D: sí muy bien \# borra la letra r.\\
182 AH: /a/ \# escribe «Ar» resultando en «licuAr».\\
183 D: muy bien ¿y licuadora?\\
184 AH: li \# escribe «Lic». cua \# escribe la letra u.\\
185 D: licúa ¿qué sigue?\\
186 AH: ¿a? /a/ \# escribe la letra A.\\
187 D: do\\
188 AH: do \# escribe «Do». dora \# escribe «ra» resultando en «LicuADora».\\
189 \# Condición: conversacion.\\
190 D: ahora por ejemplo el teléfono ¿qué es una cosa o una acción?\\
191 AH: acción acción\\
192 D: no\\
193 AH: cosa cosa\\
194 D: pero para ser una acción ¿cómo se tendría que llamar?\\
195 AH: el teléfono\\
196 D: por ejemplo si usted se quiere comunicar con el señor R\_ pero el señor R\_ está en su trabajo ¿qué hace?\\
197 AH: teléfono aquí \# con su mano izquierda cierra parcialmente el puño y lo pone al lado de su oreja izquierda. teléfono aquí\\
198 D: si usted quisiera comunicarse con él, agarra su teléfono, marca un número\\
199 AH: sí\\
200 D: ¿y cómo se le llama a eso?\\
201 AH: R\_ \# menciona el nombre de la persona que se usó de ejemplo.\\
202 D: lla\\
203 AH: ¿qué?\\
204 D: llamar\\
205 AH: llamar ah cierto\\
206 D: entonces llamar y teléfono ¿cuál es cuál, cuál es una cosa y cuál es la acción?\\
207 AH: llamar\\
208 D: ¿llamar cuál es acción o cosa?\\
209 AH: acción acción\\
210 D: ajá muy bien llamar\\
211 AH: llamar lla ...\\
212 D: ¿se acuerda de estas? \# escribe «ll». o bueno en su defecto de esta \# escribe la letra y.\\
213 \# Condición: escritura.\\
214 AH: sí \# escribe «llaMar» en la columa de acción.\\
215 D: y aquí teléfono \# señala la columa de cosa.\\
216 AH: te \# escribe «TeL» le tele ¿ele?\\
217 D: la ele ya la tiene, es esta ¿no? tele\\
218 AH: sí ¿e?\\
219 D: le /l/\\
220 AH: ¿ele?\\
221 D: la ele ya la tiene, esta es la ele \# remarca con el marcador la letra L escrita por AH. a ver ¿le serviría fingir que la escribe en su palma?\\
222 AH: \# aproxima el marcador a su palma izquierda.\\
223 D: a ver si quiere con su dedito, te le \# le muestra su palma derecha.\\
224 AH: te \# con su dedo índice derecho comienza a hacer movimientos sobre su palma izquierda dibujando «Te». le \# finaliza el movimiento sobre su palma con «Le» dando como resulado «TeLe». ah\\
225 D: ajá muy bien \# le da el marcador rojo.\\
226 AH: \# completa el escrito en el pizarrón escribiendo la letra E resultando en «TeLE».\\
227 D: telefo\\
228 AH: fo \# escribe «fo». ele fo\\
229 D: ajá ¿qué sigue?\\
230 AH: no \# escribe la letra r.\\
231 D: no no a ver léamelo\\
232 AH: tele fo no \# señala la letra r. pérame(espérame) \# usa el borrador para borrar la letra r. tele fo /n/ ¿ene? \# escribe la letra N.\\
233 D: sí muy bien\\
234 AH: fo \# escribe la letra o resultando en «TeLEfoNo» pero vuelve a escribir abajo otra secuencia «NO».\\
235 D: este ya no pero todo lo demás está bien\\
236 \\
237 \# Segmento: Dibujo de sustantivos al dictado.\\
238 \# Material: Hoja blanca tamaño carta en horizontal, lápiz y goma de borrar.\\
239 D: a ver aquí me va a dibujar un teléfono \# señala la hoja blanca. puede ser uno de casa o un teléfono celular\\
240 \# Condición: dibujo.\\
241 \# sobre la mesa se encuentra un teléfono celular con forma rectangular.\\
242 AH: \# dibuja un rectángulo comenzando por dos líneas paralelas rectas, una izquierda y otra derecha, dibuja una línea horizontal para conectar ambas líneas por la parte inferior y finaliza repitiendo esta acción en la parte superior. ¿tele, efe?\\
243 D: no no no, pero dibujar\\
244 AC: ((de dibujo)) un teléfono un celular, dibuja un celular\\
245 D: ¿este qué es? \# señala el rectángulo dibujado por AH.\\
246 AH: el el\\
247 D: ¿el teléfono?\\
248 AH: teléfono\\
249 D: okey ¿qué tienen los teléfonos, tienen botones?\\
250 AH: los botones\\
251 D: a ver dibújele los botones, por ejemplo este tiene un botón por aquí \# le muestra el teléfono celular y los botones laterales.\\
252 AH: ah sí \# dibuja del lado superior izquierdo del rectángulo 2 líneas horizontales diminutas.\\
253 D: ¿y por ejemplo cuando usted llama a alguien qué hace con el teléfono?\\
254 AH: ...\\
255 D: y por ejemplo aquí \# le muestra un teclado en la pantalla del celular. ¿qué puede teclear?\\
256 AH: teclear uno dos tres\\
257 D: ¿qué son todos estos? \# señala la pantalla del teléfono.\\
258 AH: el teléfono\\
259 D: ajá ¿pero estos qué son?\\
260 AH: dos\\
261 AC: nu\\
262 AH: números los números\\
263 D: ajá muy bien\\
264 AH: los números\\
265 D: usted cuando le va a llamar a alguien teclea el número del teléfono ¿verdad?\\
266 AH: sí sí cierto\\
267 D: a ver dibújeme aquí los números \# señala el dibujo de AH.\\
268 AH: \# dibuja dentro del rectángulo comenzando en la parte superior 9 puntos en filas de 3.\\
269 \# Condición: conversación.\\
270 D: muy bien, esto solito así solito ¿qué es? \# señala el dibujo.\\
271 AH: el teléfono\\
272 D: ajá ¿y el teléfono qué es, una cosa o una acción?\\
273 AH: una cosa\\
274 D: muy bien, y por ejemplo si usted llega con el teléfono, marca por ejemplo y ya llama, a todo eso ¿cómo se le llama?\\
275 AH: la cosa\\
276 D: no no a todo eso se le llama llamarle ¿no? por ejemplo le voy a llamar a R\_\\
277 AH: ah sí llamar\\
278 D: y todo eso que pasa en el tiempo que usted hace, a lo que se le llama llamar ¿qué es, es una acción o es una cosa?\\
279 AH: una cosa\\
280 D: no\\
281 AH: una \# voltea a ver el pizarrón. acción\\
282 D: una acción \# lo dice al mismo tiempo que AH. porque acuérdese que las acciones implican por ejemplo, generalmente pero no siempre, que haya una persona\\
283 AH: ah pues sí cierto\\
284 D: una persona que por ejemplo levante el borrador o que le llame a alguien o que escriba en la libreta\\
285 AH: sí\\
286 D: bueno intentémoslo otra vez, llamar que es todo el proceso ¿ese proceso requiere tiempo? para que yo llamé, llegue, tome el teléfono\\
287 AH: tres minutos\\
288 D: sí puede tardar tres minutos\\
289 AH: sí sí\\
290 D: entonces si requiere tiempo ¿qué es?\\
291 AH: acción\\
292 D: entonces aquí \# señala a un lado del dibujo de AH. dibújese usted usando el teléfono para llamarle a alguien\\
293 \# Condición: dibujo.\\
294 AH: \# dibuja un círculo y dentro de él tres puntos, 2 alineados de forma horizontal y un tercero en medio de ambos desplazado hacia abajo, agrega una línea curva abajo de este último, representando quizás ojos, nariz y boca respectivamente.\\
295 D: okey muy bien\\
296 AH: \# dibuja tres formas curvas irregulares comenzando en el lado derecho externo del círculo, continuando hacia la parte superior  finalmente hacia la izquierda (posible representación de pelo y/o orejas), agrega dos líneas paralelas en la parte inferior externa del círculo (posible representación del cuello) y complementa cada una con trazos rectos para formar rectángulos pequeños (posible representación de los brazos), finalmente agrega otras dos líneas rectas paralelas de los vértices centrales inferiores de estos dos rectángulos (posible representación del torso o piernas). ¿las manos?\\
297 D: por ejemplo ¿qué tiene que hacer con al menos una de las manos? agrrar el teléfono ¿verdad?\\
298 AH: el teléfono sí\\
299 D: entonces una de las manos tiene que agarrar el teléfono \# señala el rectángulo izquierdo de la figura recién dibujada.\\
300 AH: ah sí \# bibuja cuatro líneas rectas horizontales a los lados externos de los dos rectángulos pequeños (posible representación de los dedos de las manos), se rie.\\
301 DT: muy bien sí se acordó de la clase de dibujo\\
302 AH: sí cierto ajá \# agrega dos figuras circulares en los extremos inferiores de las dos lineas paralelas dibujadas anteriomente (posible representación de los pies).\\
303 D: a ver escriba llamé a mi esposo con el teléfono\\
304 \# Condición: ecritura.\\
305 AH: ¿lla?\\
306 D: sí llamé\\
307 AH: ¿llamé, lla?\\
308 D: ¿qué sonido quedamos? \# señala la palabra «llamar» escrita en el pizarrón.\\
309 AH: ah sí \# escribe «ll».\\
310 D: ¿cómo suena ese sonido? /ll/\\
311 AH: lla\\
312 D: sí sí pero haga sólo el sonido /ll/\\
313 AH: /ll/ \# al mismo tiempo que D.\\
314 D: ajá muy bien, llamé\\
315 AH: lla ¿lla?\\
316 D: lla\\
317 AH: lla lla \# escribe la letra a.\\
318 D: llamé\\
319 AH: me llamé \# escribe la letra m.\\
320 D: ¿qué nos falta?\\
321 AH: llamé \# escribe la letra e (en cursiva) resultando en «llame».\\
322 D: a \# señala debajo de lo que AH acaba de escribir.\\
323 AH: a \# escribe la letra a.\\
324 D: mi\\
325 AH: a /m/ \# alarga el sonido /m/ y escribe la letra m. a mi \# escribe la letra i resultando en «a mi».\\
326 D: esposo\\
327 AH: esposo a mi es \# escribe la letra E. es es \# escribe la letra s (en cursiva) seguida de «Po». po so \# alarga el sonido de la /s/ y conluye escribiendo «so» resultando en «EsPoso»\\
328 D: con mi teléfono \# señala abajo de lo que acaba de escribir AH.\\
329 AH: ¿con?\\
330 D: con mi teléfono\\
331 AH: con \# escribe «Con» (en cursiva). /m/ mi \# escribe la letra m. con mi \# escribe la letra i resultando en «Con mi».\\
332 D: teléfono\\
333 AH: tel mi te \# escribe la letra t. te \# escribe la letra e. le \#escribe la letra L. tele, te /l/ e \# alarga el sonido de la /l/. ¿o?\\
334 D: a ver lea qué dice\\
335 AH: con mi te /l/ e\\
336 D: pero aquí sólo dice tel ¿y qué queremos que diga? te /l/ e \# hace énfasis en la separación del sonido /l/ y /e/.\\
337 AH: ...\\
338 D: a ver ¿se acuerda cuando dibujábamos las letras en el aire para hacer las palabras?\\
339 AH: sí\\
340 D: a ver dibújeme primero te\\
341 AH: te \# con su dedo índice traza sobre la hoja la letra t.\\
342 D: ajá pero también con la siguiente letra, porque si no esta sólo sonaría ¿cómo suena esta? /t/\\
343 AH: mi\\
344 D: esta \# señala la letra t. esta solita sólo suena /t/ pero ¿qué le tenemos que pegar para que diga te\\
345 AH: te \# traza en el aire con su dedo índice la letra e.\\
346 D: muy bien, ahora usted con su dedito \# cubre con las manos la frase escrita para evitar que AH lo copie.\\
347 AH: te \# traza con su dedo índice «te». le \# traza la letra l. tele \# finaliza con la letra e resultando en «tele».\\
348 D: muy bien ahí ya sabe qué letra es la que sigue ¿no?\\
349 AH: con mi te le \# escribe la letra e. fo \# escribe «fo» en cursiva. telefo /n/ \# escribe la letra n. no \# finaliza con la letra o resultando en «teLefono».\\
350 \\
351 \# Segmento: reconocimiento de categorías de palabras: cosas y acciones.\\
352 D: muy bien, ahora con rojo me va a dónde está la cosa, sólo la cosa\\
353 AH: tele fono\\
354 D: sólo el teléfono\\
355 AH: teléfono \# con el marcador rojo subraya la palabra teléfono.\\
356 D: ajá esa es la cosa, y ahora con azul me va a señalar dónde está el verbo, la acción \# le da el marcador azul.\\
357 AH: a mi, a mi\\
358 D: ¿cuál?\\
359 AH: a mi\\
360 D: no no la acción\\
361 AH: esposo\\
362 D: no a ver tranquila \# toma el marcador azul. dígame de qué color es su cubrebocas\\
363 AH: roja ra\\
364 D: ro\\
365 AH: rosa\\
366 D: quedamos que las palabras eran llamar y teléfono, esta es la cosa \# señala en el pizarrón la palabra teléfono.\\
367 AH: sí\\
368 D: ¿y entonces cuál será la acción?\\
369 AH: llamé\\
370 D: llamar bueno sí llamar o llamé\\
371 AH: sí \# subraya con el marcador azul la palabra «llame» escrita en la hoja blanca.\\
372 D: muy bien, ahora ayer barrí mi casa \# le da el lápiz para que escriba la oración.\\
373 \# Condición: escritura.\\
374 AH: a \# escribe la letra a, voltea a ver a D.\\
375 D: ayer ¿se acuerda cual era el sonido /ll/ que acabamos de usar hace rato? en esta por ejemplo \# señala la palabra «llame» escrita en la hoja blanca. /ll/ \# hace énfasis en el sonido de la letra «y» en «ayer».\\
376 AH: ¿elle? aquí aquí \# escribe «ll».\\
377 D: puede ser aquí\\
378 AH: ayer \# termina de escribir la palabra resultando en «aller».\\
379 D: ayer barrí\\
380 AH: ba ba \# escribe «Ba». rri \# escribe «ri» resultando en «Bari». ah no barrí \# borra la letra i para sustituirla por «ri» resultando en «Barri».\\
381 D: mi casa\\
382 AH: mi casa, mi \# escribe «mi» seguido. ca \# escribe «ca» seguido de «sa» resultando en « mi casa». sa.\\
383 D: ¿con qué barre, qué usa usted para barrer?\\
384 AH: ah barre\\
385 D: ¿pero qué usa?\\
386 AH: \# hace un gesto con ambas manos, mueve ambas de forma alternada hacia afuera y adentro de su torso en un eje horizontal (posible gesto de barrer).\\
387 D: ¿cómo se le llama a eso?\\
388 DT: la cosa\\
389 D: ajá la cosa\\
390 AH: la cosa\\
391 D: ajá ¿cómo se llama la cosa con la que barre?\\
392 AH: aquí \# repite el gesto anterior con ambas manos.\\
393 D: ¿pero cómo se le llama a esto? \# dibuja en el pizarrón una escoba.\\
394 AH: ...\# se mantiene en silencio.\\
395 D: ¿con esto barre?\\
396 AH: ajá sí cierto\\
397 D: ¿se acuerda cómo se llama?\\
398 AH: no me acuerdo\\
399 D: empieza con e\\
400 AH: ...\\
401 D: es \# alarga el sonido de la /s/.\\
402 AH: escoba escoba\\
403 D: muy bien muy bien, ayer barrí mi casa con la escoba, con \# señala la hoja para indicar a AH que continúe con la escritura de la oración.\\
404 AH: \# escribe «Con». con\\
405 D: la\\
406 AH: \# escribe «La».\\
407 D: escoba\\
408 AH: es \# escribe «Es». co \# escribe «co». escoba\\
409 D: /b/\\
410 AH: esco \# escribe la letra B. ba \# escribe la letra a resultando en «EscoBa». escoba\\
411 \\
412 \# Segmento: continuación de reconocimiento de categorías de palabras: cosas y acciones.\\
413 D: ¿cuál es la acción aquí? \# señala la oración «ayer barrí mi casa con la escoba».\\
414 AH: ...\\
415 D: léame la oración en voz alta\\
416 AH: ayer barrió barrí mi casa con la escoba\\
417 D: ¿cuál es la acción? acuérdese que generalmente la acción alguien la hace, por ejemplo usted o yo o su esposo ¿verdad?\\
418 AH: sí\\
419 D: y se acuerda que la acción se lleva a cabo en el tiempo ¿verdad? pasa tiempo mientras usted hace todas esas cosas\\
420 AH: sí\\
421 D: ¿cuál es la acción?\\
422 AH: la escoba\\
423 D: no, esa es la cosa\\
424 AH: la cosa, ah ayer\\
425 D: ayer es el día, ¿pero la acción?\\
426 AH: barrí\\
427 D: sí sí es esa, señale con este color barrí \# le da el marcador azul.\\
428 AH: \# subraya con el marcador la palabra indicada.\\
429 D: y ahora la cosa con la que usted barre\\
430 AH: escoba\\
431 D: muy bien, con este \# le da el marcador rojo.\\
432 AH: \# subraya con rojo la palabra indicada.\\
433 D: ¿le está costando trabajo?\\
434 AH: no no\\
435 D: ¿segura?\\
436 AH: sí\\
437 D: entonces vamos a hacer uno más\\
438 \\
439 \# Segmento: escritura al dictado y continuación de reconocimiento.\\
440 \# Condición: conversación.\\
441 D: ¿hoy cocinó?\\
442 AH: no\\
443 D: ¿usa mucho la licuadora para coinar?\\
444 AH: sí\\
445 D: ¿qué licúa en la licuadora?\\
446 AH: ayer estaba mole de olla, la del mole de olla, mole de olla\\
447 D: ¿licuó qué chilitos?\\
448 AH: mole de olla \# interrumpe a D.\\
449 D: ¿pasilla?\\
450 AH: ma mole de olla\\
451 D: ¿sí?\\
452 AH: carne, con carne\\
453 D: muy bien ¿usó su licuadora?\\
454 AH: sí sí\\
455 D: a ver escriba en la semana\\
456 \# Condición: escritura.\\
457 AH: \# escribe «EN». en la en la \# escribe «La».\\
458 D: semana\\
459 AH: semana se ma na \# escribe conforme dice las sílabas «SeMaNa».\\
460 D: muy bien, licué\\
461 AH: licué li\\
462 D: licué\\
463 AH: li li \# sobre su palma izquierda hace un trazo con el lápiz (irreconocible) y después escribe en la hoja «fi» (en cursivas), posteriormente D nota el error de sustitución y se corrige. li\\
464 D: cu\\
465 AH: cu \# escribe la letra c. licua, li\\
466 D: ajá\\
467 AH: cua\\
468 D: licué, licué\\
469 AH: ¿licué?\\
470 D: ajá\\
471 AH: licué, a ver licué, li \# escribe la letra u. cué\\
472 D: ¿qué nos falta?\\
473 AH: licué, licué\\
474 D: a ver hasta aquí ¿qué dice? \# señala el incio de la palabra que AH escribe.\\
475 AH: licué\\
476 D: dice licu ¿qué nos falta, si aquí dice licu?\\
477 AH: cua\\
478 D: licu e \# hace énfasis en la separación del sonido /e/.\\
479 AH: licue \# escribe la letra E resultando en «ficuE».\\
480 D: ¿qué licuó para hacer su mole de olla, qué metió a la licuadora?\\
481 AH: este, una licua, una ¿qué? una\\
482 D: ¿metió cebolla?\\
483 AH: cebolla \# asiete.\\
484 D: ¿qué más metió?\\
485 AH: cebolla, ajo o ajos, ajos\\
486 D: ¿qué más?\\
487 AH: y verdura, verdura verdura\\
488 D: okey, licué las verduras\\
489 AH: ¿las verduras? \# escribe «Las». las ve, las ve \# escribe «Be». ver \# escribe la letra N. ver ver du \# escribe «du». ras \# escribe «ras» resultando en «BeNduras».\\
490 D: para el caldo de olla \# no nota el error de sustitución de r por N.\\
491 AH: ¿cuándo?\\
492 D: para el caldito de olla\\
493 AH: ¿cómo?\\
494 D: porque dijo que hizo de comer caldito de olla ¿no?\\
495 AH: sí\\
496 D: para\\
497 AH: para \# escribe la letra P. pa \# escribe la letra a. para \# escribe lo que parece ser la letra N.\\
498 D: ¿qué letra es esta? \# señala la letra que AH acaba de escribir.\\
499 AH: erre\\
500 D: si es erre está bien, para\\
501 AH: pa\\
502 D: para\\
503 AH: ¿a?\\
504 D: sí muy bien\\
505 AH: \# escribe la letra A resultando en «PaNA» (N reconocida como /rr/ por AH).\\
506 D: el caldo de olla\\
507 AH: e \# escribe la letra E. el \# finaliza con la letra L resultando en «EL».\\
508 D: el caldo\\
509 AH: caldo \# escribe la letra c. el cal \# escribe «aL». el cal\\
510 D: do\\
511 AH: el cal do \# escribe «do» resultando en «caLdo».\\
512 D: de\\
513 AH: de olla, de \# escribe «de». de de o \# escribe la letra o. lla, o\\
514 D: ¿qué sonido quedamos que \# es interrunpida por AH.\\
515 AH: o\\
516 D: no no no, a ver, /o/ ya está, el siguiente sonido es /ll/ ¿cuál es? ya lo usamos varias veces hoy\\
517 AH: ...\\
518 D: \# señala las letras «ll» escritas (no se aprecia donde exactamente).\\
519 AH: ah sí \# escribe «ll».\\
520 D: ¿cómo suena? /ll/ e\\
521 AH: elle\\
522 D: pero sólo este sonido ¿cómo suena? /ll/\\
523 AH: lle \# no se aprecia bien el sonido (falla de origen).\\
524 D: okey, de olla\\
525 AH: o, lla \# finaliza escribiendo la letra a resultando en «olla».\\
526 D: ajá \# toma la hoja para leer la producción escrita de AH. el fin de semana licué las verduras para el caldo de olla \# termina la lectura y dicta la parte faltante, le devuelve la hoja a AH. en la licuadora.\\
527 AH: en la, en en \# escribe «EN». en la li, en la \# escribe «La». li, en la li li \# escribe «Li». cua cua \# escribe «cu». do li, li cuado do \# escribe «do». ra dora \# finaliza con «ra» resultando en «Licudora».\\
528 D: ¿qué nos falta aquí? leáme toda esta palabra \# señala la última palabra escrita por AH.\\
529 AH: en la li, a a \# reconoce el error y escribe la letra A resultando en «LicuAdora».\\
530 D: muy bien muy bien\\
531 DT: muy bien\\
532 D: a ver otra vez vamos a intentarlo, yo sé que es un poquito difícil\\
533 \\
534 \# Segmento: reconocimiento de categorías de palabras: cosas y acciones.\\
535 D: ¿cuál es la cosa? léame toda la oración primero, léame su oración \# señala la oración previamente escrita por AH «en la semana licué las verduras ...» \\
536 \# Condición: lectura.\\
537 AH: en la, en la, en la semana las las verduras \# omite por completo la última palabra.\\
538 D: ¿y aquí qué dice? \# señala la palabra escrita «ficuE».\\
539 AH: fui, fui\\
540 D: no a ver\\
541 AH: en la semana fui\\
542 D: no, esta no es una efe\\
543 AH: fui \# señala la letra f de «ficuE».\\
544 D: ¿aquí qué dice?\\
545 AH: fui\\
546 D: si esta es una efe entonces hay que cambiarla por una ele, para que diga licué\\
547 AH: li \# escribe «Li». cue li ¿u?\\
548 D: ajá\\
549 AH: licué \# escribe la letra u.\\
550 D: licué \# hace énfasis en el sonido de la /e/.\\
551 AH: licué \# finaliza con la letra E resultando en «LicuE».\\
552 D: este lo vamos a tachar \# raya con pluma la palabra errónea «ficuE».\\
553 AH: sí\\
554 D: bueno, ahora sí léame\\
555 AH: en la semana licué las verduras, pa para el calso caldo de ollo de olla en la licuadora\\
556 D: muy bien ¿la licuadora qué es, es una cosa o es una acción?\\
557 AH: una cosa\\
558 D: okey muy bien\\
559 AH: una cosa\\
560 D: muy bien, ahora dibújeme aquí una licuadora chiquita\\
561 \# Condición: dibujo.\\
562 AH: \# dibuja dos lineas rectas paralelas verticales que une con una curva en la parte superior, y por la parte inferior la une con otra línea recta.\\
563 D: ¿dónde están los botones, o es de perillita su licuadora?\\
564 AH: perillas\\
565 D: ¿es de perilla?\\
566 AH: las perillas\\
567 D: ¿gira?\\
568 AH: sí\\
569 D: ah bueno dibuje su perilla.\\
570 AH: \# dentro de la figura en el costado izquerdo dibuja una columna de 3 puntos.\\
571 D: okey ¿y dónde está el vaso de la licuadora?\\
572 AH: aquí \# del lado derecho dibuja una línea curva externa (posible representación del asa del vaso de la licuadora).\\
573 DT: ah muy bien ¿el asa, no?\\
574 AH: asa sí\\
575 D: ah el asa\\
576 AH: sí el asa\\
577 D: dibújeme la tapa de la licuadora ¿dónde está su tapita?\\
578 AH: \# dibuja una línea recta en la parte superior dividiendo la primer línea curva que dibuja.\\
579 D: bueno ahí está y dice que metió ¿verduras, no?\\
580 AH: sí\\
581 D: a ver dibújeme una poquitas verduras aquí dentro \# señala el interior de la figura.\\
582 AH: \# dibuja dos líneas curvas paralelas horizontales y un círculo debajo.\\
583 D: ¿y estas qué son?\\
584 AH: este, la, el, tapa no ¿la qué? los chiles\\
585 D: muy bien, esta es la licuadora ¿y qué me dijo que era, cosa o acción?\\
586 \# Condición: reconocimiento.\\
587 AH: acción\\
588 D: no\\
589 AH: eh no, este, cosa cosa cosa\\
590 D: ¿y el verbo o la acción? ahora sí, léamela otra vez y dígame dónde está el verbo o la acción\\
591 AH: \# lee en voz baja la oración «en la semana licué». licué \# logra identificar correctamente la acción.\\
592 D: muy bien ¿me subraya este mismo? \# le señala la palabra «LicuE» y le da el marcador azul.\\
593 AH: \# hace lo indicado.\\
594 D: muy bien, y ahora la cosa\\
595 AH: las las\\
596 D: no no, la cosa la cosa, usted ya me dijo cuál era la cosa ¿dónde está?\\
597 AH: las verduras\\
598 D: sí son una cosa, aunque no son la que estamos buscando, si quiere puede señalas las verduras, las verduras sí son una cosa\\
599 AH: \# subraya verduras con el marcador rojo.\\
600 D: ¿qué otra cosa encuentra en su oración?\\
601 AH: el caldo\\
602 D: el caldo también es una cosa, muy bien\\
603 DT: ajá muy bien\\
604 AH: el caldo \# subraya ambas palabras.\\
605 D: ¿qué otra cosa hay?\\
606 AH: licuadora\\
607 D: muy bien muy bien\\
608 AH: \# subraya la palabra.\\
609 \# Condición: orientación temporal.\\
610 D: muy bien, poquito a poquito, por cierto ni le pregunté la fecha ¿qué día es hoy?\\
611 AH: este, nueve nueve\\
612 D: ¿de qué mes?\\
613 AH: de agosto\\
614 D: ¿de qué año?\\
615 AH: dos mil veinticuatro\\
616 D: ¿y ayer?\\
617 AH: ocho\\
618 D: ¿y mañana?\\
619 AH: diez\\
620 D: muy bien muy bien\\
