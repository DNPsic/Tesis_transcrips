Sesión del 09 de agosto de 2024.
ID: AH-0908-2024.
Participantes:
AH: Mujer de 65 años, informante.
D: Daniela Salinas, entrevistadora. 
DT: Dante Nava, entrevistador.
AC: Hombre, hijo de AH.

001 # Segmento: inicio de sesion
002 # Condición: lectura.
003 # Material: cuaderno de AH, pizarrón blanco y marcadores.
004 # AH lee el contenido escrito en su cuaderno previamente
005 AH: yo estaba en soriana a las doce y media con mi hija
006 D: muy bien
007 AH: en el refigrerador la ca la crema, pan español, yo yo, y aceite y dos leches y los atunes
008 D: ahora explíqueme sus dibujos
009 AH: leche atún crema aceite # sus dibujos son figuras simples con las palabras escritas dentro, señala cada y las lee.
010 D: de acuerdo y ¿dibujó cuando estaba usted en soriana?
011 AH: sí
012 D: ¿dónde?
013 AH: aquí # señala la oración que leyó al inicio.
014 D: pero el dibujo
015 AH: # ve a D sin dar respuesta.
016 D: dibújeme un pez # le da el marcador rojo de pizarrón
017 AH: ¿un pez?
018 D: un pez
019 AC: un pececito
020 AH: un pez # dibujo un óvalo con una líea pequeña en la parte inferior
021 D: ¿le falta algo a ese pez?
022 AH: las # dibuja dos lineas más desde los lados del óvalo, una del lado izquierdo y otra en el derecho.
023 D: sus aletitas ¿y su colita?
024 AH: colita # dibuja dos puntos dentr del óvalo.
025 D: muy bien a parte de esto me dijo que hizo más ¿no?
026 AH: a ver
027 D: ¿esta no? # le muestra otra hoja de su cuaderno
028 AH: a ver
029 # Escrito: la hoja muestra en la parte superior «8 de agosto fecha 2024», a continuación se lee (primer párrafo) «los regalos de los sicologos y las playeras del doctor alejandro» (segundo párrafo) «la doctora Daniela el doctor Guillermo», el uso de minúsculas y mayúsculas es inconsistente a diferencia de lo representadoa aquí.
030 # Escrito: la hoja muestra en la parte superior «viernes 9  de Agosto 2024», a continuación (primer párrafo) «ayer fui a soriana a comprar sandia y tortillas», (segundo párrafo) «yo vi la televicion», finalmente se observan 3 dibujos y debajo de cada uno las palabras «television, sandia, tortillas», el primer dibujo es un rectángulo con 4 puntos en su parte superior, el segundo es una forma ovalada con 6 puntos en su interior, el trecero son 6 puntos alineados verticalmente y en dos columnas.
031 D: muy bien, léame esta primera oración # señala la primera oración.
032 # Condición: lectura.
033 AH: fuia a soriana a comprar sandía y tortillas
034 D: muy bien
035 AH: yo yo va yo vi vi la televisión
036 D: muy bien dígame aquí cuál es el verbo # señala la primera oración.
037 AH: fui
038 D: ¿qué otro verbo hay?
039 AH: comprar
040 D: ¿y en esta otra oración cuál es el verbo? # señala la segunda oración.
041 AH: yo
042 D: no, acuérdese que los verbos son acciones
043 AH: yo vi, vi vi
044 D: muy bien, por ejemplo en la primera oración usted ya me dijo que el verbo es fui
045 AH: ajá sí
046 D: y también comprar mientras que en la segunda es vi
047 AH: ah sí
048 D: pero por ejemplo dígame cosas que podemos encontrar en esa oración, cosas no verbos
049 AH: fui
050 D: no ese quedamos que era el verbo, quedamos que las cosas no sienten, por ejemplo si yo le pego a esto # da dos golpes al borrador del pizarrón. no siente
051 AH: ajá
052 D: sólo si alguien más los mueve, generalmente no tienen intenciones, o no se mueve solo
053 AH: ajá
054 D: por ejemplo si yo dibujo # dibuja una rebanada de una sandía en el pizarrón.
055 AH: ah sandía
056 D: ajá muy bien ¿esto qué es una cosa o es un verbo?
057 AH: este cosa
058 D: ¿la sandía siente?
059 AH: no no
060 D: muy bien y por ejemplo si yo dibujo # dibuja un rectangulo y unas figuras humanas dentro, al lado un óvalo con círculos de diferentes tamaños. imagínese que esto es una televisión ¿esto sería una cosa o una acción?
061 AH: no, una acción
062 D: ¿una acción? por qué
063 AH: la televisión estaba bien
064 D: pero usted el ver, que usted esté por ejemplo # dibuja una figura humana al lado del dibujo anterior. imagínese que aquí está usted, y está observando la televisión # dibuja una flecha desde la cabeza de la figura humana hasta el dibujo de la televisión. pero usted está viéndolo, usted experimenta usted sí siente
065 AH: sí
066 D: usted percibe las imágenes ¿de acuerdo?
067 AH: cosa las cosas
068 DT: ajá muy bien
069 D: sí, usted percibe las cosas, usted por ejemplo puede ver las hojas
070 AH: sí sí
071 D: la bolsa, mis bibujos
072 AH: sí
073 D: pero quien siente es usted
074 AH: ajá
075 D: y usted lo que está haciendo es ver
076 AH: ver
077 D: ese sí es el verbo, que usted experimente todo eso es el verbo
078 AH: ah
079 D: pero esto # señala el dibujo de la televisión. sólo es una cosa sólo existe, y no hace nada más ni siente nada más
080 AH: sí
081 D: entonces la televisión ¿es una cosa o es un verbo?
082 AH: una cosa
083 D: muy bien, y por ejemplo ¿ese reloj? # señala el reloj de pared.
084 AH: una cosa
085 D: ¿y sus lentes?
086 AH: mis lentes
087 D: ¿son una cosa o son una acción?
088 AH: una cosa una cosa
089 D: y por ejemplo cuando sus niñas están jugando, jugando ¿jugar es una acción o es una cosa?
090 AH: acción acción
091 D: y por ejemplo cuando sus hijos están trabajando
092 AH: acción
093 D: ¿y la puerta, la puerta qué es?
094 AH: cosa cosa
095 D: muy bien a ver un último ejemplo, la silla ¿es una cosa o una acción?
096 AH: cosa
097 D: ahora yo le voy a decir unas palabras y usted las va a escribir
098 AH: sí
099 
100 # Segmento: clasificación de cosas y acciones al dictado.
101 # Material: pizarrón y marcador rojo. en el pizarrón se observan dos columnas, en la izquierda dice acción y en la derecha cosa.
102 D: el papel de baño ¿qué es una cosa o una acción?
103 AH: cosa
104 D: a ver escriba ahí papel de baño
105 AH: ¿pa?
106 D: papel
107 # Condición: escritura.
108 AH: pa # escribe «Pa». pa # escribe la letra P.
109 D: pape
110 AH: ¿ele?
111 D: no no aquí leame qué dice # señala lo que AH acaba de escribir.
112 AH: pa
113 D: pap hasta acá dice pap, pape # hace énfasis en el sonido de la e
114 AH: ¿ele?
115 D: no, antes de la ele, falta un sonido pe /p-e/ # hace énfases en los dos sonidos.
116 AH: ah pe
117 D: pero la letra pe ya está aquí ¿cuál es siguiente sonido? /p/ es este sonido # señala la última letra P escrita. y el siguiente /ee/
118 AH: ah ((enrique)) # escribe la letra E.
119 D: ahora sí ¿qué nos falta?
120 AH: ele
121 D: ajá muy bien, ¿papel de?
122 AH: higiénico
123 D: de
124 AH: de # escribe «de» con letra cursiva.
125 D: muy bien, papel de baño
126 AH: ba # escribe «va»
127 D: con esa la podemos dejar de momento, baño
128 AH: ¿eñe?
129 D: sí muy bien
130 AH: # escribe la letra Ñ
131 D: ¿qué nos falta?
132 AH: /o/ # escribe la letra o
133 # Condición: conversacion.
134 DT: muy bien
135 D: ahora por ejemplo, caminar ¿caminar es una cosa o una acción?
136 AH: acción acción
137 DT: muy bien
138 D: sí muy bien, por ejemplo cuando caminamos hay movimiento, escriba caminar
139 AH: caminar # escribe «camiNar» en la columa de acción, la escribe mientras dice en voz alta la palabra separando por sílabas ca-mi-nar.
140 # Condición: conversacion
141 D: muy bien ahora por ejemplo licuar ¿licuar qué es una acción o es una cosa?
142 AH: cosa
143 D: ¿licuar?
144 AH: acción
145 D: sí pero a ver, le voy a dibujar una licuadora # dibuja una licuadora en el pizarrón con algunos ingredientes dentro. cuando usted licua hay movimiento ¿verdad?
146 AH: sí
147 D: cuando usted es quien lo hace, cuando usted prende la licadora, se mueve ¿verdad?
148 AH: sí cierto
149 D: pero sola no hace nada en sí misma ¿verdad?
150 AH: no no
151 D: la licuadora sí es una cosa, sólo la licuadora, esto # dibuja un círculo encerrando la licuadora. iría aquí en cosa, pero licuar, el que pueda usted llegar y tocar el botón y empeice a funcionar, esto que es un jitomate un chile y un ajo, se haga un caldo, todo eso ya es licuar pero todo el proceso ¿cierto?
152 AH: sí
153 D: entonces ¿para licuar qué hace usted? a ver cuénteme
154 AH: este acción
155 D: sí pero ¿qué hace primero?
156 AH: una licuadora
157 D: pero usted llega ¿y qué hace con la licuadora?
158 AH: este licuando
159 D: ajá usted licua pero
160 AH: jitomates
161 D: llega y presiona un botón ¿verdad?
162 AH: sí cierto
163 D: y luego cuando presiona el botón ¿qué pasa?
164 AH: ...
165 D: ¿qué pasa con el jitomate, en qué se convierte?
166 AH: una licuadora licuadora
167 D: pero se convierte en una salsa ¿verdad?
168 AH: una salsa sí cierto
169 D: entonces ¿licuar es una acción o es una cosa?
170 AH: acción acción
171 D: porque hay todo un proceso que usted hace ¿verdad?
172 AH: sí sí
173 D: ¿y la licuadora qué es?
174 AH: cosa cosa
175 D: muy bien muy bien, a ver entonces ¿licuar? # le da el marcador rojo para que escriba.
176 # Condición: escritura.
177 AH: licuar # escribe «licur». 
178 D: no ¿qué nos falta?
179 AH: la a
180 D: sí muy bien # borra la letra r.
181 AH: /a/ # escribe «Ar» resultando en «licuAr»
182 D: muy bien ¿y licuadora?
183 AH: li # escribe «Lic». cua # escribe la letra u
184 D: licua ¿qué sigue?
185 AH: ¿a? /a/ # escribe la letra A
186 D: do
187 AH: do # escribe «Do» dora # escribe «ra» resultando en «LicuADora»
188 # Condición: conversacion
189 D: ahora por ejemplo el teléfono ¿qué es una cosa o una acción?
190 AH: acción acción
191 D: no
192 AH: cosa cosa
193 D: pero para ser una acción ¿cómo se tendría que llamar?
194 AH: el teléfono
195 D: por ejemplo si usted se quiere comunicar con el señor R_ pero el señor R_ está en su trabajo ¿qué hace?
196 AH: teléfono aquí # con su mano izquierda cierra parcialmente el puño y lo pone al lado de su oreja izquierda. teléfono aquí
197 D: si usted quisiera comunicarse con él, agarra su teléfono, marca un número
198 AH: sí
199 D: ¿y cómo se le llama a eso?
200 AH: R_ # menciona el nombre de la persona que se usó de ejemplo.
201 D: lla
202 AH: ¿qué?
203 D: llamar
204 AH: llamar ah cierto
205 D: entonces llamar y teléfono ¿cuál es cuál, cuál es una cosa y cuál es la acción?
206 AH: llamar
207 D: ¿llamar cuál es acción o cosa?
208 AH: acción acción
209 D: ajá muy bien llamar
210 AH: llamar lla ...
211 D: ¿se acuerda de estas? # escribe «ll». o bueno en su defecto de esta # escribe la letra y.
212 # Condición: escritura.
213 AH: sí # escribe «llaMar» en la columa de acción.
214 D: y aquí teléfono # señala la columa de cosa.
215 AH: te # escribe «TeL» le tele ¿ele?
216 D: la ele ya la tiene, es esta ¿no? tele
217 AH: sí ¿e?
218 D: le /l/
219 AH: ¿ele?
220 D: la ele ya la tiene, esta es la ele # remarca con el marcador la letra L escrita por AH. a ver le serviría fingir que la escribe en su palma?
221 AH: # aproxima el marcador a su palma izquierda.
222 D: a ver si quiere con su dedito, te le # se muestra su palma derecha.
223 AH: te # con su dedo índice derecho comienza a hacer movimientos sobre su palma izquierda dibujando «Te». le # finaliza el movimiento sobre su palma con «Le» dando como resulado «TeLe» ah
224 D: ajá muy bien # le da el marcador rojo.
225 AH: # completa el escrito en el pizarrón escribiendo la letra E resultando en «TeLE».
226 D: telefo
227 AH: fo # escribe «fo». ele fo
228 D: ajá ¿qué sigue?
229 AH: no # escribe la letra r.
230 D: no no a ver léamelo
231 AH: tele fo no # señala la letra r. pérame(espérame) # usa el borrador para borrar la letra r. tele fo /n/ ¿ene? # escribe la letra N.
232 D: sí muy bien
233 AH: fo # escribe la letra o resultando en «TeLEfoNo» pero vuelve a escribir abajo otra secuencia «NO».
234 D: este ya no pero todo lo demás está bien
235 
236 # Segmento: Dibujo de sustantivos al dictado.
237 # Material: Hoja blanca tamaño carta en horizontal, lápiz y goma de borrar.
238 D: a ver aquí me va a dibujar un teléfono # señala la hoja blanca. puede ser uno de casa o un teléfono celular
239 # Condición: dibujo.
240 # sobre la mesa se encuentra un teléfono celular con forma rectangular.
241 AH: # dibuja un rectángulo comenzando por dos líneas paralelas rectas, una izquierda y otra derecha, dibuja una línea horizontal para conectar ambas líneas por la parte inferior y finaliza repitiendo esta acción en la parte superior. ¿tele, efe?
242 D: no no no, pero dibujar
243 AC: ((de dibujo)) un teléfono un celular, dibuja un celular
244 D: ¿este qué es? # señala el rectángulo dibujado por AH.
245 AH: el el
246 D: ¿el teléfono?
247 AH: teléfono
248 D: okey ¿qué tienen los teléfonos, tienen botones?
249 AH: los botones
250 D: a ver dibújele los botones, por ejemplo este tiene un botón por aquí # le muestra el teléfono celular y los botones laterales.
251 AH: ah sí # dibuja del lado superior izquierdo del rectángulo 2 líneas horizontales diminutas
252 D: ¿y por ejemplo cuando usted llama a alguien qué hace con el teléfono?
253 AH: ...
254 D: y por ejemplo aquí # le muestra un teclado en la pantalla del celular. ¿qué puede teclear?
255 AH: teclear uno dos tres
256 D: ¿qué son todos estos? # señala la pantalla del teléfono.
257 AH: el teléfono
258 D: ajá ¿pero estos qué son?
259 AH: dos
260 AC: nu
261 AH: números los números
262 D: ajá muy bien
263 AH: los números
264 D: usted cuando le va a llamar a alguien teclea el número del teléfono ¿verdad?
265 AH: sí sí cierto
266 D: a ver dibújeme aquí los números # señala el dibujo de AH.
267 AH: # dibuja dentro del rectángulo comenzando en la parte superior 9 puntos en filas de 3.
268 # Condición: conversación.
269 D: muy bien, esto solito así solito ¿qué es? # señala el dibujo
270 AH: el teléfono
271 D: ajá ¿y el teléfono qué es, una cosa o una acción?
272 AH: una cosa
273 D: muy bien, y por ejemplo si usted llega con el teléfono, marca por ejemplo y ya llama, a todo eso ¿cómo se le llama?
274 AH: la cosa
275 D: no no a todo eso se le llama llamarle ¿no? por ejemplo le voy a llamar a R_
276 AH: ah sí llamar
277 D: y todo eso que pasa en el tiempo que usted hace, a lo que se le llama llamar ¿qué es, es una acción o es una cosa?
278 AH: una cosa
279 D: no
280 AH: una # voltea a ver el pizarrón. acción
281 D: una acción # lo dice al mismo tiempo que AH. porque acuérdese que las acciones implican por ejemplo, generalmente pero no siempre, que haya una persona
282 AH: ah pues sí cierto
283 D: una persona que por ejemplo levante el borrador o que le llamé a alguien o que escriba en la libreta
284 AH: sí
285 D: bueno intentémoslo otra vez, llamar que es todo el proceso ¿ese proceso requiere tiempo? para que yo llamé, llegue, tome el teléfono
286 AH: tres minutos
287 D: sí puede tardar tres minutos
288 AH: sí sí
289 D: entonces si requiere tiempo ¿qué es?
290 AH: acción
291 D: entonces aquí # señala a un lado del dibujo de AH. dibújese usted usando el teléfono para llamarle a alguien
292 # Condición: dibujo.
293 AH: # dibuja un círculo y dentro de él tres puntos, 2 alineados de forma horizontal y un tercero en medio de ambos desplazado hacia abajo, agrega una línea curva abajo de este último, representando quizás ojos, nariz y boca respectivamente.
294 D: okey muy bien
295 AH: # dibuja tres formas curvas irregulares comenzando en el lado derecho externo del círculo, continuando hacia la parte superior  finalmente hacia la izquierda (posible representación de pelo y/o orejas), agrega dos líneas paralelas en la parte inferior externa del círculo (posible representación del cuello) y complementa cada una con trazos rectos para formar rectángulos pequeños (posible representación de los brazos), finalmente agrega otras dos líneas rectas paralelas de los vértices centrales inferiores de estos dos rectángulos (posible representación del torso o piernas). ¿las manos?
296 D: por ejemplo ¿qué tiene que hacer con al menos una de las manos? agrrar el teléfono ¿verdad?
297 AH: el teléfono sí
298 D: entonces una de las manos tiene que agarrar el teléfono # señala el rectángulo izquierdo de la figura recién dibujada.
299 AH: ah sí # bibuja cuatro líneas rectas horizontales a los lados externos de los dos rectángulos pequeños (posible representación de los dedos de las manos), se rie.
300 DT: muy bien sí se acordó de la clase de dibujo
301 AH: sí cierto ajá # agrega dos figuras circulares en los extremos inferiores de las dos lineas paralelas dibujadas anteriomente (posible representación de los pies).
302 D: a ver escriba llamé a mi esposo con el teléfono
303 # Condición: ecritura.
304 AH: ¿lla?
305 D: sí llamé
306 AH: ¿llamé, lla?
307 D: ¿qué sonido quedamos? # señala la palabra «llamar» escrita en el pizarrón.
308 AH: ah sí # escribe «ll»
309 D: ¿cómo suena ese sonido? /ll/
310 AH: lla
311 D: sí sí pero haga sólo el sonido /ll/
312 AH: /ll/ # al mismo tiempo que D.
313 D: ajá muy bien, llamé
314 AH: lla ¿lla?
315 D: lla
316 AH: lla lla # escribe la letra a.
317 D: llamé
318 AH: me llamé # escribe la letra m.
319 D: ¿qué nos falta?
320 AH: llamé # escribe la letra e (en cursiva) resultando en «llame»
321 D: a # señala debajo de lo que AH acaba de escribir.
322 AH: a # escribe la letra a.
323 D: mi
324 AH: a /m/ # alarga el sonido /m/ y escribe la letra m. a mi # escribe la letra i resultando en «a mi»
325 D: esposo
326 AH: esposo a mi es # escribe la letra E. es es # escribe la letra s (en cursiva) seguida de «Po». po so # alarga el sonido de la /s/ y conluye escribiendo «so» resultando en «EsPoso»
327 D: con mi teléfono # señala abajo de lo que acaba de escribir AH.
328 AH: ¿con?
329 D: con mi teléfono
330 AH: con # escribe «Con» (en cursiva). /m/ mi # escribe la letra m. con mi # escribe la letra i resultando en «Con mi».
331 D: teléfono
332 AH: tel mi te # escribe la letra t. te # escribe la letra e. le #escribe la letra L. tele, te /l/ e # alarga el sonido de la /l/. ¿o?
333 D: a ver lea qué dice
334 AH: con mi te /l/ e
335 D: pero aquí sólo dice tel ¿y qué queremos que diga? te /l/ e # hace énfasis en la separación del sonido /l/ y /e/.
336 AH: ...
337 D: a ver ¿se acuerda cuando dibujábamos las letras en el aire para hacer las palabras?
338 AH: sí
339 D: a ver dibújeme primero te
340 AH: te # con su dedo índice traza sobre la hoja la letra t.
341 D: ajá pero también con la siguiente letra, porque si no esta sólo sonaría ¿cómo suena esta? /t/
342 AH: mi
343 D: esta # señala la letra t. esta solita sólo suena /t/ pero ¿qué le tenemos que pegar para que diga te
344 AH: te # traza en el aire con su dedo índice la letra e.
345 D: muy bien, ahora usted con su dedito # cubre con las manos la frase escrita para evitar que AH lo copie.
346 AH: te # traza con su dedo índice «te». le # traza la letra l. tele # finaliza con la letra e resultando en «tele».
347 D: muy bien ahí ya sabe qué letra es la que sigue ¿no?
348 AH: con mi te le # escribe la letra e. fo # escribe «fo» en cursiva. telefo /n/ # escribe la letra n. no # finaliza con la letra o resultando en «teLefono».
349 
350 # Segmento: reconocimiento de categorías de palabras: cosas y acciones.
351 D: muy bien, ahora con rojo me va a dónde está la cosa, sólo la cosa
352 AH: tele fono
353 D: sólo el teléfono
354 AH: teléfono # con el marcador rojo subraya la palabra teléfono.
355 D: ajá esa es la cosa, y ahora con azul me va a señalar dónde está el verbo, la acción # le da el marcador azul.
356 AH: a mi, a mi
357 D: ¿cuál?
358 AH: a mi
359 D: no no la acción
360 AH: esposo
361 D: no a ver tranquila # toma el marcador azul. dígame de qué color es su cubrebocas
362 AH: roja ra
363 D: ro
364 AH: rosa
365 D: quedamos que las palabras eran llamar y teléfono, esta es la cosa # señala en el pizarrón la palabra teléfono.
366 AH: sí
367 D: ¿y entonces cuál será la acción?
368 AH: llamé
369 D: llamar bueno sí llamar o llamé
370 AH: sí # subraya con el marcador azul la palabra «llame» escrita en la hoja blanca.
371 D: muy bien, ahora ayer barrí mi casa # le da el lápiz para que escriba la oración.
372 # Condición: escritura.
373 AH: a # escribe la letra a, voltea a ver a D.
374 D: ayer ¿se acuerda cual era el sonido /ll/ que acabamos de usar hace rato? en esta por ejemplo # señala la palabra «llame» escrita en la hoja blanca. /ll/ # hace énfasis en el sonido de la letra «y» en «ayer».
375 AH: ¿elle? aquí aquí # escribe «ll»
376 D: puede ser aquí
377 AH: ayer # termina de escribir la palabra resultando en «aller»
378 D: ayer barrí
379 AH: ba ba # escribe «Ba». rri # escribe «ri» resultando en «Bari». ah no barrí # borra la letra i para sustituirla por «ri» resultando en «Barri».
380 D: mi casa
381 AH: mi casa, mi # escribe «mi» seguido. ca # escribe «ca» seguido de «sa» resultando en « mi casa». sa.
382 D: ¿con qué barre, qué usa usted para barrer?
383 AH: ah barre
384 D: ¿pero qué usa?
385 AH: # hace un gesto con ambas manos, mueve ambas de forma alternada hacia afuera y adentro de su torso en un eje horizontal (posible gesto de barrer).
386 D: ¿cómo se le llama a eso?
387 DT: la cosa
388 D: ajá la cosa
389 AH: la cosa
390 D: ajá ¿cómo se llama la cosa con la que barre?
391 AH: aquí # repite el gesto anterior con ambas manos.
392 D: ¿pero cómo se le llama a esto? # dibuja en el pizarrón una escoba.
393 AH: ...# se mantiene en silencio.
394 D: ¿con esto barre?
395 AH: ajá sí cierto
396 D: ¿se acuerda cómo se llama?
397 AH: no me acuerdo
398 D: empieza con e
399 AH: ...
400 D: es # alarga el sonido de la /s/.
401 AH: escoba escoba
402 D: muy bien muy bien, ayer barrí mi casa con la escoba, con # señala la hoja para indicar a AH que continúe con la escritura de la oración.
403 AH: # escribe «Con». con
404 D: la
405 AH: # escribe «La».
406 D: escoba
407 AH: es # escribe «Es». co # escribe «co». escoba
408 D: /b/
409 AH: esco # escribe la letra B. ba # escribe la letra a resultando en «EscoBa». escoba
410 
411 # Segmento: continuación de reconocimiento de categorías de palabras: cosas y acciones.
412 D: ¿cuál es la acción aquí? # señala la oración «ayer barrí mi casa con la escoba».
413 AH: ...
414 D: léame la oración en voz alta
415 AH: ayer barrió barrí mi casa con la escoba
416 D: ¿cuál es la acción? acuérdese que generalmente la acción alguien la hace, por ejemplo usted o yo o su esposo ¿verdad?
417 AH: sí
418 D: y se acuerda que la acción se lleva a cabo en el tiempo ¿verdad? pasa tiempo mientras usted hace todas esas cosas
419 AH: sí
420 D: ¿cuál es la acción?
421 AH: la escoba
422 D: no, esa es la cosa
423 AH: la cosa, ah ayer
424 D: ayer es el día, ¿pero la acción?
425 AH: barrí
426 D: sí sí es esa, señale con este color barrí # le da el marcador azul.
427 AH: # subraya con el marcador la palabra indicada.
428 D: y ahora la cosa con la que usted barre
429 AH: escoba
430 D: muy bien, con este # le da el marcador rojo.
431 AH: # subraya con rojo la palabra indicada
432 D: ¿le está costando trabajo?
433 AH: no no
434 D: ¿segura?
435 AH: sí
436 D: entonces vamos a hacer uno más
437 
438 # Segmento: escritura al dictado y continuación de reconocimiento.
439 # Condición: conversación.
440 D: ¿hoy cocinó?
441 AH: no
442 D: ¿usa mucho la licuadora para coinar?
443 AH: sí
444 D: ¿qué licua en la licuadora?
445 AH: ayer estaba mole de olla, la del mole de olla, mole de olla
446 D: ¿licuó qué chilitos?
447 AH: mole de olla # interrumpe a D.
448 D: ¿pasilla?
449 AH: ma mole de olla
450 D: ¿sí?
451 AH: carne, con carne
452 D: muy bien ¿usó su licuadora?
453 AH: sí sí
454 D: a ver escriba en la semana
455 # Condición: escritura.
456 AH: # escribe «EN». en la en la # escribe «La»
457 D: semana
458 AH: semana se ma na # escribe conforme dice las sílabas «SeMaNa».
459 D: muy bien, licué
460 AH: licué li
461 D: licué
462 AH: li li # sobre su palma izquierda hace un trazo con el lápiz (irreconocible) y después escribe en la hoja «fi» (en cursivas), más adelanté D nota el error de sustitución y se corrige. li
463 D: cu
464 AH: cu # escribe la letra c. licua, li
465 D: ajá
466 AH: cua
467 D: licué, licué
468 AH: ¿licué?
469 D: ajá
470 AH: licué, a ver licué, li # escribe la letra u. cué
471 D: ¿qué nos falta?
472 AH: licué, licué
473 D: a ver hasta aquí ¿qué dice? # señala el incio de la palabra que AH escribe.
474 AH: licué
475 D: dice licu ¿qué nos falta, si aquí dice licu?
476 AH: cua
477 D: licu e # hace énfasis en la separación del sonido /e/
478 AH: licue # escribe la letra E resultando en «ficuE»
479 D: ¿qué licuó para hacer su mole de olla, qué metió a la licuadora?
480 AH: este, una licua, una ¿qué? una
481 D: ¿metió cebolla?
482 AH: cebolla # asiete.
483 D: ¿qué más metió?
484 AH: cebolla, ajo o ajos, ajos
485 D: ¿qué más?
486 AH: y verdura, verdura verdura
487 D: okey, licué las verduras
488 AH: ¿las verduras? # escribe «Las». las ve, las ve # escribe «Be». ver # escribe la letra N. ver ver du # escribe «du». ras # escribe «ras» resultando en «BeNduras».
489 D: para el caldo de olla # no nota el error de sustitución de r por N.
490 AH: ¿cuándo?
491 D: para el caldito de olla
492 AH: ¿cómo?
493 D: porque dijo que hizo de comer caldito de olla ¿no?
494 AH: sí
495 D: para
496 AH: para # escribe la letra P. pa # escribe la letra a. para # escribe lo que parece ser la letra N.
497 D: ¿qué letra es esta? # señala la letra que AH acaba de escribir.
498 AH: erre
499 D: si es erre está bien, para
500 AH: pa
501 D: para
502 AH: ¿a?
503 D: sí muy bien
504 AH: # escribe la letra A resultando en «PaNA» (N reconocida como /rr/ por AH).
505 D: el caldo de olla
506 AH: e # escribe la letra E. el # finaliza con la letra L resultando en «EL».
507 D: el caldo
508 AH: caldo # escribe la letra c. el cal # escribe «aL». el cal
509 D: do
510 AH: el cal do # escribe «do» resultando en «caLdo»
511 D: de
512 AH: de olla, de # escribe «de». de de o # escribe la letra o. lla, o
513 D: ¿qué sonido quedamos que # es interrunpida por AH.
514 AH: o
515 D: no no no, a ver, /o/ ya está, el siguiente sonido es /ll/ ¿cuál es? ya lo usamos varias veces hoy
516 AH: ...
517 D: # señala las letras «ll» escritas (no se aprecia donde exactamente).
518 AH: ah sí # escribe «ll».
519 D: ¿cómo suena? /ll/ e   
520 AH: elle
521 D: pero sólo este sonido ¿cómo suena? /ll/
522 AH: lle # no se aprecia bien el sonido (falla de origen).
523 D: oker, de olla
524 AH: o, lla # finaliza escribiendo la letra a resultando en «olla».
525 D: ajá # toma la hoja para leer la producción escrita de AH. el fin de semana licué las verduras para el caldo de olla # termina la lectura y dicta la parte faltante, le regres la hoja a AH. en la licuadora.
526 AH: en la, en en # escribe «EN» en la li, en la # escribe «La». li, en la li li # escribe «Li». cua cua # escribe «cu». do li, li cuado do # escribe «do» ra, dora # finaliza con «ra» resultando en «Licudora».
527 D: ¿qué nos falta aquí? leáme toda esta palabra # señala la última palabra escrita por AH.
528 AH: en la li, a a # reconoce el error y escribe la letra A resultando en «LicuAdora».
529 D: muy bien muy bien
530 DT: muy bien
531 D: a ver otra vez vamos a intentarlo, yo sé que es un poquito difícil
532 
533 # Segmento: reconocimiento de categorías de palabras: cosas y acciones.
534 D: ¿cuál es la cosa? léame toda la oración primero, léame su oración # señala la oración previamente escrita por AH «en la semana licué las verduras ...» 
535 # Condición: lectura.
536 AH: en la, en la, en la semana las las verduras # omite por completo la última palabra.
537 D: ¿y aquí qué dice? # señala la palabra escrita «ficuE».
538 AH: fui, fui
539 D: no a ver
540 AH: en la semana fui
541 D: no, esta no es una efe
542 AH: fui # señala la letra f de «ficuE».
543 D: ¿aquí qué dice?
544 AH: fui
545 D: si esta es una efe entonces hay que cambiarla por una ele, para que diga licué
546 AH: li # escribe «Li». cue li ¿u?
547 D: ajá
548 AH: licué # escribe la letra u.
549 D: licué # hace énfasis en el sonido de la /e/.
550 AH: licué # finaliza con la letra E resultando en «LicuE»
551 D: este lo vamos a tachar # raya con pluma la palabra errónea «ficuE».
552 AH: sí
553 D: bueno, ahora sí léame
554 AH: en la semana licué las verduras, pa para el calso caldo de ollo de olla en la licuadora
555 D: muy bien ¿la licuadora qué es, es una cosa o es una acción?
556 AH: una cosa
557 D: okey muy bien
558 AH: una cosa
559 D: muy bien, ahora dibújeme aquí una licuadora chiquita
560 # Condición: dibujo.
561 AH: # dibuja dos lineas rectas paralelas verticales que une con una curva en la parte superior, y por la parte inferior la une con otra línea recta.
562 D: ¿dónde están los botones, o es de perillita su licuadora?
563 AH: perillas
564 D: ¿es de perilla?
565 AH: las perillas
566 D: ¿gira?
567 AH: sí
568 D: ah bueno dibuje su perilla.
569 AH: # dentro de la figura en el costado izquerdo dibuja una columna de 3 puntos.
570 D: okey ¿y dónde está el vaso de la licuadora?
571 AH: aquí # del lado derecho dibuja una línea curva externa (posible representación del asa del vaso de la licuadora)
572 DT: ah muy bien ¿el asa, no?
573 AH: asa sí
574 D: ah el asa
575 AH: sí el asa
576 D: dibújeme la tapa de la licuadora ¿dónde está su tapita?
577 AH: # dibuja una línea recta en la parte superior dividiendo la primer línea curva que dibuja.
578 D: bueno ahí está y dice que metió ¿verduras, no?
579 AH: sí
580 D: a ver dibújeme una poquitas verduras aquí dentro # señala el interior de la figura.
581 AH: # dibuja dos líneas curvas paralelas horizontales y un círculo debajo.
582 D: ¿y estas qué son?
583 AH: este, la, el, tapa no ¿la qué? los chiles
584 D: muy bien, esta es la licuadora ¿y qué me dijo que era, cosa o acción?
585 # Condición: reconocimiento.
586 AH: acción
587 D: no
588 AH: eh no, este, cosa cosa cosa
589 D: ¿y el verbo o la acción? ahora sí, léamela otra vez y dígame dónde está el verbo o la acción
590 AH: # lee en voz baja la oración «en la semana licué» licué # logra identificar correctamente la acción.
591 D: muy bien ¿me subraya este mismo? # le señala la palabra «LicuE» y le da el marcador azul.
592 AH: # hace lo indicado.
593 D: muy bien, y ahora la cosa
594 AH: las las
595 D: no no, la cosa la cosa, usted ya me dijo cuál era la cosa ¿dónde está?
596 AH: las verduras
597 D: sí son una cosa, aunque no son la que estamos buscando, si quiere puede señalas las verduras, las verduras sí son una cosa
598 AH: # subraya verduras con el marcador rojo.
599 D: ¿qué otra cosa encuentra en su oración?
600 AH: el caldo
601 D: el caldo también es una cosa, muy bien
602 DT: ajá muy bien
603 AH: el caldo # subraya ambas palabras.
604 D: ¿qué otra cosa hay?
605 AH: licuadora
606 D: muy bien muy bien
607 AH: # subraya la palabra.
608 # Condición: orientación temporal.
609 D: muy bien, poquito a poquito, por cierto ni le pregunté la fecha ¿qué día es hoy?
610 AH: este, nueve nueve
611 D: ¿de qué mes?
612 AH: de agosto
613 D: ¿de qué año?
614 AH: dos mil veinticuatro
615 D: ¿y ayer?
616 AH: ocho
617 D: ¿y mañana?
618 AH: diez
619 D: muy bien muy bien
