\section{Sesión del 22 de agosto del 2024}
\noindent
ID: AH-2208-2024\\
Participantes:
AH: Mujer de 65 años, informante.\\
D: Daniela Salinas, entrevistadora.\\
DT: Dante Nava, entrevistador.\\
AC: Mujer, familiar de AH.\\
\\

\noindent
001 \# Segmento: preparación de los materiales.\\
002 \# Condición: conversación.\\
003 \# la grabación inicia en medio de una conversación entre AH y D.\\
004 ...\\
005 D: ¿nada?\\
006 AH: no\\
007 D: ¿entonces a dónde le gustaba salir?\\
008 AH: con mi esposo\\
009 D: ¿pero a dónde?\\
010 AH: a nadar\\
011 D: ¿a dónde sí le gustaba?\\
012 AH: pero este, nadar nada\\
013 D: nadar nada, ¿pero a dónde sí?\\
014 AH: este, bailado bailando también\\
015 D: ¿iban a bailar?\\
016 AH: sí sí, en Acapulco también\\
017 D: ahí sí le gustaba\\
018 AH: sí\\
019 D: le gustaba la fiesta\\
020 AH: con Veracruz también, Veracruz también sí con Veracruz\\
021 DT: ¿qué le gusta más Veracruz o Acapulco?\\
022 AH: Acapulco\\
023 DT: muy bien\\
024 \\
025 \# Segmento: orientación temporal.\\
026 \# Condición: conversación.\\
027 DT: ¿qué fecha es hoy?\\
028 AH: veintidos \\
029 D: ¿veintidos de qué mes?\\
030 AH: de octubre\\
031 DT: no\\
032 D: ¿segura?\\
033 AH: veintidos de pérame(espérame) veintidos de agosto, veinte de agosto veinte de agosto\\
034 DT: ¿de qué año?\\
035 AH: dos mil veinticuatro\\
036 DT: ¿y qué día de la semana es?\\
037 AH: jueves jueves\\
038 DT: ¿y cómo que hora será?\\
039 AH: pus(pues) no sé hombre\\
040 D: ¿aproximadamente?\\
041 DT: a ver más o menos\\
042 AH: a las once y media o las diez y media, diez y veinte, diez y veinte\\
043 DT: casi casi\\
044 D: muy bien\\
045 AH: diez y veinte\\
046 D: diez y cuarto\\
047 AH: ah mira\\
048 DT: por cinco minutos\\
049 AH: sí cierto\\
050 \# Segmento: series automatizadas e inversas.\\
051 DT: muy bien, a ver dígame los meses del año\\
052 AH: enero febrero marzo abril mayo junio julio agosto septiembre octubre noviembre diciembre\\
053 DT: ahora al revés\\
054 AH: diciembre, noviembre, este noviembre \# repite en voz baja y cerrando los ojos. /s/ agosto ah no septiembre ago septiembre, pérame(espérame)\\
055 DT: sí sí\\
056 AH: septiembre agosto, di este agosto, y agosto \# repite en voz baja con los ojos cerrados. no no\\
057 DT: está bien no pasa nada, a ver, ahora cuente del quince al treinta\\
058 AH: \# se aproxima a DT con duda.\\
059 DT: del quince\\
060 AH: del quince\\
061 DT: al treinta\\
062 AH: eh ¿cuánto?\\
063 DT: a ver, quince\\
064 AH: quince\\
065 DT: dieciséis\\
066 AH: dieciséis diecisiete, dieciocho diecinueve, \# los siguientes números los dice con mayor velocidad y omite los número entre 26 y 30. veinte veintiuno veintidos veintitrés veinticuatro veinticinco treinta \# se ríe.\\
067 DT: bueno ahí se salto unos\\
068 AH: sí cierto\\
069 D: bueno está bien, muy bien\\
070 \\
071 \# Segmento: reporte de actividades.\\
072 \# Condición: conversación.\\
073 DT: platíqueme qué hizo, ¿qué le dejamos de tarea?\\
074 AH: este, las palabras\\
075 DT: a ver ¿podemos verlo?\\
076 AH: acción, cosa\\
077 DT: a ver ¿qué puso en acción? a ver si se acuerda\\
078 AH: este, cama\\
079 DT: a ver \# le devuelve la libreta a AH donde realiza sus actividades en casa. a ver, acción, acá estamos \# señala la columna que tiene escritas varias acciones en la libreta.\\
080 \# Condición: lectura.\\
081 \# Escrito: En la libreta de AH se aprecia una cabecera con la fecha: «fecha MiERoLeS 21 de agosto de 2024». Posteriormente se observan dos columnas, la de la izquierda se titulo «accion» y la derecha «Cosa». Debajo de la primera comlumna se observan las palabras: «TraPEAr, Comer, NadAr, Barer, PiNTar, Cocer, Hacer, ReMEMENdAR, PEiNar». Debajo de la segunda: «sala, tostadas, tocador, silla, Mesa, PoLLO, CoMida, ROPa, MUÑecA»\\
082 AH: acción\\
083 DT: a ver ¿qué puso?\\
084 AH: comer\\
085 DT: ajá\\
086 AH: nadar, ay pérame(espérame)\\
087 DT: ah sus lentes\\
088 AH: ajá cierto\\
089 DT: a ver ¿en acción qué puso?\\
090 AH: ¿acción?\\
091 DT: ajá\\
092 AH: a an, este, a a trapear, comer nadar, ba barrer, pintar, cocer, hice hacer, remendar /.re rei re./ reinar ¿cómo?\\
093 DT: a ver ¿este qué es?\\
094 AH: pei nar ah peinar, tocar, servir\\
095 DT: muy bien, a ver ¿qué es remendar, cómo se hace?\\
096 \# Condición: mímica de acciones.\\
097 AH: así \# con ambas manos toca su playera y la mueve ligeramente de arriba hacia abajo.\\
098 DT: sí muy bien, y por ejemplo  ¿peinar?\\
099 AH: pues ((así)) \# con ambas manos toca su pelo suavemente.\\
100 DT: eso y ¿servir?\\
101 AH: ah pos(pues) la sopa \# con ambas manos cerradas en puños, alterna una y otra adelante y atrás.\\
102 DT: la sopa muy bien\\
103 AH: sí\\
104 DT: a ver ahora ¿de cosas? a ver si se acuerda qué escribió\\
105 AH: a ver\\
106 DT: estas que me dijo son acciones\\
107 AH: sí sí\\
108 DT: y a ver ahora cosas\\
109 AH: este, cosas, leer\\
110 DT: no, esa es acción\\
111 AH: ajá\\
112 DT: a ver \# le muestra el escrito en su libreta.\\
113 AH: sala, ah\\
114 DT: cosas, estas, las cosas usted no las hace\\
115 AH: no no no\\
116 DT: ¿ya vio?\\
117 AH: sí cierto\\
118 DT: a ver, vamos a hacer otro ejercicio, a ver ¿una acción?\\
119 \# Condición: reconocimiento de categorías de palabras: acciones y cosas.\\
120 AH: trapear\\
121 DT: muy bien ¿y una cosa?\\
122 AH: este, leer\\
123 DT: esa es una acción\\
124 AH: sí sí sí\\
125 DT: ¿y una cosa?\\
126 AH: ¿cosa?\\
127 DT: sí\\
128 AH: este ...\\
129 DT: esto \# le muestra un libro pequeño.\\
130 AH: ...\\
131 DT: ¿qué es esto?\\
132 AH: lib la libreta\\
133 DT: ¿cómo se llama esto?\\
134 AH: ¿libreta libreta? \# se ríe.\\
135 DT: libro\\
136 AH: libro ajá sí sí\\
137 DT: ¿sí?\\
138 AH: sí sí\\
139 DT: ¿qué hace con un libro?\\
140 AH: este\\
141 DT: la acción\\
142 AH: acción\\
143 DT: ¿qué acción hace con el libro?\\
144 AH: este ...\\
145 DT: a ver, yo lo voy a hacer y usted me va a decir qué es \# comienza a hojear el libro pequeño.\\
146 AH: sí\\
147 DT: ¿qué estoy haciendo? a ver dígame\\
148 AH: leer, leer\\
149 DT: muy bien ¿y qué más hizo de tarea?\\
150 AH: este, aquí estaba en soriana(centro comercial)\\
151 DT: ¿fue al soriana?\\
152 AH: sola sola \# susurrando\\
153 DT: ¿sola?\\
154 AH: sola\\
155 DT: ¿ya la dejaron ir sola?\\
156 AH: sola \# señala a AC.\\
157 AC: sola\\
158 DT: órale\\
159 AC: no muy seguido pero de vez en cuando pues va\\
160 AH: pero \# hace un gesto con la mano derecha de «negación» en repetidas ocasiones.\\
161 DT: no sabe\\
162 AH: no no\\
163 DT: ¿quién sabe entonces?\\
164 AH: \# señala a AC.\\
165 DT: ¿nada más?\\
166 AH: sí sí\\
167 D: ah ¿es secreto?\\
168 AH: sí verdad\\
169 D: está bien pero con mucho cuidado\\
170 DT: sí con mucho cuidado, por los coches y todo eso\\
171 AH: ah sí\\
172 DT: ¿y qué sintió de que fue al soriana?\\
173 AH: ay bien bonito hombre, bien bonito hombre\\
174 DT: ¿ya solita?\\
175 AH: sí\\
176 DT: ¿qué hizo ahí en soriana?\\
177 AH: este, crema, un bolillo\\
178 D: ¿eso compró?\\
179 AH: sí\\
180 D: muy bien ¿qué más compró?\\
181 AH: a un esprai(«spray» para el cabello) \# con la mano derecha se toca el pelo suavemente.\\
182 DT: ah para el cabellito\\
183 AH: sí sí\\
184 D: usted que siempre anda guapa\\
185 AH: sí ay no hombre, bien feo bien fea\\
186 DT: no cómo cree\\
187 D: para nada\\
188 DT: bueno pues nada más con mucho cuidado cuando vaya\\
189 AH: sí\\
190 DT: ¿y no le dan miedo los coches?\\
191 AH: no no\\
192 DT: ¿sí puede sola?\\
193 AH: sí cierto\\
194 DT: ¿y no se tropieza?\\
195 AH: no no, por mis pies, también bien feos, mi pies están bien bonitos \# se ríe.\\
196 DT: ¿y qué más hizo?\\
197 AH: trapié la sala, el pasillo\\
198 D: muy bien\\
199 \\
200 \# Segmento: habla espontánea en conversación.\\
201 \# Material: pizarrón blanco y marcador.\\
202 \# la conversación incia con D preguntando a AH qué más acciones realizó.\\
203 AH: este, sopa también\\
204 D: no pero la sopa no es una acción ¿cuál es la acción?\\
205 AH: este, trapié\\
206 D: ¿qué otra cosa?\\
207 AH: trapié \# susurrando. la sala\\
208 D: trapeó la sala\\
209 DT: a ver, la otra vez nos contó que en su cocina tiene muchas cosas\\
210 AH: ah sí\\
211 DT: ¿sí se acuerda?\\
212 AH: sí sí\\
213 DT: ¿a ver qué tiene en su cocina?\\
214 AH: en la casa \# niega con la cabeza. en mi com ¿en qué?\\
215 DT: la cocina\\
216 AH: la cocina\\
217 DT: ¿qué hay ahí?\\
218 AH: la estufa, la sala \# niega con la cabeza\\
219 DT: ¿la? a ver acuérdese\\
220 AH: el espérame, el (?), la estufa, el ¿qué?\\
221 DT: ajá la estufa\\
222 AH: el el ay hombre\\
223 DT: ahorita se acuerda\\
224 AH: ajá a ver, la estufa y ¿qué es? \# voltea a ver a AC.\\
225 DT: a ver acuérde no haga trampas\\
226 AH: la estufa, ay muchas tra mu/sh/ trastes\\
227 DT: trastes\\
228 AH: muchos trastes, muchos /.trest trestes./\\
229 DT: ¿qué más, dónde guarda la comida?\\
230 AH: en el refrigerador\\
231 DT: en el refrigerador ¿y dónde guarda el aceite y las latas?\\
232 AH: el aceite \# extiende su brazo izquierdo señalando al aire. en la, en la ¿qué es?\\
233 DT: ahí está ya sabe qué es\\
234 AH: en la, ay dios bendito ...\\
235 D: ¿cómo es?\\
236 AH: aquí en la \# extiende ambos brazos con los dedos índices levantados, mueve ambos brazos hacia arriba ligeramente.\\
237 DT: ¿y cómo, se abre así? \# junta ambas manos con los puños cerrados para posteriormente abrirlos con un movimiento.\\
238 AH: sí sí aquí \# hace el mismo movimiento que DT. ajá ay dios bendito\\
239 DT: ¿empieza con /a/? \# alarga ligeramente el sonido.\\
240 AH: /.alacera./ alacena alacena\\
241 DT: muy bien ¿y qué hace usted en la cocina?\\
242 AH: este, la sopa\\
243 DT: ¿qué acciones?\\
244 AH: la sopa, /.pi pi piscadilla pec picadilla./ picadillo, ayer, picadillo\\
245 DT: ¿pero qué hizo con el picadillo?\\
246 AH: este, estaba bien sabrosa, pero bien sabroso, bien sabroso\\
247 DT: ¿y qué hizo con el picadillo, lo aventó?\\
248 AH: no no, en la en la \# junta y separa los puños tres veces.\\
249 DT: ¿lo vendió?\\
250 AH: no hombre no \# se ríe.\\
251 DT: ¿qué hizo entonces? a ver platíqueme\\
252 AH: en la /s/ en las \# junta y separa los puños ligeramente. en la, las cucharas y la cuchara, este la cuch ...\\
253 DT: a ver si hacemos esto \# baja la mano derecha con el puño cerrado para lentamente aproximarlo a su boca, haciendo el gesto de «comer con cuchara».\\
254 AH: la sopa\\
255 DT: pero qué esto haciendo, a ver usted hágalo así \# juntando las puntas de los dedos de la mano derecha, la aproxima a su boca, haciendo el gesto de «comer».\\
256 AH: \# junta las puntas de los dedos de la mano derecha, los coloca sobre su labio inferior, abre la boca, aleja rápidamente la mano y cierra la boca para inmediatamente acerca la mano de nuevoa.\\
257 DT: así \# repite el gesto de «comer». con la tortillita\\
258 AH: comiendo, comiendo\\
259 DT: muy bien, esa es la acción\\
260 AH: ajá comiendo\\
261 DT: ¿qué más?\\
262 AH: y sopa\\
263 DT: ¿pero qué otra acción?\\
264 AH: comiendo\\
265 DT: ya me dijo comiendo, muy bien, vamos a anotarlo acá \# escribe «comiendo» en el pizarrón blanco.\\
266 D: ya dijo comer, trapear\\
267 AH: ajá\\
268 DT: comiendo, trapear \# sigue escribiendo las palabras en el pizarrón.\\
269 AH: ah mira \# se ríe.\\
270 DT: ya nos dijo dos ¿qué más hizo? acciones\\
271 AH: ¿acciones? estaba en la tarea, la tarea, la tarea\\
272 DT: a ver, ahora ya dejamos la cocina, ahora la sala ¿qué hay en la sala, en su sala qué tiene?\\
273 AH: la mus la música, la música\\
274 DT: ¿y qué hace con la música?\\
275 AH: \# con su dedo índice izquierdo señala su oreja izquierda.\\
276 DT: ¿qué es, cómo se llama eso?\\
277 AH: la música estaba bien /s/ ((fuera)) bien fuerte bien fuerte\\
278 DT: ¿pero usted qué estaba haciendo?\\
279 AH: ...\\
280 DT: ¿se come la música?\\
281 AH: no hombre\\
282 DT: ¿entonces qué se hace?\\
283 AH: bai bailando, bailando\\
284 DT: ah muy bien\\
285 AH: bailando sí\\
286 DT: otra acción, ya nos dijo tres \# escribe «bailando» en el pizarrón. una y ya\\
287 AH: ah bueno\\
288 DT: ¿qué tiene en su recámara?\\
289 AH: el tocador, el tocador\\
290 DT: ¿y qué hace?\\
291 AH: estaba en el tocar, estaba en la televisión, en la televisión\\
292 DT: ¿pero qué hace con la televisión?\\
293 AH: estaba bien, bien, la televisión estaba, las noticias, las noticias\\
294 DT: pero usted ¿qué estaba haciendo?\\
295 AH: en la, aquí en la /m/, en la\\
296 DT: ¿estaba bailando con la televisión?\\
297 AH: no no no\\
298 DT: ¿entonces qué hizo?\\
299 AH: \# comienza a reír. espérame, estaba, ay dios bendito\\
300 DT: estaba ¿cocinando la televisión?\\
301 AH: no no hombre\\
302 DT: no se cocina la televisión ¿entonces qué se hace con la televisión?\\
303 AH: estaba en la, ay dios bendito, en las noticias en las noticias\\
304 DT: ¿qué estaba haciendo usted?\\
305 AH: las noticias\\
306 DT: ¿con los ojos?\\
307 AH: no, los los \# se lleva la mano a los labios, con las puntas de los dedos juntas, haciendo el mismo gesto usado previamente para «comer». los oídos, los oídos, con los oídos\\
308 DT: ¿estaba oyendo?\\
309 AH: sí con los oídos\\
310 DT: a ver diga, oyendo\\
311 AH: oí\\
312 DT: ¿oí qué?\\
313 AH: oí\\
314 DT: ¿qué oyó?\\
315 AH: en la televisión, la televisión\\
316 DT: muy bien, vamos a ponerlo aquí \# anota «oí» en el pizarrón. muy bien, cuatro verbos\\
317 \\
318 \# Segmento: conversación sobre estado de ánimo.\\
319 DT: a ver, platíquenos ¿cómo se ha sentido?\\
320 AH: bien gracias, gracias a dios\\
321 DT: ¿ya mejor?\\
322 AH: uy sí, sí\\
323 DT: porque la semana pasada la veíamos\\
324 AH: no pero también con mi nieta \# señala a AC (no es su nieta).\\
325 DT: ¿sí?\\
326 AH: ay sí qué bonita, qué bonita qué bonito, qué bonito\\
327 DT: a ver ¿por qué? cuéntenos\\
328 AH: bien bonita, bien bonita \# señala a AC. mi nieta (?)\\
329 DT: ¿ya se siente mejor?\\
330 AH: ay sí hombre\\
331 DT: sí se nota\\
332 AH: sí\\
333 DT: sí, es que luego la notábamos que venía\\
334 D: muy triste o decaída\\
335 AC: y es que más se levantó, es que bueno, nosotros no vivíamos con ella, y tiene al lado al nieto consentido, al favorito\\
336 DT: ¿sí, su nieto, cómo se llama?\\
337 AH: D\_\\
338 D: ah tu es esposo es su nieto consentido \# dice a AH.\\
339 AH: sí\\
340 D: ah qué bonito\\
341 AC: y de ahí como que se levantó\\
342 D: ¿viven con ella o viven al lado?\\
343 AC: nos mudamos al lado y entonces ahí ya luego le ayudo, luego la veo\\
344 D: ah qué padre\\
345 AC: o luego D\_ está con ella, así como que estamos más ahí\\
346 AH: sí\\
347 D: claro, ahí presentes\\
348 AH: sí dios bendito\\
349 DT: sí se nota que ya viene más\\
350 D: más feliz\\
351 AH: sí hombre bien feliz hombre, con mi nieto mi nieta, mi nieta\\
352 D: ahora ella va a ser la consentida\\
353 AH: ay sí\\
354 DT: a ver vamos a ver, un ejercicio aquí rápido\\
355 \# Condición: habla espontánea.\\
356 AH: bien burlones mis hijos, bien burlones\\
357 DT: ¿siguen sus hijos?\\
358 AH: uy sí\\
359 ... \# se le comenta a AC sobre evitar comentarios inapropiados sobre la condición de AH.\\
360 \# Segmento: reconocimiento de emociones con alternancia semiótica.\\
361 \# Materiales: pizarrón blanco y marcador rojo.\\
362 \# Dibujo: en el pizarrón hay dos representaciones simples de caras, una representa «feliz» y la otra «triste».\\
363 DT: a ver ¿esta qué es? \# señala la cara «feliz» en el pizarrón.\\
364 AH: la boca\\
365 DT: sí pero ¿¿cuál es la emoción?\\
366 AH: emoción, aquí mira, bien feliz\\
367 DT: bien feliz ¿y acá? \# señala la cara «triste» en el pizarrón.\\
368 AH: bien gra, bien este, bien ay bien qué\\
369 DT: esta es feiz ¿y esta? \# señala de nuevo la cara «triste».\\
370 AH: bien bien ¿qué? bien apachur bien pachurrada, bien apachurrada, bien apachurrada\\
371 DT: ¿triste?\\
372 AH: bien triste, bien triste\\
373 \# Condición: copia de dibujos.\\
374 DT: ahora las vas a copiar aquí \# le da una hoja blanca y una pluma azul. a ver ¿cuál va a dibujar primero?\\
375 AH: este \# no se aprecia cuál de los dibujos señala.\\
376 DT: ¿cuál es?\\
377 AH: la cara, la cara feliz\\
378 DT: ah muy bien\\
379 AH: \# copia el dibujo de la cara feliz, sin embargo la boca da dibuja de forma invertida, el dibujo a copiar muestra un tercio de círculo como sonrisa, al copiarlo AH invierte este tercio de círculo, similar de la cara «triste».\\
380 DT: a ver ahora póngale aquí cuál es ¿es la feliz? \# señala con el dedo debajo de la copia de AH.\\
381 AH: feliz sí\\
382 DT: a ver escriba por favor\\
383 AH: fe, fe /l/ liz, feliz \# Escritura: «FeLis».\\
384 DT: muy bien ¿y cuál nos falta?\\
385 AH: bien\\
386 DT: a ver ahora dibúje esta por favor \# señala la cara «triste».\\
387 AH: \# comienza a copiar el dibujo, al llegar a la boca se detiene. ay no bien fea \# finaliza dibujando una línea curva.\\
388 D: le salió bien\\
389 DT: muy bien\\
390 AH: feliz triste, feliz triste\\
391 DT: ah faltó escribir\\
392 AH: triste, triste, tri /s/, tris te \# Escritura: «TrsTE»\\
393 D: ¿qué nos falta ahí?\\
394 AH: tris, tris\\
395 DT: a ver lea la palabra\\
396 AH: ¿e?\\
397 DT: no no no\\
398 AH: tris\\
399 DT: ¿dónde está la /i/ por ejemplo?\\
400 AH: ¿i?\\
401 DT: ajá\\
402 AH: ah, tri, tris \# Escritura: agrega la letra i: «TrisTE».\\
403 DT: ah muy bien muy bien\\
404 D: muy bien\\
405 AH: ah mira\\
406 \\
407 \# Segmento: lectura de un párrafo de un libro infantil, comprensión y recuerdo de la lectura.\\
408 \# Material: libro infantil.\\
409 \# Condición: lectura.\\
410 \# Lectura: «El rey Tulio se sentía muy enfermo y muy triste. Y tenía razón de sentir tanta tristeza. Sus médicos le habían dicho que la única forma en que podría mejorar su salud era que dejara de ser rey».\\
411 DT: por favor léanos los que dice ahí\\
412 AH: el rey, el rey (?) el rey /.cu tusio./ el rey Tulio Tulio, /.sen se sienta./ muy enfermo, muy tristeza muy tre muy triste y tenía razón de sentir tanta tristeza por sus /.meca medidi./ médicos, médicos /.de habla abiar de abi de abiar./ dicho que la /.uni unidad confor conforme en una./ en que podría mejorar su salud, era /s/ que dejara de rey, de de, re reinar.\\
413 \# Lectura: «-Yo no puedo dejar de ser rey- decía Tulio. Así nací. Pero sus médicos insistían».\\
414 AH: yo no puede no pude dejar de ser rey decir /.jud jutu ju ju./ ¿Julio, /.tudio./?\\
415 DT: a ver ¿dónde vamos, acá?\\
416 AH: ajá este, Julio\\
417 DT: a ver vea bien la letra\\
418 AH: decía /.tu Julio./\\
419 DT: a ver la había dicho bien\\
420 AH: Tulio, Tulio\\
421 DT: muy bien\\
422 AH: así así /.nada nada para./ pero sus médicos /.esta en estaban médicos instan instante instiante./\\
423 \# Condición: preguntas sobre la lectura.\\
424 DT: bueno hasta ahí está bien, y bueno, para terminar estar partecita ¿cómo se llamaba el rey?\\
425 AH: el rey chulo \# comienza a reírse. ¿Tulio, el Tulio?\\
426 DT: eso muy bien\\
427 AH: Tulio\\
428 DT: ¿y cómo se sentía?\\
429 AH: bien enfermo, bien enfermo\\
430 DT: ¿y qué carita era la de él? \# señala los dibujos en el pizarrón.\\
431 AH: bien enfermo\\
432 DT: ¿la feliz?\\
433 AH: feliz\\
434 DT: ¿o la triste?\\
435 AH: bien triste, ((pero)) bien triste\\
436 DT: ¿y habló con alguien, el rey habló con alguien?\\
437 AH: no\\
438 DT: ¿alguien le había dicho algo?\\
439 AH: no\\
440 DT: ¿había médicos?\\
441 AH: sí, los médicos los médicos\\
442 DT: ¿le dijeron algo los médicos al rey?\\
443 AH: estaba bien enfermo\\
444 DT: muy bien muy bien\\
445 \\
446 \# Segmento: acciones como procesos.\\
447 \# Materiales: láminas con dibujos que representan acciones.\\
448 D: a ver primero dígame en esta qué es lo que observa, en esta primera\\
449 \# Material: lámina de una niña inflando un globo: 1) el globo está desinflado, 2) el globo está ligeramente inflando, 3) el globo está completamente inflado.\\
450 AH: el cabello\\
451 D: ¿es una niña o es un niño?\\
452 AH: una niña una niña\\
453 D: ¿qué está agarrando con las manos\\
454 AH: globo el globo\\
455 D: muy bien ¿de esta imagen a esta imagen ¿qué cambia? \# señala el primer cuadro de la lámina y posteriormente el segundo.\\
456 AH: el globo\\
457 D: ¿qué tiene el globo?\\
458 AH: \# acerca su mano derecha a la boca y sopla ligeramente.\\
459 D: sí ¿es más grande o más pequeño?\\
460 AH: grande\\
461 D: sí, está más grande ¿qué tal en esta? \# señala el último segmento de la imagen.\\
462 AH: el globo está bien enfermo \# se lleva ambas manos a la boca. estaba bien feo \# se ríe.\\
463 D: okey, ¿cómo de le decimos a eso?\\
464 AH: apachurrado, apachurrado\\
465 D: sí por ejemplo aquí está un poquito apachurrado ¿pero cómo le llamamos a tener un globo, ponerlo en la boca y soplar aire, y qué pasa con el globo?\\
466 AH: bien grande\\
467 D: sí ¿pero cómo le decimos? que el globo se\\
468 AH: se infec\\
469 D: se ¿i?\\
470 AH: se infla, se infla\\
471 D: okey muy bien, infar es un verbo, es un verbo porque las acciones necesitan un proceso ¿se acuerda que una vez de hablé de eso?\\
472 AH: a pos (pues) sí cierto\\
473 D: este es el proceso, el que pase el globo de esto, a esto y luego a esto ¿podemos ver cómo cambia en el tiempo verdad?\\
474 AH: sí sí bien inflado\\
475 D: y por eso es una acción o un verbo\\
476 AH: acciones, acción\\
477 D: ¿entonces cómo quedamos que se llama esta acción, la acción de?\\
478 AH: el globo\\
479 D: ¿pero cuál era la acción? el globo no es la acción, la acción es ¿i?\\
480 AH: inflado inflado\\
481 D: entonces \# escribe en el pizarrón «acción» y «cosa». ¿inflar dónde va?\\
482 AH: cosas\\
483 D: no\\
484 AH: acción\\
485 D: ¿por qué? porque conlleva un proceso, y entonces el globo dónde irá?\\
486 AH: acciones\\
487 D: no\\
488 AH: cosa\\
489 D: le vamos a escribir aquí hasta arriba a todo\\
490 AH: ah inflar\\
491 D: \# escribe en la parte superior de la lámina «inflar».\\
492 AH: ah mira\\
493 D: donde todo esto es inflar, todos estos tres\\
494 AH: ah sí cierto\\
495 D: y si bien se hace la acción con una cosa, que en este caso es el globo, el globo no es la acción ¿verdad?\\
496 AH: no no, cosa\\
497 D: entonces el globo ¿cosa o acción?\\
498 AH: cosa cosa\\
499 D: entonces aquí escríbame globo \# señala la zona del pizarrón donde está escrito «cosa».\\
500 \# Condición: escritura.\\
501 AH: ¿globo?\\
502 D: sí acá\\
503 AH: glo, glo \# Escitura: g.\\
504 D: /g/ /l/\\
505 AH: ¿glo?\\
506 D: ¿este cómo suena? \# señala la letra g.\\
507 AH: ga\\
508 D: ¿cuál es el siguiente sonido?\\
509 AH: glo\\
510 D: glo\\
511 AH: ¿o?\\
512 D: antes de la /o/ nos falta uno /l/, ¿qué sonido es?\\
513 AH: ¿ele?\\
514 D: sí\\
515 AH: ele, ele\\
516 D: a ver escriba aquí una ele \# señala la parte inferior del pizarrón.\\
517 AH: ele, ele \# escribe la letra F. ele\\
518 D: esta suena /f/\\
519 AH: e ¿gato?\\
520 D: no, como en luz\\
521 AH: globo\\
522 D: a ver escríbame aquí luz \# señala una parte vacía del pizarrón.\\
523 AH: luz luz \# Escitura: «Lus».\\
524 D: ¿ya vio que aquí sí pudo escribirla?\\
525 AH: ah\\
526 D: esa es la ele, globo\\
527 AH: \# Escitura: L.\\
528 D: ¿glo?\\
529 AH: ah \# Escitura: o. ¿bo, bo?\\
530 D: sí, bo\\
531 AH: ¿bo?\\
532 D: /b/\\
533 AH: ¿be?\\
534 D: sí\\
535 AH: be, glo, bo, glo, bo\\
536 D: ahora léame qué dice\\
537 AH: globo\\
538 D: no, léalo con cuidado, hay un error y quiero que usted lo identifique\\
539 AH: a ver\\
540 D: ¿cuál es?\\
541 AH: be\\
542 D: ¿dónde está nuestra be?\\
543 AH: \# borra un elemento pero no se aprecia en el video.\\
544 D: /b/\\
545 AH: ¿R\_? \# menciona el nombre de un familiar, compensación que usa para hallar fonemas y grafías.\\
546 D: esa suena /rr/ y nosotros queremos /b/\\
547 AH: ¿be?\\
548 D: como en barco, a ver escríbame aquí barco\\
549 AH: a mira \# Escitura: v. globo \# Escitura final: «glovo»\\
550 D: okey con esa la vamos a dejar porque suenan igual, ahora el siguiente es este \\
551 \# Material: lámina de crecimiento de un árbol 1) un niño coloca semillas en un agujero, 2) un árbol pequeño, 3) un árbol grande.\\
552 D: este es otro verbo, todo esto es un verbo ¿qué está haciendo el niño?\\
553 AH: el árbol estaba, plantiando plantando plantando\\
554 D: muy bien, de aquí podemos sacar dos verbos, uno es este que ya me dijo, plantar ¿si es un verbo qué es, una acción o una cosa?\\
555 AH: cosa, acción acción\\
556 D: entonces ese lo vamos a poner porque usted lo dijo, plantar \# señala debajo de la palabra «acción» escrita en el pizarrón.\\
557 AH: plantar, pla, plan \# Escitura: «PLaN».\\
558 D: /t/\\
559 AH: ¿te?\\
560 D: sí\\
561 AH: ah te, plan tar ¿erre? \# Escitura: «Ta».\\
562 D: muy bien\\
563 AH: ah cierto \# Escitura final: «PLaNTar».\\
564 D: muy bien, le salió muy rápido\\
565 AH: sí\\
566 D: ¿el iguiente cuál es, aquí el arbolito de qué tamaño es?\\
567 AH: pequeñita, pequeñito\\
568 D: ¿y aquí?\\
569 AH: bien grande, bien mediano, mediano mediano\\
570 D: okey aquí es mediano ¿y qué tal aquí?\\
571 AH: el árbol bien grande\\
572 D: entonces ¿a todo eso cómo le llamamos, que el árbol?\\
573 AH: ...\\
574 D: ¿qué pasa con el árbol? primero está así \# señala el primer dibujo de la lámina. ¿luego?\\
575 AH: bien /f/ bien grande, la mitad\\
576 D: mediano ¿no?\\
577 AH: mediano, mediano\\
578 D: ¿y luego?\\
579 AH: bien grande\\
580 D: ¿cómo se le llama a eso, cuál es la acción?\\
581 AH: el árbol está bien fea, bien feo\\
582 D: ¿qué pasa con el árbol?\\
583 AH: bien grande\\
584 D: ¿pero qué pasa, cuál es la acción, el verbo?\\
585 AH: ...\\
586 D: ¿cre?\\
587 AH: crece, crece\\
588 D: muy bien, el árbol crece ¿cierto?\\
589 AH: sí\\
590 D: ¿y eso qué es? acuérdese que las acciones son todo esto\\
591 AH: crece\\
592 D: ¿pero qué es, es una acción o es una cosa?\\
593 AH: acción\\
594 D: muy bien muy bien\\
595 AH: acción\\
596 D: a ver escríbame aquí crecer \# señala una parte en blanco del pizarrón debajo de la palabra «plantar».\\
597 AH: cre cre /s/ \# Escitura: «cr»\\
598 D: ¿cuál es la que sigue?\\
599 AH: sa ser ¿e?\\
600 D: primero cre ¿qué nos falta ahí?\\
601 AH: cre ser\\
602 D: ¿cuál es la siguiente para que diga cre?\\
603 AH: ¿e?\\
604 D: ajá ¿cuál es la e?\\
605 AH: ¿E\_? \# menciona el nombre de otro familiar como compensación. \# Escitura: E.\\
606 D: muy bien, crecer\\
607 AH: /s/ ¿ese?\\
608 D: bueno la vamos a dejar con esa\\
609 AH: crecer crecer \# Escitura final: «crEser»\\
610 D: muy bien muy bien, ya tenemos otro ¿verdad?\\
611 AH: sí cierto\\
612 D: aquí quedamos que el árbol, tanto se planta como crece ¿verdad? a ver escríbame aquí árbol. \# señala una zona en blanco del pizarrón debajo de la plabra «globo».\\
613 AH: árbol\\
614 D: esa es la cosa\\
615 AH: árbol, árbol \# Escitura: «ArPOL»\\
616 D: ¿qué está mal aquí?\\
617 AH: pe\\
618 D: esa la puede convertir ¿no? \# señala la letra P.\\
619 AH: ah con be \# Escritura: cambia la P por la B.\\
620 \# Material: lámina con dibujos que representan el proceso de ponerse un calcetín: 1) pie sin nada puesto, 2) pie con un calcetín a la mitad, 3) pie cubierto por el calcetín.\\
621 D: ¿aquí cuál será el verbo, qué vemos aquí?\\
622 AH: el pie\\
623 D: ¿y ahora qué tiene aquí el pie?\\
624 AH: la sal este, la, los calcetines\\
625 D: ¿y cuál es la diferencia de esta imagen a esta? \# señala primero la imagen donde el calcetín está a la mitad del pie y después señala la imagen en la que el calcetín está puesto en el pie completamente.\\
626 AH: los calcetines están bien feos \# se ríe.\\
627 D: sí puede que estén feos pero ¿cuál es la diferencia, dónde está aquí el calcetín y dónde esta aquí el calcetín? \# señala las imágenes en el mismo orden.\\
628 AH: en el ¿quién? en este, en el calcetín el calcetín, estaban bien feos \# señala la imagen con el calcetín completamente puesto en el pie.\\
629 D: ¿pero el calcetín dónde está todavía mal puesto?\\
630 AH: ah, ¿bien qué?\\
631 D: ¿dónde está mal puesto?\\
632 AH: mal puesto, aquí \# señala la imagen con el calcetin a medio poner.\\
633 D: ¿y dónde está bien puesto?\\
634 AH: bien feo bien benito bien bonito bien bonito \# señala la imagen con el calcetín completamente puesto.\\
635 D: muy bien ya está bien puesto aquí ¿verdad?\\
636 AH: sí\\
637 D: ¿y cómo se le llama a la acción meter el piesito en el calcetín, cómo le llamos a eso?\\
638 AH: los calcetines están bien bien\\
639 D: sí pero los calcetines los /p/\\
640 AH: ¿calcetines?\\
641 D: ¿los po?\\
642 AH: pone, me pone me pon\\
643 D: ajá ¿usted se qué?\\
644 AH: me /.poné el tel el caltes./ el calcetín\\
645 D: ¿pero alguien se los pone o usted se los pone?\\
646 AH: yo yo\\
647 D: ¿entonces, yo me?\\
648 AH: yo me\\
649 DT: a ver todo completo dígalo ¿qué hace usted con los calcetines?\\
650 AH: yo me yo me yo me voy\\
651 D: yo me ¿po?\\
652 AH: me pongo el calcetín\\
653 D: muy bien muy bien ¿ya vio?\\
654 \\
655 \# Segmento: reconocimiento de categorías de palabras: cosas y acciones.\\
656 \# Materiales: láminas que representan acciones (misma que en el segmento previo).\\
657 D: ¿aquí la cosa qué quedamos que era? \# señala la lámina con las imágenes del árbol. ¿el qué?\\
658 AH: el árbol\\
659 D: ¿aquí cuál es la cosa? \# señala la imagen con el globo siendo inflado.\\
660 AH: el globo el globo\\
661 D: y esas son las cosas no lo verbos ¿aquí cuál es la cosa? si aquí era el globo ¿aquí cuál es la cosa? \# le muestra la lámina con las imágenes del calcetín.\\
662 AH: la\\
663 D: la cosa\\
664 AH: acción\\
665 D: no no pero la cosa\\
666 AH: ...\\
667 D: ¿qué es esto?\\
668 AH: el calcetín\\
669 D: esa es la cosa ¿entonces la acción cuál es?\\
670 AH: calcetín\\
671 D: la acción\\
672 AH: acción acción acción\\
673 D: sí pero en estas imágenes ¿cuál es la acción? \# señala la lámina con el proceso de «ponerse un calcetín».\\
674 AH: acción acción\\
675 D: ¿pero cuál es?\\
676 AH: acción\\
677 D: a ver vea esta ¿y cuál es la acción?\\
678 AH: calcetín\\
679 D: no esa es la cosa ¿cuál quedamos que era la acción\\
680 AH: calcetín\\
681 D: a ver ¿qué dice aquí?\\
682 AH: ponerse\\
683 D: esa es la acción\\
684 AH: ponerse\\
685 D: a ver aquí escríbalo \# señala la zona del pizarrón de acciones. o poner, como usted quiera\\
686 \# Condición: escritura.\\
687 AH: poner poner /s/ se, ponerse \# Escritura: «PoNERce»\\
688 D: ¿y entonces qué nos ponemos? \# señala la zona del pizarrón de cosas.\\
689 AH: cosas\\
690 D: ¿y en este caso cuál era la cosa, el?\\
691 AH: no ya no, este acción\\
692 D: ajá pero aquí en las cosas ¿qué nos ponemos un qué?\\
693 AH: acción\\
694 D: era un ¿cal?\\
695 AH: calcetín, calcetín\\
696 D: sí esa era la cosa, a ver escríbalo aquí \# señala la zona del pizarrón de cosas.\\
697 AH: cal, cal ce, cal ce, ti, calcetines \# Escritura: «CaLcETiN»\\
698 D: así está bien\\
699 AH: ajá\\
700 \# Condición: conversación.\\
701 D: ¿le sigue costando mucho trabajo?\\
702 AH: no no\\
703 D: ¿crees que ya pude diferenciar las acciones de las cosas?\\
704 AH: sí hombre sí\\
705 D: ¿qué tienen las acciones que las cosas no?\\
706 AH: este cosas aquí \# señala la zonas del pizarrón donde se escribieron cosas en las actividades anteriores.\\
707 D: a ver dígame un ejemplo de una cosa, dígame algo que usted veo o se acuerde que sea una cosa\\
708 AH: aquí en la puerta ¿cómo? la puerta\\
709 D: la puerta es una cosa, muy bien\\
710 AH: sí la puerta\\
711 D: hora déme un ejemplo de una acción, una acción, una cosa no, acuérde que las acciones tienen cambios ¿verdad?\\
712 AH: la silla la silla\\
713 D: la silla no, es una cosa, estas están más difíciles ¿verdad? \# señala la zona del pizarrón donde se escribieron acciones.\\
714 AH: ah pues sí hombre\\
715 D: ¿pero qué podemos hacer con la silla? por ejemplo ¿cómo se le llama a pasar de esta posición a esta? \# se levanta de la silla y vuelve a sentarse.\\
716 DT: ¿qué hizo?\\
717 D: ¿yo me?\\
718 AH: ...\\
719 D: ¿cómo se le llama a pasar de esta posición a esta otra? \# se levanta y vuelve a sentarse en su asiento.\\
720 DT: ¿qué hizo?\\
721 D: ¿cómo se le llama a eso? yo me se\\
722 AH: yo me me ca me me ¿qué?\\
723 D: yo me se\\
724 AH: me senté me senté\\
725 D: muy bien esa sí es la acción porque vio que me moví y vio que hubo un cambio\\
726 AH: me senté me senté\\
727 D: pasé de estar parada a estar sentada ¿verdad?\\
728 AH: me senté me senté\\
729 \\
