\section{Sesión del 05 de septiembre del 2024}
\noindent
\textbf{ID}: AH-0509-2024\\
Participantes:\\
\textbf{AH}: Mujer, 65 años, informante.\\
\textbf{D}: Daniela Salinas, entrevistadora\\
\textbf{DT}: Dante Nava, entrevistador\\
\textbf{AC}: Mujer, familiar de AH.

\noindent
001 \# Segmento: orientación temporal.\\
002 \# Condición: conversación.\\
003 DT: muy bien ¿me puede decir la fecha de hoy?\\
004 AH: cinco, cinco\\
005 DT: ¿qué más?\\
006 AH:  de, cinco de jun, de jun, de cinco, cinco de septiembre, de septiembre\\
007 DT: sí ¿qué más?\\
008 AH: dos mil veinticuatro\\
009 DT: a ver todo completo otra vez\\
010 AH: cinco de agosto, de septiembre, dos mil veinticuatro\\
011 DT: ajá ¿y qué día de la semana es?\\
012 AH: jueves\\
013 DT: muy bien ¿y ayer qué día fue?\\
014 AH: mie, miércoles\\
015 DT: ¿qué número?\\
016 AH: cuatro\\
017 DT: sí muy bien ¿y mañana?\\
018 AH: seis, seis\\
019 DT: muy bien ¿qué día de la semana?\\
020 AH: este, viernes viernes\\
021 DT: muy bien\\
022 \\
023 \# Segmento: reporte de actividades.\\
024 \# Condición: conversación.\\
025 DT: a ver platíquenos qué hizo en su casa de tarea\\
026 AH: ah la tarea \# busca su libreta.\\
027 DT: a ver ¿pero qué le dejamos, se acuerda?\\
028 AH: este, palbras\\
029 DT: ¿qué palabras le dejamos?\\
030 \# Condición: lectura.\\
031 AH: yo trapié la cocina, la recámara, en la mañana estaba desayunando con, estaba /.deso deseyuando./ sopa y huevo, yo aquí /.ye ya./ aquí sopa y pechuga, atole y agua, yo comí tre, verdura, verdura, chi chile chile pasilla, y agua, atole con queso, yo cené atole y pan tostado y ce cero \# lee rápidamente varias hojas de su cuaderno.\\
032 \# Condición: conversación.\\
033 DT: órale\\
034 AH: ajá, a por el azúcar, por la azúcar, ayer estaba bien feo, este ¿cuánto? doscientos cincuenta, doscientos veinte\\
035 DT: ¿de azúcar?\\
036 AH: sí\\
037 DT: ¿de usted?\\
038 AH: sí, y ya al ratito ya\\
039 DT: ¿bajó?\\
040 AH: sí, este, ciento veinte\\
041 DT: hay que cuidarse eh\\
042 D: con más razón, las cocas prohibidas, ya no, ni de las chiquitas\\
043 \# Condición: lectura.\\
044 AH: estaba durmiendo, me /.vo./ me levanté a las /.senco seis sin tres sei./ seis treinta\\
045 DT: ¿de la mañana?\\
046 AH: con R\_\\
047 \# Escrito: 30 de agosto del 2024- mi diario- estaba dormida  y- me levanté a las 6:30- despues desayune con R.\\
048 \# Condición: conversación.\\
049 DT: ¿también se levanta temprano es señor R\_?\\
050 AH: Sí por los, televisión, por la tele\\
051 DT: ¿le televisión?\\
052 AH: no por las\\
053 DT: ¿por?\\
054 AH: por la tarea, por ay no hombre \# comienza a reírse.\\
055 DT: ¿se levanta a hacer tarea a las seis de la mañana?\\
056 AH: no, voy con R\_ con R\_\\
057 DT: ¿él se va a trabajar?\\
058 AH: sí, ajá\\
059 DT: ah ya es un cuaderno nuevo\\
060 AH: sí hombre\\
061 DT: ¿ya se acabó el otro?\\
062 AH: no ya, este R\_ me lo compró\\
063 DT: ¿se lo compró?\\
064 AH: me lo compró\\
065 \\
066 \# Segmento: conversación libre.\\
067 \# Condición: conversación.\\
068 AH: ¿enfermo?\\
069 DT: sí me enfermé\\
070 AH: bien chismosa yo también, bien chismosa\\
071 D: observadora\\
072 AH: sí\\
073 DT: observadora, también los psicólogos tenemos que andar de chismosos\\
074 AH: pues sí, sí cierto\\
075 DT: tenemos que andar preguntando de todo\\
076 AH: pues sí cierto\\
077 DT: ¿y usted no se ha enfermado?\\
078 AH: no, gracias a dios no\\
079 DT: ¿no?\\
080 AH: gracias a dios\\
081 D: tiene mejores defenzas que nosotros \# se ríe.\\
082 AH: ah pues quien sabe\\
083 DT: y come muy bien por lo que \# señala el cuaderno de AH.\\
084 AH: ay sí\\
085 DT: ¿qué le gusta comer de todo esto?\\
086 AH: este, sopa\\
087 DT: ¿qué es lo que más le gusta?\\
088 AH: este, chilaquiles\\
089 DT: \# ve a AH con cara de sorpresa.\\
090 AH: uy sí bien sabrosos\\
091 DT: ¿de cuáles, rojos o verdes?\\
092 AH: verdes\\
093 DT: ¿verdes le gustan más?\\
094 AH: verdes verdes ¿y tú? \# se ríe.\\
095 D: a mi me gustan más los rojos\\
096 AH: ah pus (pues) sí\\
097 DT: ¿a ti? \# se dirige a D.\\
098 D: me gustaban más los verdes pero ahora más los rojos\\
099 AH: ah ¿tú crees? \# comienza a reírse y señala alternadamente a AT y D.\\
100 DT: ya la convertí \# todos se ríen.\\
101 AH: con tu novio \# señala a D.\\
102 DT: pues ya era hora de que se pasara a los rojos ¿no?\\
103 AH: ah pus (pues) sí \# se ríe.\\
104 DT: ¿y cómo se prepara sus chilaquiles?\\
105 AH: este con /.quesa./ con queso, y chilaquiles\\
106 DT: ¿usted los hace?\\
107 AH: sí hombre\\
108 DT: ¿a ver cómo los hace?\\
109 AH: este\\
110 DT: paso por paso\\
111 AH: chilaquiles, este, verdes, chiles verdes, tomate /.jeto./ jitomate, ah no tomate no, tomate\\
112 DT: porque no le gustan los rojos\\
113 AH: sí, tomate y mucho chile\\
114 DT: ¿qué chile es es?\\
115 AH: chile verde\\
116 DT: ¿no es serrano?\\
117 AH: ah no, chile verde\\
118 DT: así ¿chile verde?\\
119 AH: sí sí, chile verde\\
120 DT: ah porque yo lo hago con serrano, y a veces con, se le hecha jalapeño a veces\\
121 AH: ay ¿a poco?\\
122 DT: ¿usted no le hecha jalapeño?\\
123 AH: no no\\
124 DT: ¿y no le pone pollo?\\
125 AH: sí, sí pollo\\
126 DT: ¿y usted también lo hace, el pollo?\\
127 AH: sí\\
128 DT: ¿cómo lo hace?\\
129 AH: este, con agua, con agua y agua, y este, po con pollo\\
130 DT: ¿pero qué hace con el agua y con el pollo?\\
131 AH: este voy a voy, este, lavando, lavando lavando, el pollo\\
132 DT: ¿lava el pollo?\\
133 AH: sí sí\\
134 DT: ¿y después de lavarlo?\\
135 AH: este, espérame, este, el agua y este, en la estufa, en la estufa\\
136 DT: ¿qué hace en la estufa?\\
137 AH: este, no me acuerdo\\
138 DT: muy bien\\
139 \\
140 Segmento: \\
141 \\
142 \# Segmento: inducción de verbos mediante oraciones simples incompletas escritas.\\
143 \# Material: pizarrón blanco, marcadores azul y rojo.\\
144 \# Condición: conversación.\\
145 DT: ¿qué dice aquí? \# le muestra el pizarrón blanco,\\
146 \# Escrito: (en el pizarrón) se muestra una línea horizontal seguida de la palabra «sopa».\\
147 AH: sopa\\
148 DT: sí muy bien, aquí falta algo \# señala la línea horizontal--- aquí arriba va algo que falta\\
149 AH: en la estufa \# se ríe.\\
150 DT: no\\
151 AH: ¿no?\\
152 DT: aquí falta un verbo ¿qué hacemos con la sopa?\\
153 AH: sopa ...\\
154 DT: ¿la lavamos, se lava la sopa?\\
155 AH: sí, no no no\\
156 DT: ¿qué se hace con la sopa?\\
157 AH: este, en la estufa, en la estufa\\
158 DT: ¿pero qué se hace? el verbo queremos\\
159 AH: ...\\
160 DT: a ver hágale así \# con los dedos juntos de la mano derecha, los aproxima a la boca.\\
161 AH: \# hace el mismo movimiento con la mano izquierda.\\
162 DT: ¿qué es esto? \# repite el mismo movimiento.\\
163 AH: este, /.camiar./ comiendo comiendo\\
164 DT: a ver escríba aquí \# le da el marcador azul y señala la línea horizontal en el pizarrón.\\
165 \# Condición: escitura.\\
166 AH: comiendo, co, mi, comien \# escritura: «comiE», observa a DT con duda.\\
167 DT: va bien va bien\\
168 AH: comien, do, comiendo, do \# escitura final: «comiEbo».\\
169 DT: bueno aquí tenemos un error \# borra «bo».\\
170 AH: ajá\\
171 DT: \# le acerca el pizarrón.\\
172 AH: ¿e?\\
173 DT: ya está la /e/, comien, /n/ \# alarga un poco el sonido de /n/.\\
174 AH: ¿ene? ene\\
175 DT: sí\\
176 AH: comien, do \# escritura: agrega «Nbo» resultando en «comiENbo».\\
177 DT: ¿esta cómo suena? \# señala la letra b.\\
178 AH: be\\
179 DT: a ver ¿entonces qué dice aquí?\\
180 AH: do do\\
181 DT: a ver, esa estaba mal \# borra «bo»--- es la /d/\\
182 AH: ¿de?\\
183 DT: sí\\
184 AH: ¿de?\\
185 DT: ¿cómo es?\\
186 AH: ¿de? ah, comien, do \# escritura: agrega «Po» resultando en «comiENPo».\\
187 \# Condición: conversación.\\
188 DT: ¿esta cómo suena? \# señala la letra P.\\
189 AH: ¿de?\\
190 DT: a ver véala bien\\
191 AH: de\\
192 DT: esta que usted escribió es esta \# escribe la letra P en el pizarrón.\\
193 AH: ah sí cierto\\
194 DT: ¿cómo suena?\\
195 AH: de\\
196 DT: y entonces ¿esta cómo suena? \# escribe la letra d al lado de la P.\\
197 AH: ah pos (pues) sí mira\\
198 DT: ¿cómo suena esta? \# señala la letra d.\\
199 AH: de\\
200 DT: ¿y esta? \# señala la letra P.\\
201 AH: pe \# sonríe.\\
202 DT: ajá muy bien ¿ya vio qué pasó ahí?\\
203 AH: ajá\\
204 \# Condición: escritura.\\
205 DT: mire \# borra la letra P en «comiENPo».\\
206 AH: \# escritura: agrega la letra b resultando en «comiENbo».\\
207 DT: a pero mire \# señala de forma alternada la letra b que Ah recién escribió y la letra d usada de modelo.\\
208 AH: ah sí ya\\
209 DT: está al revés\\
210 AH: sí cierto \# comienza a reírse.\\
211 DT: sí\\
212 AH: ah cierto \# escritura: cambia la b por la d resultando en «comiENdo».\\
213 DT: muy bien, entonces mire, esto es el verbo \# con el marcador azul encierra en un óvalo la palabra «comiendo» que AH escribió--- es lo que hacemos, ¿ya vio? lo que podemos hacer\\
214 \# Condición: conversación.\\
215 AH: ah sí\\
216 DT: ¿a ver qué otra cosa le gusta comer?\\
217 AH: café con leche, no pues ya mucho café no ya no\\
218 D: ¿le hace daño?\\
219 AH: ¿mande? uy sí\\
220 DT: por la presión ¿verdad?\\
221 AH: ay sí, ciento diez ciento veinte, ciento diez\\
222 D: ¿y el tesito no le gustará mejor? en lugar del café\\
223 AH: sí este, atole nada más\\
224 D: pero sin mucha azúcar\\
225 AH: no ya no nada, nada de azúcar\\
226 D: muy bien muy bien\\
227 AH: nada de azúcar, ay sí ya está bien feo\\
228 D: uno se acostumbra\\
229 AH: un bolillo hombre, también\\
230 D: pero uno se acostumbra\\
231 AH: ¿sí?\\
232 DT: ¿no le gusta las verduras?\\
233 AH: sí sí\\
234 DT: como la sopa, sopa de verduras, eso es muy bueno que coma\\
235 AH: sí\\
236 DT: y carne\\
237 AH: ajá cierto\\
238 DT: pollo y pescado principalmente\\
239 AH: ¿a poco?\\
240 DT: es lo más saludable\\
241 AH: ¿a poco?\\
242 DT: como atún\\
243 AH: ¿a poco, atún? atún, ahí esta bien\\
244 DT: ¿no le gusta?\\
245 AH: no\\
246 DT: ¿el atún no le gusta?\\
247 AH: sí sí\\
248 DT: a ver, vamos a hacer otro ejercicio ¿aquí qué dice? \# le muestra el pizarrón blanco con la palabra «atole» escrita al lado de una línea horizontal.\\
249 AH: atole\\
250 DT: ajá, aquí nos falta el verbo \# señala la línea horizontal a la izquierda de la palabra «atole».\\
251 AH: ajá\\
252 DT: ¿qué se hace con el atole?\\
253 AH: ...\\
254 DT: a ver el atole nos lo \# se lleva los dedos de la mano izquierda lentamente hacia la boca--- a ver usted hágale así\\
255 AH: \# imita el gesto de DT.\\
256 DT: ¿se come?\\
257 AH: no\\
258 DT: ¿qué se hace?\\
259 AH: bebí, bebí, bebiendo\\
260 DT: eso \# le da el marcador azul.\\
261 \# Condición: escritura.\\
262 AH: bebiendo, be, be, bi, bie, bie, bien, do, bebiendo \# escritura: «BeBiNto»\\
263 \# Condición: conversación.\\
264 DT: bueno ¿ya vio que acá siempre nos falta el verbo? \# señala la palabra que AH acaba de escribir.\\
265 AH: ajá\\
266 DT: siempre el verbo va a acompañar a la cosa\\
267 AH: ajá \# observa la palabra que acaba de escibir.\\
268 DT: el verbo es lo que hacemos nosotros\\
269 AH: \# no presta atención y contiúa viendo su escritura.\\
270 DT: aquí ya se dio cuenta de algo ¿verdad?\\
271 AH: sí sí\\
272 DT: a ver usted borre \# le da el borrador.\\
273 \# Condición: escritura.\\
274 AH: \# escritura: borra la letra t cambiándola por d, resultando en «BeBiNdo».\\
275 DT: aquí nos falta otra cosa \# señala «iN» en la palabra escrita.\\
276 AH: be ¿i?\\
277 DT: aquí está la i, falta algo aquí \# vuelve a señalar la misma parte de la palabra.\\
278 AH: bebien\\
279 DT: bebiendo\\
280 AH: bebien\\
281 DT: aquí en medio falta algo\\
282 AH: bebien, do, bebien, do \# escritura: hace unn trazo irreconocible en la zona indicada de la palabra.\\
283 DT: okey, a ver vamos a repetir esta parte de aquí \# borra a partir de la letra N resultando en «BeBi».\\
284 AH: ah sí\\
285 DT: no pasa nada, lo que nos interesa es que vea que va el verbo y luego la cosa, a ver, bebi ¿qué sigue?\\
286 AH: bebien, do\\
287 DT: en\\
288 AH: en\\
289 DT: en \# alarga el sonido de la /e/.\\
290 AH: bebiendo \# escritura: «BeBiNdo».\\
291 D: a ver léalo\\
292 AH: bebiendo\\
293 DT: vamos poco a poquito\\
294 D: léalo, acuérdese, lentito\\
295 AH: bebien, ah /e/\\
296 D: ajá muy bien, siempre cuando lo lee se da cuenta de qué errores tiene ¿verdad?\\
297 AH: ah cierto\\
298 D: así que cuendo tenga duda, lea lo que escibrió, porque a lo mejor se da cuenta de que le falta algo, o que puso algo que no\\
299 AH: ah, bebien \# escritura final: «BEBiENdo».\\
300 \\
301 \# Segmento: continuación de inducción de verbos mediante oraciones simples incompletas escritas.\\
302 \# Materiales: hoja blanca, plumas de colores negro y azul.\\
303 \# Condición: conversación.\\
304 DT: a ver ¿qué hace usted en el parque?\\
305 AH: este, ¿en el parque? este, futbol futbol\\
306 DT: ¿usted juega?\\
307 AH: sí\\
308 DT: ¿a poco sí?\\
309 AH: sí\\
310 DT: órale\\
311 D: ¿en serio?\\
312 AH: sí sí\\
313 D: ¿con quién juega?\\
314 AH: con R\_\\
315 D: ah qué bonito, qué padre\\
316 AH: sí sí\\
317 D: ¿desde cuándo?\\
318 AH: este, dos años\\
319 DT: muy bien\\
320 AH: dos años\\
321 D: ¿le gusta jugar?\\
322 AH: sí\\
323 DT: a ver vamos a intentar con estos\\
324 \# Escrito: \_\_\_\_\_futbol- \_\_\_\_\_pollo- \_\_\_\_\_televisión.\\
325 DT: vamos a hacer lo mismo que estábamos haciendo aquí \# señala el pizarrón.\\
326 AH: ajá\\
327 DT: ¿trae sus lentes?\\
328 AH: sí \# busca en su bolsa y saca unos lentes.\\
329 DT: muy bien, ¿son nuevos?\\
330 AH: sí, en el tianguis, cuatro cu, catorce quince pesos\\
331 DT: ¿y sí ve bien?\\
332 AH: sí\\
333 DT: ah bueno, a ver\\
334 AH: veinte pesos\\
335 DT: ¿veinte pesos?\\
336 AH: ajá \# comienza a reírse.\\
337 DT: órale, bien baratos\\
338 AH: sí\\
339 DT: muy bien \# le da la pluma azul--- a ver vamos con el primero ¿qué dice aquí? \# señala la primera línea de la hoja.\\
340 AH: futbol\\
341 DT: ¿qué hace?\\
342 AH: este, estaba en el futbol\\
343 DT: sí ¿pero cuál es el verbo?\\
344 D: la acción\\
345 AH: este, ay no me acuerdo\\
346 DT: ¿qué hace? a ver, acuérdese cuando esta ahí en el parque ¿qué hace usted?\\
347 AH: aquí está \# mueve ambos brazos de forma alternada hacia adelante y hacia atrás a la altura del rostro.\\
348 DT: ajá ¿qué eso, cómo se llama eso?\\
349 AH: este, en el futbol, en la/.mar el perca merca./ no el mercado no \# se ríe--- en la, en el\\
350 DT: a ver ¿comemos futbol?\\
351 AH: no hombre no, este, en el, el /.mer percado merc./ no pérame (espérame)\\
352 DT: sí no se preocupe\\
353 AH: este en el parque, en el parque\\
354 DT: muy bien, pero queremos la acción\\
355 AH: el parque, en el parque, estaba el futbol\\
356 DT: sí pero falta ahí algo\\
357 AH: ¿qué?\\
358 D: ¿estaba qué?\\
359 AH: sí estaba, estaba\\
360 DT: estaba\\
361 AH: estaba \# aproxima la mano con la pluma a la hoja.\\
362 DT: a ver no, falta aquí algo, estaba \# señala la línea horizontal en «\_\_\_\_\_futbol».\\
363 AH: es\\
364 DT: estaba\\
365 AH: estaba\\
366 DT: falta algo, futbol\\
367 AH: estaba \# aproxima la mano con la pluma a la hoja. ¿e?\\
368 DT: ¿qué estaba haciendo con el futbol?\\
369 AH: estaba en el parque\\
370 DT: ¿estaba comiendo futbol?\\
371 AH: no no no\\
372 DT: ¿estaba limpiando?\\
373 AH: no no no\\
374 DT: ¿entonces qué estaba haciendo en el parque?\\
375 AH: ...\\
376 DT: es algo que hacen los niños también\\
377 AH: ah sí\\
378 DT: con sus juguetes\\
379 AH: en el parque\\
380 DT: ajá ¿qué estaba haciendo?\\
381 AH: eh, ah \# mueve ambos brazos de forma alternada hacia adelante y hacia atrás--- estaba \# repite el movimiento con los brazos.\\
382 DT: sí es la acción, el lo que hacemos\\
383 AH: estaba, en el futbol estaba, estaba, no no \# niega con la cabeza.\\
384 DT: ¿hace algo con las piernas?\\
385 AH: no no no \# niega con la cabeza.\\
386 DT: ¿con los brazos?\\
387 AH: no no, así con este \# cerrando los puños de ambas manos, levanta los brazos a la altura del rostro y los alterna hacia adelante y hacia atrás en repetidas ocasiones.\\
388 DT: ¿con lo brazos?\\
389 AH: sí \# afirma con la cabeza rápidamente.\\
390 DT: ¿qué hace con lo brazos?\\
391 AH: ay pérame (espérame)\\
392 DT: sí sí no se preocupe\\
393 AH: estaba, estaba, en el parque estaba, est, en el parque estaba, estaba, en el parque estaba, en el parque estaba, ay dios bendito \# niega con la cabeza.\\
394 DT: está bien, no se preocupe\\
395 AH: estaba en el parque estaba en el futbol\\
396 DT: mire, le vamos a ayudar \# escribe en el pizarrón para regular a AH.\\
397 \# Escrito: en el parque- estaba \_\_\_\_\_- futbol.\\
398 \# Condición: lectura.\\
399 DT: a ver vamos a leer esto\\
400 AH: ah, en el paque estaba, ah \# muestra sorpresa.\\
401 DT: aquí ya vio que nos falta\\
402 AH: sí\\
403 DT: a ver vamos a leerlo todo completo\\
404 AH: en el parque estaba \# lee hasta «estaba» ignorando «futbol»--- en el parque\\
405 DT: en el parque estaba, aquí nos falta algo\\
406 AH: estaba\\
407 DT: y luego futbol\\
408 AH: ay hombre\\
409 DT: aquí nos falta algo \# señala la línea en blanco antes de futbol--- que es el verbo\\
410 AH: ah, en el parque estaba\\
411 DT: a ver\\
412 AH: en el parque es\\
413 DT: voy a poner un verbo que se me ocurre a mi\\
414 AH: sí\\
415 DT: \# escribe «limpiando» sobre la línea.\\
416 AH: ah limpiar\\
417 DT: a ver léalo\\
418 AH: limpiando, limpiando\\
419 DT: ¿y está bien?\\
420 AH: no\\
421 DT: ¿por qué no?\\
422 AH: no\\
423 DT: a ver vamos a le\\
424 AH: estaba ((árbol)) limpiando \# interrumpe a DT--- no\\
425 DT: a ver vamos a leerlo todo poco a poquito\\
426 AH: en el parque estaba limpiando el fultbol, no \# niega con la cabeza.\\
427 DT: no ¿verdad?\\
428 AH: no no\\
429 DT: ¿entonces qué va a aquí? \# borra «limpiando»--- no es limpiando, no es comer ¿qué será?\\
430 AH: ta (está) bien difícil\\
431 DT: ¿difícil?\\
432 D: ¿si le pones una imagen? \# dirigiéndose a DT.\\
433 DT: a ver vamos a probar con una imagen.\\
434 \\
435 \# Segmento: inducción del verbo jugar usando un vídeo corto.\\
436 \# Material: teléfono celular donde se muestra un vídeo donde se observa a un jugador de futbol pateando un balón y un portero saltando a atraparlo.\\
437 \# Condición: conversación.\\
438 DT: a ver ¿qué están haciendo aquí? \# le acerca el teléfono para que vea el vídeo.\\
439 AH: las, las canchas, las canchas\\
440 DT: sí ¿pero qué están haciendo?\\
441 AH: el futbol, con el futbol\\
442 DT: ajá pero ¿qué están haciendo?\\
443 AH: estaba, no no \# niega con la cabeza.\\
444 D: estaban /j/\\
445 AH: jugando jugando\\
446 DT: ahí está ¿ya vio que sí puede?\\
447 AH: jugando jugando\\
448 DT: a ver \# le da el marcador azul.\\
449 \# Condición: escritura.\\
450 AH: ju, ju, ju, juga, jugando, jugando \# escritura: «gugando»\\
451 DT: sí, nada más esta letra es con la que hemos tenido problemas \# señala la primera g.\\
452 AH: ah sí\\
453 DT: se acuerda que también tenemos esta letra \# escribe la letra j.\\
454 AH: ah sí cierto\\
455 DT: ¿esta cómo suena?\\
456 AH: ja je\\
457 DT: no \# se corrige ya que en ocasiones AH confunde el sonido de la letra /j/ y el nombre de la letra «g».\\
458 AH: ja ga, ja ja\\
459 DT: así suena \# borra la j y escribe la g en su lugar.\\
460 \# Condición: conversación.\\
461 AH: ah sí\\
462 DT: ¿ya vio?\\
463 AH: sí\\
464 DT: jugando en el verbo\\
465 AH: ah\\
466 DT: es lo que nos faltaba\\
467 AH: ah mira\\
468 DT: mire lo voy a quitar \# borra «jugando».\\
469 AH: si\\
470 DT: si lo quito no tiene sentido, vamos a leerlo todo\\
471 \# Condición: lectura.\\
472 AH: en el parque estaba jugando \# agrega «jugando» a pesar de que ya no se encuentra escrita la palabra.\\
473 DT: exacto, ahí está, a ver otra vez vamos a escribirlo\\
474 \# Condición: esc\\
475 AH: ju, ga, juga, ndo, jugando \# escritura: «jugaNdo»--- jugando\\
476 DT: muy bien, esto es el verbo \# señala lo que AH acaba de escribir.\\
477 \# Condición: conversación.\\
478 AH: ah mira\\
479 DT: es lo que nos falta ¿ya vio?\\
480 AH: sí\\
481 DT: vamos a hacer otro intento ¿se le hizo difícil?\\
482 AH: sí\\
483 \\
484 \# Segmento: inducción de verbos hablados.\\
485 \# Material: pizarrón blanco, marcador azul y rojo.\\
486 \# Condición: conversación.\\
487 DT: a ver, díganos qué hace usted\\
488 AH: trapiando (trapeando) la sala\\
489 DT: vamos a anotar ese \# escribe en el pizarrón «trapeando»--- trapeando, muy bien ya nos dijo uno\\
490 AH: sí, ah\\
491 DT: ahí va uno \# pone una paloma (marcador de verificación) al lado--- ¿qué más cosas hace?\\
492 AH: este, los trastes\\
493 DT: ajá\\
494 AH: los trastes\\
495 DT: ¿qué hace con los trastes?\\
496 AH: este, lavando los trastes\\
497 DT: muy bien\\
498 D: muy bien\\
499 AH: lavando los trastes\\
500 DT: muy bien vamos a anotarlo aquí \# escribe «lavando»--- lavando ¿qué más hace?\\
501 AH: lavando los trastes, en la est, en la estufa\\
502 DT: ¿qué hace en la estufa?\\
503 AH: en la estufa estaba bien /.cochambras./ cochambrosa\\
504 DT: ¿cochambrosa?\\
505 AH: sí cochambroso cochambrosa\\
506 DT: ¿y qué hace en la estufa? a ver le voy a dar una pista, co\\
507 AH: ...\\
508 DT: en la estufa co\\
509 AH: co\\
510 DT: ¿coci?\\
511 AH: cocinando, cocinando\\
512 DT: muy bien\\
513 D: ajá muy bien\\
514 AH: cocinando\\
515 D: y también respecto a lo de que estaba cochambroso se puede sacar otro ¿no?\\
516 AH: pues sí\\
517 D: estaba ¿qué haciendo con la estufa, cuando estaba sucia?\\
518 AH: bien /.cha./ bien sucio bien sucia\\
519 D: ajá pero ¿usted qué estaba haciendo?\\
520 AH: cochambrosa, cochambrosa\\
521 DT: a ver si algo está sucio\\
522 AH: cochambrosa\\
523 DT: ajá ¿qué hacemos para que ya no esté así?\\
524 AH: este, lavándola lavándola\\
525 DT: ahí está\\
526 D: muy bien\\
527 AH: lavándola\\
528 DT: muy bien, otro verbo ¿qué otro se le ocurre? va muy bien\\
529 AH: ¿sí?\\
530 DT: sí ya lleva cuatro\\
531 AH: ah mira\\
532 DT: ¿qué hace en la sala?\\
533 AH: en la sala esta, en la sala trapeando\\
534 DT: eso ¿qué más?\\
535 AH: trapiando la ma, la el baño\\
536 DT: ajá ¿en el baño?\\
537 AH: el baño estaba sucio\\
538 DT: ¿y qué hacemos cuando algo está sucio?\\
539 AH: durm, este, soña \# comienza a reír.\\
540 DT: está sucio y le da sueño\\
541 AH: no, estaba trapiando, estaba trapiando\\
542 DT: ajá muy bien\\
543 AH: estaba trapiando, trapiando la /s/ la\\
544 DT: bueno muy bien, ese ya lo había dicho, a ver vamos a ver, en el comedor ¿qué se hace en el comedor?\\
545 AH: este, comiendo\\
546 DT: ah mire, muy bien\\
547 AH: comiendo\\
548 DT: bueno está más fácil ¿verdad?\\
549 AH: sí cierto\\
550 DT: ¿y en la cocina?\\
551 AH: estaba, este, en ...\\
552 D: ¿generalmente qué se hace en la cocina?\\
553 AH: la cocina, este, la cocina ...\\
554 DT: de hecho ahí en el nombre ya lo dice, cocina\\
555 AH: cocina\\
556 DT: ¿qué hacemos en la cocina?\\
557 AH: en la cocina, esta\\
558 DT: ¿coci?\\
559 AH: en la coina estaba, estaba\\
560 DT: de hecho ya me lo dijo, a ver ¿qué hace en la cocina de todo esto? \# le muestra el pizarrón.\\
561 AH: cocinando\\
562 DT: ajá muy bien\\
563 AH: cocinando\\
564 DT: todos estos verbos nos los dijo ahorita\\
565 AH: cocinando\\
566 DT: a ver, uno más y ya ¿qué hace en su cuaderno?\\
567 AH: la tarea\\
568 DT: sí sí hace la tarea pero ¿cuál es la acción?\\
569 AH: ...\\
570 DT: sí es la tarea, muy bien pero ¿qué hacemos? a ver\\
571 AH: este\\
572 DT: ¿usted también le hace así no? \# hace un garabato sobre el pizarrón.\\
573 AH: sí\\
574 DT: ¿qué es esto?\\
575 \\
576 AH: espérame\\
577 DT: sí\\
578 AH: la tarea\\
579 DT: ¿qué hacemos?\\
580 AH: la tarea estaba \# con su dedo índice derecho traza sobre su palma derecha lo que parece ser un par de letras--- bien difícil \# se ríe.\\
581 DT: a ver para ayudarle vamos a hacer toda la oración desde cero\\
582 AH: sí sí\\
583 DT: a ver, dígame yo estaba\\
584 AH: yo estaba, yo estaba\\
585 DT: ¿qué es esto? \# hace trazos en el aire con el marcador rojo.\\
586 AH:yo, yo \# hace trazos con su dedo índice derecho sobre su palma izquierda.\\
587 DT: yo estaba, a ver dígalo\\
588 AH: yo estaba\\
589 DT: ¿qué estaba haciendo? \# hace trazos en el aire con el marcador.\\
590 AH: estaba\\
591 DT: es \# alarga el sonido de la /s/.\\
592 AH: estaba ...\\
593 DT: a ver, vamos a escribirlo acá, es más usted me va a ayudar a escribirlo \# pone el pizarrón al alcance de AH y le da el marcador azul--- a ver, yo estaba\\
594 AH: ¿yo?\\
595 DT: ¿se acuerda cómo es?\\
596 \# Condición: escritura.\\
597 AH: yo \# escritura: «yo».\\
598 DT: eso, estaba\\
599 AH: es, es, estaba, estaba \# escritura: «EStABa»--- yo estaba\\
600 DT: muy bien, a ver ponga una línea, ahí nos falta algo \# señala al lado de la palabra que AH acaba de escribir.\\
601 AH: \# traza una línea horizontal en el lugar indicado.\\
602 DT: aquí nos va a faltar algo \# señala la línea que AH acaba de trazar.\\
603 AH: yo estaba\\
604 DT: y vamos a poner acá \# señala debajo de las palabras previas--- en el cuaderno\\
605 AH: en el, en el \# escritura: «EN Le».\\
606 DT: cuaderno\\
607 AH: en el cua, cua, cua, der, cua, der, cuaderno, cuaderno \# escritura: «CUAPor»\\
608 DT: a ver, vamos a ir leyendo poco a poquito, yo le voy a ir marcando con el azul, a ver ¿aquí qué dice? \# subraya «EN».\\
609 \# Condición: lectura.\\
610 AH: en\\
611 DT: ¿aquí qué dice? \# subraya «Le».\\
612 AH: le, en el\\
613 DT: aquí pasó algo ¿ya vio?\\
614 AH: ha, cua\\
615 DT: está al revés \# señala «Le».\\
616 AH: ah \# se ríe.\\
617 DT: ¿qué dice?\\
618 AH: u\\
619 DT: ¿aquí qué dice? en esta partecita \# señala con el marcador «Le».\\
620 AH: en el, en el\\
621 DT: aquí solito \# con el borrador y la mano cubre el resto de elementos para que sólo sea visible «Le».\\
622 AH: el, el\\
623 DT: a ver fíjese\\
624 AH: ah, el\\
625 DT: a ver vamos otra vez \# borra «Le»--- el\\
626 \# Condición: escritura.\\
627 AH: ¿el? e, el \# escritura: A--- e\\
628 DT: no, a ver, acá abajito \# señala la parte inferior del pizarrón--- el\\
629 AH: en el \# escritura: «EN Le».\\
630 DT: a ver, otra vez pasó algo, este este, aquí nos pasó algo \# encierra con el marcador azul «Le».\\
631 AH: a ver, en el, ele\\
632 DT: a ver ¿cuál es el primer sonido?\\
633 AH: ele\\
634 DT: /e/\\
635 AH: ele\\
636 DT: /e/, /l/ ¿cuál es el primer sonido?\\
637 AH: /e/ \# repite el sonido de la /e/ varias veces.\\
638 DT: /e/ ajá\\
639 AH: ah \# escritura: cambia la L por E resultando en «Ee».\\
640 DT: exacto ¿ya vio? entonces esta no va acá \# borra la e--- ¿cuál va entonces ahí?\\
641 AH: en el\\
642 DT: el\\
643 AH: ah \# escritura final: «EL».\\
644 DT: ahí está, eso es lo que nos faltaba acá, vamos a borrar esto \# borra «EN Le CUAPor»--- muy bien entonces llevamos en el\\
645 AH: \# escritura: «EL».\\
646 DT: entonce ya podemos borrar esto \# borra «EN EL» que se encontraba en la parte inferior--- cuaderno\\
647 AH: cua, cua, cua, de, ¿de?\\
648 DT: de\\
649 AH: cuade, cuader, cuader \# escritura: «CUAPEN»--- no no \# niega con al cabeza.\\
650 DT: a ver \# le acerca el borrador--- ¿le ayudo o usted puede?\\
651 AH: a ver\\
652 DT: vamos a leer lento, porque eso nos ayuda a darnos cuenta\\
653 AH: ah sí\\
654 DT: vamos a ir poco a poquito, a ver ¿aquí qué dice?\\
655 \# Condición: lectura.\\
656 AH: erre erre \# interrumpe a DT.\\
657 DT: a ver ¿aquí qué dice? \# subraya «cua».\\
658 AH: en el cua\\
659 DT: \# señala la letra P.\\
660 AH: de ¿de?\\
661 DT: mire \# señala la letra P--- esta ¿cómo suena?\\
662 AH: de\\
663 DT: nos pasó hace rato igual, esta es la que usted escibió \# escribe la letra P en la parte inferior del pizarrón.\\
664 AH: ¿R\_? \# menciona el nombre de un familiar que usa como apoyo--- con R\_ \# se ríe.\\
665 DT: \# escribe al lado P la letra d--- ¿cuál suena cuaderno?\\
666 AH: este \# señala la letra d.\\
667 DT: ¿esta o esta? \# señala de forma alternada la letra P y luego la d.\\
668 AH: esta \# señala la letra d.\\
669 DT: ¿sí?\\
670 AH: esta \# señala la letra d.\\
671 DT: a ver, vamos a escribirlo otra vez \# borra las últimas letras para mantener «CUA»--- esto está bien, cua\\
672 AH: cua\\
673 DT: ahí está, falta derno\\
674 AH: es este \# señala la letra d.\\
675 DT: ajá\\
676 AH: \# intenta tomar el marcador de DT.\\
677 DT: a ver con su plumón.\\
678 \# Condición: escritura.\\
679 AH: ajá \# escritura: D.\\
680 DT: a ver\\
681 AH: cua, der, cuader \# escritura: «DEN».\\
682 DT: a ver, aquí nos faltó algo \# borra la letra N--- cuader\\
683 AH: ¿erre?\\
684 DT: erre\\
685 AH: ah sí, cuader, no, ¿no? \# escritura: «rl» hace un trazo recto que aquí representamos como la letra l.\\
686 DT: no\\
687 AH: no, ¿uno? \# rapidamente traza en el aire sobre su palma izquierda una línea recta.\\
688 DT: a ver \# borra la letra l.\\
689 AH: cuader\\
690 DT: no\\
691 AH: no \# escritura: «A» en la parte inferior del pizarrón--- ¿no?\\
692 DT: no, esa es la /a/\\
693 AH: cuaderno \# traza lo que parece ser la letra A sin le línea horizontal.\\
694 DT: así como si estuviera diciendo no\\
695 AH: ah, cuaderno \# escritura: «N» completa el trazo anterior para formar la letra.\\
696 DT: exacto muy bien\\
697 AH: cuader, no \# escritura final: «CUADErNO»\\
698 DT: bueno, estuvo un poco difícil\\
699 AH: sí ((verdad)) ahora sí\\
700 DT: ahora vamos a leerlo todo completo, a ver ¿puede leerlo todo completo?\\
701 AH: yo estaba en el cuaderno\\
702 \# Condición: conversación.\\
703 DT: okey, aquí nos falta algo como en las actividades de hace rato \# señala la línea horizontal.\\
704 AH: yo estaba, yo estaba ¿en el parque? \# se ríe.\\
705 DT: aquí falta qué estaba haciendo en el cuaderno\\
706 AH: ah, en el cuaderno\\
707 DT: ¿qué hace?\\
708 AH: estaba, estaba desayunando, desayunando \# se ríe.\\
709 DT: desayunando en el cuaderno\\
710 AH: \# continua riendo.\\
711 DT: pero sí es un verbo\\
712 AH: ¿qué es hijo?\\
713 DT: a ver ¿qué estaba haciendo en el cuaderno?\\
714 AH: este la tarea, la tare\\
715 DT: sí pero cómo se le dice a, yo agarrao el plumón y me pongo ¿a? \# finge escribir en el aire sobre el pizarrón.\\
716 AH: ¿a?\\
717 DT: ¿qué estoy haciendo cuando me pongo a?\\
718 AH: tarea\\
719 DT: pero ese no es el verbo\\
720 AH: ajá\\
721 DT: ¿estaba comiendo en el cuaderno?\\
722 AH: no no hombre\\
723 DT: ¿limpiando en el cuaderno?\\
724 AH: no no no\\
725 DT: ¿viajando?\\
726 AH: no hombre\\
727 DT: ¿entonces qué?\\
728 AH: estaba, el cuaderno\\
729 DT: a ver, yo voy a hacer la acción \# toma la libreta de AH y comienza a escribir con la pluma negra--- a ver ¿qué estoy haciendo, cómo se llama esto que estoy haciendo?\\
730 AH: ¿los verbos?\\
731 DT: no, estoy \# mueve la pluma en el aire.\\
732 AH: ...\\
733 DT: usted también lo hace, cuando agarra aquí su lápiz y se pone ¿a?\\
734 AH: ajá, ay no me acuerdo\\
735 DT: empieza con es\\
736 AH: ¿escribiendo?\\
737 DT: sí\\
738 AH: escribiendo\\
739 DT: muy bien, ese es el verbo que queremos\\
740 AH: escribiendo\\
741 DT: ¿sí escribe en el cuaderno?\\
742 AH: sí hombre\\
743 DT: ah pues ahí está\\
744 \# Condición: escritura.\\
745 AH: escribiendo, es, es, cri, escri, escri, biendo, biendo, escribiendo, escribiendo, escri \# escritura: «ESrBindo».\\
746 DT: a ver le ayudo, vamos a empezar otra vez \# borra para dejar «ES» únicamente--- a ver otra vez, esto esta bien, ¿cri?\\
747 AH: cri \# escritura: «ESc» agrega la letra c.\\
748 DT: ¿qué sigue?\\
749 AH: cri, bien \# escritura: «ESci» agrega la letra i.\\
750 DT: a ver ya se saltó varios sonidos \# borra las dos letras que acaba de escribir AH--- a ver, vamos a hacelo acá abajo \# señala la parte inferior del pizarrón--- escribiendo\\
751 AH: es, cri \# escritura: «ESC».\\
752 DT: ahí falta una /rr/\\
753 AH: ¿erre?\\
754 DT: sí\\
755 AH: ah pues sí \# intenta alcanzar el borrador.\\
756 DT: a ver pero sí va bien, aquí va \# señala el espacio vacío después de la letra C.\\
757 AH: escri, bie bi\\
758 DT: /b/\\
759 AH: bi, bie, escribie, bi bie, ndo \# escritura: «ESCriBiNdo».\\
760 DT: ahora sí la hizo bien esta, pero nos falta algo\\
761 AH: sí, ene\\
762 DT: está bien pero nos falta algo, vamos poco a poquito otra vez\\
763 AH: escri\\
764 DT: a ver vamos a empezar desde acá \# subraya «es»--- a ver aquí ¿qué dice?\\
765 AH: es\\
766 DT: \# subraya «cri».\\
767 AH: cri\\
768 DT: \# subraya «bi».\\
769 AH: ende, /e/ \# reconoce la letra faltante.\\
770 DT: eso\\
771 AH: ha cierto \# escritura final: «ESCriBiENdo».\\
772 DT: ¿ya vio que si vamos poquito a poco se da cuenta?\\
773 AH: ah sí cierto\\
774 DT: así vamos a estarle haciendo\\
775 AH: ah cierto\\
776 DT: muy bien eso es lo que nos faltaba ¿entonces qué dice todo completo?\\
777 \# Condición: lectura.\\
778 AH: yo estaba escribiendo en el cueaderno\\
779 DT: eso, así es como queremos que ya empiece a escribir\\
780 AH: ah cierto\\
781 DT: es medio redundante ¿no?\\
782 AH: ay sí hombre, bien difícil hombre\\
783 DT: sí está difícil, pero cuando empiece a escribir así, va a ver que se le va a facilitar un montón\\
784 \\
785 \# Segmento: escritura de diario.\\
786 \# Material: libreta de AH, pluma negra, pizarrón blanco y marcadores azul y rojo.\\
787 \# Condición: escritura.\\
788 \# Escrito: Fecha\_\_\_\_- Nombre\_\_\_\_- Lugar\_\_\_\_- 1 ¿Qué hice hoy?.\\
789 DT: es lo que hemos hecho en otras ocasiones\\
790 AH: fecha\\
791 DT: fecha, empezamos\\
792 AH: cinco, de, septiembre, de octubre, septiembre\\
793 DT: septiembre\\
794 AH: de septiembre, sep, se, de se\\
795 DT: sep /p/ \# hace énfasis en el sonido de /p/.\\
796 AH: ¿pe, pe?\\
797 DT: sí\\
798 AH: ti ¿ti? tiem\\
799 DT: a ver siga, como usted pueda hacerlo\\
800 AH: bre, no \# niega con la cabeza.\\
801 DT: está bien, no pasa nada ¿de qué año?\\
802 AH: dos mil veinticuatro, dos mil, veinticuatro \# escritura: «5 de sepime».\\
803 DT: a ver su nombre, si quiere no completo\\
804 AH: A\_ \# escibe su nombre con letra cursiva.\\
805 DT: que simpre le queda bien bonito\\
806 AH: ay hombre ¿sí?\\
807 DT: sí pues sí escribe bien bonito\\
808 AH: ajá\\
809 DT: y el lugar ¿se acuerda dónde estamos?\\
810 AH: aquí con los sicólogos\\
811 DT: sí pero la otra vez esbribimos otra cosa, la escuela ¿cómo se llama?\\
812 AH: la escuela\\
813 DT: ¿cómo se llama?\\
814 AH: en la escuela, en la escuela /.lo e sie si el sico sicol./ no \# se ríe.\\
815 DT: ¿cómo se llama la avenida? esta grandota que está aquí\\
816 AH: no me acuerdo\\
817 DT: /s/ ¿sa?\\
818 AH: (?)\\
819 DT: sara\\
820 AH: Zaragoza, ah Zaragoza, ¿za?\\
821 DT: sí\\
822 AH: za, zara, go\\
823 DT: a ver ahí vamos zara\\
824 AH: zarago\\
825 DT: go\\
826 AH: ¿el gato?\\
827 DT: sí\\
828 AH: zarago, zarago\\
829 DT: ahí sí va\\
830 AH: /s/ a \# alarga el sonido de la /s/--- Zaragoza \# escritura: «Caragosa».\\
831 \# Condición: conversación.\\
832 DT: a ver ¿y qué sigue?\\
833 AH: ¿qué hice hoy? \# lee la libreta.\\
834 DT: ¿a ver qué hizo hoy? con todos los verbos que aprendimos hoy, bueno no que aprendimos, que dijo ahorita\\
835 AH: ¿qué hice hoy? el parque, en el parque\\
836 DT: ¿qué hizo en el parque, se acuerda que usamos algo con futbol?\\
837 AH: sí\\
838 DT: a ver ¿cómo era?\\
839 AH: en el parque\\
840 DT: ¿pero qué hizo en el parque?\\
841 AH: en el parque ...\\
842 DT: ¿comió en el parque?\\
843 AH: no, en el parque, en el futbol, en el futbol\\
844 DT: ¿qué hacíamos con el futbol?\\
845 AH: en el futbol\\
846 DT: ¿se acuerda qué era?\\
847 AH: no\\
848 DT: ¿comemos futbol?\\
849 AH: no no\\
850 DT: ¿jugamos?\\
851 AH: sí jugamos, jugamos\\
852 DT: ¿cómo respondería?\\
853 AH: jugamos\\
854 DT: usted usted ¿qué hizo hoy?\\
855 AH: jugamos\\
856 DT: ¿nosotros?\\
857 AH: yo\\
858 DT: usted\\
859 AH: yo\\
860 DT: a ver póngalo\\
861 AH: ju\\
862 DT: yo\\
863 \# Condición: escritura.\\
864 AH: yo \# escritura: «yo».\\
865 \# Condición: conversación.\\
866 DT: ajá ¿qué dijimos?\\
867 AH: yo en el parque\\
868 DT: ¿pero qué era el verbo? ju\\
869 AH: /.fulba./ en el futbol\\
870 DT: pero ¿ju?\\
871 AH: jugamos, jugamos\\
872 DT: a ver\\
873 AH: yo jugamos\\
874 DT: a ver pero cómo\\
875 AH: yo jugamo jugara jugase jugamos\\
876 DT: a ver ahí va ahí va\\
877 AH: yo /j/\\
878 DT: usted, ajá, va bien va bien\\
879 AH: yo est\\
880 DT: ¿ju?\\
881 AH: jugando ¿jugamos? jugando jugamos\\
882 D: ¿yo jugamos?\\
883 AH: jugamos\\
884 DT: a ver vamos ahí a hacer una ayuda\\
885 AH: sí, ajá\\
886 DT: eso está difícil porque ya tiene tiempo que no lo trabajamos\\
887 AH: sí\\
888 DT: mire \# escribe en el pizarrón «Yo jugamos» la primera palbra en azul y la segunda en rojo--- si lo pongo así, léalo\\
889 AH: ah \# intenta copiar el la hoja blanca.\\
890 DT: léalo léalo, no lo copie\\
891 AH: yo jugamos\\
892 DT: ¿a poco así está bien?\\
893 AH: no no\\
894 DT: ¿cómo sería entonces?\\
895 AH: yo jugaré\\
896 DT: a ver\\
897 \# Condición: escritura.\\
898 AH: ju ju, jugaré \# no se aprecia la escritura.\\
899 \# Condición: conversación.\\
900 DT: sí está bien ¿pero qué cree? fíjese qué nos preguntó \# señala el escrito en el cuaderno de AH--- ¿qué hice hoy?\\
901 AH: ah\\
902 DT: ¿qué hice hoy, hoy jugaré, suena bien?\\
903 AH: no\\
904 DT: vamos a intentarlo otra vez\\
905 AH: sí\\
906 DT: pero está bien, ya me dijo bien el verbo, a ver acá abajo vamos a poner otra vez, yo\\
907 \# Condición: escritura.\\
908 AH: yo \# escritura: «yo».\\
909 DT: ¿de qué otra forma puede decir jugaré?\\
910 D: bueno, jugar\\
911 AH: jugaré, jugar\\
912 DT: a ver otra forma\\
913 AH: jugaré\\
914 DT: otra\\
915 AH: jugaré\\
916 DT: mire ya me dijo jugaré \# lo escibe en el pizarrón--- este ya esta bien, le vamos a poner su palomita\\
917 AH: yo jugaré\\
918 DT: ese está bien\\
919 D: ¿y si lo hubiera hecho? jugar, ayer\\
920 DT: ¿si fuera ayer?\\
921 AH: jugaré, jugaré\\
922 D: no no, pero ¿si hubiera sido ayer? porque si usted jugará, todavía no lo ha hecho\\
923 AH: no\\
924 D: ¿verdad?\\
925 AH: sí\\
926 DT: es para otro día\\
927 D: pero si usted ya hubiera jugado, cómo se diría?\\
928 DT: ayer\\
929 D: por ejemplo ayer\\
930 AH: yo jugaría, jugará, jugaría jugaría, jugaría\\
931 \\
932 \# Segmento: conjugación de verbos hablados.\\
933 \# Material: pizarrón blanco y marcadores azul y rojo.\\
934 \# Condición: conversación.\\
935 DT: estuvo bien porque nos dijo tiempos\\
936 AH: ah sí cierto\\
937 DT: porque nos dijo jugaré, jugaría, jugamos\\
938 AH: jugaría sí cierto\\
939 DT: eso es muy importante\\
940 AH: sí\\
941 DT: vamos a hacer un ejercicio, con jugar, seguimos en jugar\\
942 AH: jugaré\\
943 DT: a ver cómo lo diríamos, ¿mañana? \# muestra en el pizarrón la palabra «mañana».\\
944 AH: mañana jugaría\\
945 DT: eso muy bien\\
946 AH: jugaría jugaría\\
947 DT: a ver con este \# le da el marcador rojo.\\
948 AH: jugaría, jugaré\\
949 DT: ¿mañana?\\
950 AH: mañana, mañana, yo jugaré\\
951 DT: ajá también está bien\\
952 AH: yo jugaré, yo jugaré\\
953 DT: jugaré, ya lo dijo también\\
954 \# Condición: escritura.\\
955 AH: jugaré, ju, ga, re, jugaré \# escritura: «yo jugaRe».\\
956 \# Condición: conversación.\\
957 DT: eso muy bien, y sí tiene sentido ¿ya vio?\\
958 AH: pos (pues) sí hombre\\
959 DT: porque mañana jugaré\\
960 AH: ah mira\\
961 DT: ¿ya vio?\\
962 D: todavía no pasa\\
963 AH: sí\\
964 DT: en un plan, para mañana\\
965 AH: ah sí\\
966 DT: y si ahora por ejemplo ponemos así \# escribe «Ayer» en el pizarrón.\\
967 AH: ayer, jugaría\\
968 DT: a ver escúche ¿ayer jugaría?\\
969 AH: no, ayer, ayer jugué /.juguería./ ayer jugaré jugará jugaré\\
970 DT: no, a ver, ayer\\
971 AH: ayer\\
972 DT: ya pasó\\
973 D: algo que ya pasó\\
974 AH: ayer yo jugaría\\
975 DT: no esa ya está, esa ya dijimos que es mañana, esa ya está bien\\
976 AH: ah sí\\
977 DT: ¿pero ayer? ya pasó\\
978 AH: ha sí\\
979 DT: ya lo hicimos\\
980 AH: ajá cierto\\
981 DT: ¿cómo lo diría?a\\
982 AH: ayer yo, jugaría\\
983 DT: no porque ese es para mañana, ese ya está, ese ya está bien\\
984 AH: ayer /.jugueré./ jugaría\\
985 DT: no ese ya está\\
986 AH: ajá\\
987 DT: mire, en mañana ya está jugaría \# escribe «jugaría»--- estos dos ya están perfectos\\
988 AH: sí\\
989 DT: pero ahora necesitamos uno para ayer\\
990 AH: ayer jugaría\\
991 DT: este ya está mire \# subraya «jugaría».\\
992 D: ese ya no se puede\\
993 AH: jugará, jugará\\
994 D: estas ya no las puede decir \# señala «jugaré, jugaría» escritas en el pizarrón--- ninguna de las dos\\
995 DT: sí, estas ya se usaron\\
996 AH: sí\\
997 DT: estas ya no pasan a lo que sigue\\
998 AH: ayer\\
999 DT: ¿ayer?\\
1000 AH: ayer\\
1001 DT: este ya está, este ya está \# señala las palabras escritas en el pizarrón.\\
1002 AH: yo, jugaría, jugaría no\\
1003 D: ¿cómo dice que limpió ayer?\\
1004 AH: limpié\\
1005 D: ajá muy bien\\
1006 AH: limpié\\
1007 D: ¿y cómo se dirá que usted estaba jugando? pero ayer\\
1008 DT: a ver ¿cómo lo dijo? limpié ¿verdad?\\
1009 AH: limpié\\
1010 DT: muy bien porque esto ya pasó ayer \# escribe «limpié» debajo de «ayer».\\
1011 AH: ayer limpié\\
1012 DT: ayer limpié\\
1013 AH: ah mira\\
1014 DT: ¿y con jugar?\\
1015 AH: yo jugaría\\
1016 DT: no porque estamos en ayer\\
1017 AH: ayer limpié, limpié\\
1018 DT: y ayer ¿con jugar cómo sería?\\
1019 AH: limpiaría, limpiaré\\
1020 DT: eso sería en mañana, pero estamos en ayer, es más este ya lo vamos a quitar \# borra todo lo relacionado a «mañana» del pizarrón--- mañana ya lo hizo muy bien\\
1021 AH: sí\\
1022 DT: ahora estamos en ayer, algo que ya pasó\\
1023 AH: limpié\\
1024 DT: ¿y ahora con jugar?\\
1025 AH: limpiaría\\
1026 DT: no porque estamos en ayer, ahorita estamos ayer\\
1027 AH: ayer limpié, limpié\\
1028 DT: ese ya está, vamos a ponerle palomita\\
1029 AH: yo limpiaría\\
1030 DT: ¿tiene sentido eso?\\
1031 AH: no\\
1032 DT: esto está bien\\
1033 AH: ayer limpiaría\\
1034 DT: limpié ya lo acabamos, muy bien, ahora vamos a usar jugar \# escribe «jugar»--- ahora vamos a usar este ¿cómo quedaría con ayer?\\
1035 AH: ayer\\
1036 DT: ¿ayer?\\
1037 D: para que suene igual que este por ejemplo, pero con jugar \# señala «limpié».\\
1038 AH: jugar\\
1039 D: ayer ¿ju?\\
1040 AH: jugaré, jugaría\\
1041 DT: ese es para mañana ¿no?\\
1042 AH: ajá\\
1043 DT: bueno está bien vamos a cambiar esto \# borra «jugar»--- este lo dijo muy bien, a ver por ejemplo con comer\\
1044 AH: comer\\
1045 DT: ayer\\
1046 AH: ayer comí\\
1047 DT: ahí está\\
1048 D: muy bien\\
1049 AH: ayer comí\\
1050 D: y ese sí lo pudo decir bien\\
1051 AH: ayer comí\\
1052 DT: ahí está muy bien\\
1053 AH: ah mira\\
1054 DT: muy bien, a ver, con cocer\\
1055 AH: cocí\\
1056 DT: con cocer ¿ayer?\\
1057 AH: ayer cos, ayer\\
1058 DT: ¿ayer?\\
1059 AH: ayer cos\\
1060 DT: ya lo había dicho\\
1061 AH: ayer comimos\\
1062 DT: tiene sentido también pero ¿con cocer?\\
1063 AH: cocer, ah cocer\\
1064 DT: ¿ayer?\\
1065 AH: ayer cocí\\
1066 D: muy bien\\
1067 DT: ahí está\\
1068 AH: ayer cocí, ayer cocí ayer cocí\\
1069 DT: muy bien \# escibre en el pizarrón «cocí».\\
1070 AH: ah fíjate\\
1071 DT: muy  bien sí ya nos dijo un montón otra vez\\
1072 AH: sí\\
1073 DT: a ver ¿qué más? con trapear\\
1074 AH: trapeando\\
1075 DT: ¿ayer?\\
1076 AH: ayer trapié\\
1077 DT: ahí está\\
1078 D: muy bien\\
1079 AH: ayer trapié\\
1080 D: ¿con caminar?\\
1081 AH: ayer trapié, ayer trapié\\
1082 DT: ¿con caminar?\\
1083 AH: caminando, caminando\\
1084 DT: ¿ayer?\\
1085 D: ¿ayer?\\
1086 AH: ayer caminé\\
1087 DT: ahí está\\
1088 AH: ayer caminó caminé\\
1089 DT: muy bien\\
1090 D: ¿con jugar?\\
1091 AH: caminar, caminó caminé\\
1092 D: muy bien ¿y con jugar?\\
1093 AH: ¿mande?\\
1094 D: ¿con jugar, ayer?\\
1095 AH: ayer /.go co camimo./ caminamos, caminé \# lee el pizarrón.\\
1096 DT: todo esto ya nos lo dijo\\
1097 AH: ajá\\
1098 DT: y a ver ahora ¿jugar?\\
1099 AH: ...\\
1100 DT: ¿ayer?\\
1101 AH: ayer\\
1102 DT: jugar, ayer\\
1103 AH: ayer jugué\\
1104 DT: ahí está\\
1105 D: muy bien\\
1106 DT: sí pudo\\
1107 AH: ayer jugué\\
1108 DT: muy bien\\
1109 D: eso es lo que estábamos buscando\\
1110 AH: ayer jugué\\
1111 DT: mira ya todos lo verbos que nos dijo\\
1112 AH: ah pues sí\\
1113 D: y este que le había costado trabajo sí lo pudo decir ¿ya vio?\\
1114 AH: sí sí\\
1115 DT: todos estos son verbos\\
1116 AH: ah mira\\
1117 D: en pasado\\
1118 DT: en pasado\\
1119 AH: ah mira\\
1120 DT: los verbos, que es lo que hacemos\\
1121 AH: sí\\
1122 DT: lo podemos hacer ayer\\
1123 AH: ah mira\\
1124 DT: o ahorita o mañana ¿ya vio?\\
1125 AH: ah fíjate\\
1126 DT: ahora vamos a hacer verbos hoy \# borra todo el pizarrón y escribe «hoy»--- a ver ¿qué dice ahí?\\
1127 AH: hoy\\
1128 DT: ¿con comer? hoy ¿cómo lo diría?\\
1129 AH: hoy /.comiré comiré./ comimos\\
1130 DT: puede ser, está bien pero, voy a ponerle un ejemplo, voy a decir estoy comiendo, hoy estoy comiendo\\
1131 AH: hoy estoy comiendo\\
1132 DT: vamos a anotarlo aquí \# escribe «comiendo»--- muy bien, le ayudé pero se la vamos a poner buena\\
1133 AH: sí\\
1134 DT: a ver ahora con limpiar\\
1135 AH: limpiar los trastes, limpié\\
1136 DT: hoy\\
1137 D: pero como si lo estuviera haciendo ahorita\\
1138 AH: hoy trapié\\
1139 DT: eso ya nos lo dijo, pero ahora vamos a intentarlo así \# señala «comiendo»--- a ver ¿qué dice aquí?\\
1140 AH: comiendo\\
1141 DT: ¿hoy estoy?\\
1142 AH: estoy comiendo\\
1143 DT: ¿y ahora con trapear?\\
1144 AH: yo trapié, ya trapié, yo trapié\\
1145 DT: y ahora ¿hoy estoy?\\
1146 AH: ¿hoy?\\
1147 DT: hoy estoy\\
1148 AH: ...\\
1149 DT: tra\\
1150 AH: trapeando\\
1151 DT: ahí está, eso es lo que queremos\\
1152 AH: trapeando\\
1153 DT: muy bien \# escribe «trapeando».\\
1154 AH: sí\\
1155 DT: ya lleva dos\\
1156 AH: ah cierto\\
1157 DT: ahora con limpiar\\
1158 AH: limpiando el suelo\\
1159 DT: ahí está ya lo dijo, limpiando\\
1160 AH: el suelo\\
1161 DT: limpiando \# escribe «limpiando».\\
1162 AH: ajá liampiando\\
1163 D: ¿y con picar? de las verduras por ejemplo\\
1164 AH: picar pechuga \# se ríe.\\
1165 D: también se pueden picar las pechugas\\
1166 AH: sí\\
1167 DT: a ver pero vamos a intentar hacerlo así ¿picar?\\
1168 AH: picar\\
1169 D: ¿hoy estoy?\\
1170 AH: hoy piqué, picando la verdura\\
1171 DT: picando muy bien\\
1172 AH: la verdura\\
1173 DT: ¿y con reír?\\
1174 AH: reír tsi si tsi si \# intenta fingir una risa, y comienza a reír.\\
1175 DT: siempre se ríe mucho\\
1176 AH: sí cierto\\
1177 DT: a ver ¿y cómo quedaría así? \# señala las palabras escritas en el pizarrón.\\
1178 AH: estaba res ¿cómo?\\
1179 DT: estaba\\
1180 AH: estaba due estaba ¿qué?\\
1181 DT: a ver, reír\\
1182 AH: reír, con reír\\
1183 DT: pero lo queremos así \# señala las palabras escritas en el pizarrón--- ahorita, hoy\\
1184 AH: estaba risa no\\
1185 DT: cerca cerca\\
1186 D: ¿hoy estoy?\\
1187 AH: hoy estoy\\
1188 D: hoy esto ¿ri?\\
1189 AH: risa, riendo\\
1190 DT: ahí está\\
1191 AH: estaba riendo\\
1192 D: muy bien muy bien\\
1193 AH: riendo\\
1194 DT: ya nos dijo un montón otra vez\\
1195 AH: sí\\
1196 DT: un último, para alcanzar lo que hicimos hace rato, caminar\\
1197 AH: eh ...\\
1198 DT: con caminar\\
1199 AH: caminar, caminar\\
1200 DT: ¿hoy estoy?\\
1201 AH: hoy caminando, caminando\\
1202 DT: eso muy bien \# escribe «caminando»--- muy bien otra vez ya nos dijo seis\\
1203 AH: ah mira cierto\\
1204 DT: seis ayer y seis hoy\\
1205 AH: ah bueno\\
1206 DT: muy bien\\
1207 AH: sí\\
1208 D: ¿y todos estos qué son, son?\\
1209 AH: verbos\\
1210 D: muy bien\\
1211 DT: verbos\\
1212 AH: sí hombre\\
