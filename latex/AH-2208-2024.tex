\section{Sesión del 22 de agosto del 2024}
\noindent
\textbf{ID}: AH-2208-2024\\
Participantes:\\
\textbf{AH}: Mujer de 65 años, informante\\
\textbf{D}: Daniela Salinas, entrevistadora\\
\textbf{DT}: Dante Nava, entrevistador\\
\textbf{AC}: Mujer, familiar de AH\\
\\
\noindent
001 \# Segmento: preparación de los materiales.\\
002 \# Condición: conversación.\\
003 \# la grabación inicia en medio de una conversación entre AH y D.\\
004 ...\\
005 D: ¿nada?\\
006 AH: no\\
007 D: ¿entonces a dónde le gustaba salir?\\
008 AH: con mi esposo\\
009 D: ¿pero a dónde?\\
010 AH: a nadar\\
011 D: ¿a dónde sí le gustaba?\\
012 AH: pero este, nadar nada\\
013 D: nadar nada, ¿pero a dónde sí?\\
014 AH: este, bailado bailando también\\
015 D: ¿iban a bailar?\\
016 AH: sí sí, en Acapulco también\\
017 D: ahí sí le gustaba\\
018 AH: sí\\
019 D: le gustaba la fiesta\\
020 AH: con Veracruz también, Veracruz también sí con Veracruz\\
021 DT: ¿qué le gusta más Veracruz o Acapulco?\\
022 AH: Acapulco\\
023 DT: muy bien\\
024 \\
025 \# Segmento: orientación temporal.\\
026 \# Condición: conversación.\\
027 DT: ¿qué fecha es hoy?\\
028 AH: veintidos \\
029 D: ¿veintidos de qué mes?\\
030 AH: de octubre\\
031 DT: no\\
032 D: ¿segura?\\
033 AH: veintidos de pérame(espérame) veintidos de agosto, veinte de agosto veinte de agosto\\
034 DT: ¿de qué año?\\
035 AH: dos mil veinticuatro\\
036 DT: ¿y qué día de la semana es?\\
037 AH: jueves jueves\\
038 DT: ¿y cómo que hora será?\\
039 AH: pus(pues) no sé hombre\\
040 D: ¿aproximadamente?\\
041 DT: a ver más o menos\\
042 AH: a las once y media o las diez y media, diez y veinte, diez y veinte\\
043 DT: casi casi\\
044 D: muy bien\\
045 AH: diez y veinte\\
046 D: diez y cuarto\\
047 AH: ah mira\\
048 DT: por cinco minutos\\
049 AH: sí cierto\\
050 \# Segmento: series automatizadas e inversas.\\
051 DT: muy bien, a ver dígame los meses del año\\
052 AH: enero febrero marzo abril mayo junio julio agosto septiembre octubre noviembre diciembre\\
053 DT: ahora al revés\\
054 AH: diciembre, noviembre, este noviembre \# repite en voz baja y cerrando los ojos--- /s/ agosto ah no septiembre ago septiembre, pérame(espérame)\\
055 DT: sí sí\\
056 AH: septiembre agosto, di este agosto, y agosto \# repite en voz baja--- con los ojos cerrados no no\\
057 DT: está bien no pasa nada, a ver, ahora cuente del quince al treinta\\
058 AH: \# se aproxima a DT con duda.\\
059 DT: del quince\\
060 AH: del quince\\
061 DT: al treinta\\
062 AH: eh ¿cuánto?\\
063 DT: a ver, quince\\
064 AH: quince\\
065 DT: dieciséis\\
066 AH: dieciséis diecisiete, dieciocho diecinueve, \# los siguientes números los dice con mayor velocidad y omite los número entre 26 y 30--- veinte veintiuno veintidos veintitrés veinticuatro veinticinco treinta \# se ríe.\\
067 DT: bueno ahí se salto unos\\
068 AH: sí cierto\\
069 D: bueno está bien, muy bien\\
070 \\
071 \# Segmento: reporte de actividades.\\
072 \# Condición: conversación.\\
073 DT: platíqueme qué hizo, ¿qué le dejamos de tarea?\\
074 AH: este, las palabras\\
075 DT: a ver ¿podemos verlo?\\
076 AH: acción, cosa\\
077 DT: a ver ¿qué puso en acción? a ver si se acuerda\\
078 AH: este, cama\\
079 DT: a ver \# le devuelve la libreta a AH donde realiza sus actividades--- a ver, acción, acá estamos \# señala la columna que tiene escritas varias acciones en la libreta.\\
080 \# Condición: lectura.\\
081 \# Escrito: En la libreta de AH se aprecia una cabecera con la fecha: «fecha MiERoLeS 21 de agosto de 2024» Posteriormente se observan dos columnas, la de la izquierda se titulo «accion» y la derecha «Cosa». Debajo de la primera comlumna se observan las palabras: «TraPEAr, Comer, NadAr, Barer, PiNTar, Cocer, Hacer, ReMEMENdAR, PEiNar». Debajo de la segunda: «sala, tostadas, tocador, silla, Mesa, PoLLO, CoMida, ROPa, MUÑecA».\\
082 AH: acción\\
083 DT: a ver ¿qué puso?\\
084 AH: comer\\
085 DT: ajá\\
086 AH: nadar, ay pérame(espérame)\\
087 DT: ah sus lentes\\
088 AH: ajá cierto\\
089 DT: a ver ¿en acción qué puso?\\
090 AH: ¿acción?\\
091 DT: ajá\\
092 AH: a an, este, a a trapear, comer nadar, ba barrer, pintar, cocer, hice hacer, remendar /.re rei re./ reinar ¿cómo?\\
093 DT: a ver ¿este qué es?\\
094 AH: pei nar ah peinar, tocar, servir\\
095 DT: muy bien, a ver ¿qué es remendar, cómo se hace?\\
096 AH: así \# con ambas manos toca su playera y la mueve ligeramente de arriba hacia abajo.\\
097 DT: sí muy bien, y por ejemplo ¿peinar?\\
098 AH: pues ((así)) \# con ambas manos toca su pelo suavemente.\\
99 DT: eso y ¿servir?\\
100 AH: ah pos(pues) la sopa \# con ambas manos cerradas en puños, alterna una y otra adelante y atrás.\\
101 DT: la sopa muy bien\\
102 AH: sí\\
103 DT: a ver ahora ¿de cosas? a ver si se acuerda qué escribió\\
104 AH: a ver\\
105 DT: estas que me dijo son acciones\\
106 AH: sí sí\\
107 DT: y a ver ahora cosas\\
108 AH: este, cosas, leer\\
109 DT: no, esa es acción\\
110 AH: ajá\\
111 DT: a ver \# le muestra el escrito en su libreta.\\
112 AH: sala, ah\\
113 DT: cosas, estas, las cosas usted no las hace\\
114 AH: no no no\\
115 DT: ¿ya vio?\\
116 AH: sí cierto\\
117 DT: a ver, vamos a hacer otro ejercicio, a ver ¿una acción?\\
118 \# Condición: reconocimiento de categorías de palabras: acciones y cosas.\\
119 AH: trapear\\
120 DT: muy bien ¿y una cosa?\\
121 AH: este, leer\\
122 DT: esa es una acción\\
123 AH: sí sí sí\\
124 DT: ¿y una cosa?\\
125 AH: ¿cosa?\\
126 DT: sí\\
127 AH: este ...\\
128 DT: esto \# le muestra un libro pequeño.\\
129 AH: ...\\
130 DT: ¿qué es esto?\\
131 AH: lib la libreta\\
132 DT: ¿cómo se llama esto?\\
133 AH: ¿libreta libreta? \# se ríe.\\
134 DT: libro\\
135 AH: libro ajá sí sí\\
136 DT: ¿sí?\\
137 AH: sí sí\\
138 DT: ¿qué hace con un libro?\\
139 AH: este\\
140 DT: la acción\\
141 AH: acción\\
142 DT: ¿qué acción hace con el libro?\\
143 AH: este ...\\
144 DT: a ver, yo lo voy a hacer y usted me va a decir qué es \# comienza a hojear el libro pequeño.\\
145 AH: sí\\
146 DT: ¿qué estoy haciendo? a ver dígame\\
147 AH: leer, leer\\
148 DT: muy bien ¿y qué más hizo de tarea?\\
149 AH: este, aquí estaba en soriana(centro comercial)\\
150 DT: ¿fue al soriana?\\
151 AH: sola sola \# susurrando.\\
152 DT: ¿sola?\\
153 AH: sola\\
154 DT: ¿ya la dejaron ir sola?\\
155 AH: sola \# señala a AC.\\
156 AC: sola\\
157 DT: órale\\
158 AC: no muy seguido pero de vez en cuando pues va\\
159 AH: pero \# hace un gesto con la mano derecha de «negación» en repetidas ocasiones.\\
160 DT: no sabe\\
161 AH: no no\\
162 DT: ¿quién sabe entonces?\\
163 AH: \# señala a AC.\\
164 DT: ¿nada más?\\
165 AH: sí sí\\
166 D: ah ¿es secreto?\\
167 AH: sí verdad\\
168 D: está bien pero con mucho cuidado\\
169 DT: sí con mucho cuidado, por los coches y todo eso\\
170 AH: ah sí\\
171 DT: ¿y qué sintió de que fue al soriana?\\
172 AH: ay bien bonito hombre, bien bonito hombre\\
173 DT: ¿ya solita?\\
174 AH: sí\\
175 DT: ¿qué hizo ahí en soriana?\\
176 AH: este, crema, un bolillo\\
177 D: ¿eso compró?\\
178 AH: sí\\
179 D: muy bien ¿qué más compró?\\
180 AH: a un esprai(«spray» para el cabello) \# con la mano derecha se toca el pelo suavemente.\\
181 DT: ah para el cabellito\\
182 AH: sí sí\\
183 D: usted que siempre anda guapa\\
184 AH: sí ay no hombre, bien feo bien fea\\
185 DT: no cómo cree\\
186 D: para nada\\
187 DT: bueno pues nada más con mucho cuidado cuando vaya\\
188 AH: sí\\
189 DT: ¿y no le dan miedo los coches?\\
190 AH: no no\\
191 DT: ¿sí puede sola?\\
192 AH: sí cierto\\
193 DT: ¿y no se tropieza?\\
194 AH: no no, por mis pies, también bien feos, mi pies están bien bonitos \# se ríe.\\
195 DT: ¿y qué más hizo?\\
196 AH: trapié la sala, el pasillo\\
197 D: muy bien\\
198 \\
199 \# Segmento: habla espontánea en conversación.\\
200 \# Material: pizarrón blanco y marcador.\\
201 \# la conversación incia con D preguntando a AH qué más acciones realizó.\\
202 AH: este, sopa también\\
203 D: no pero la sopa no es una acción ¿cuál es la acción?\\
204 AH: este, trapié\\
205 D: ¿qué otra cosa?\\
206 AH: trapié \# susurrando--- la sala.\\
207 D: trapeó la sala\\
208 DT: a ver, la otra vez nos contó que en su cocina tiene muchas cosas\\
209 AH: ah sí\\
210 DT: ¿sí se acuerda?\\
211 AH: sí sí\\
212 DT: ¿a ver qué tiene en su cocina?\\
213 AH: en la casa \# niega con la cabeza--- en mi com ¿en qué?\\
214 DT: la cocina\\
215 AH: la cocina\\
216 DT: ¿qué hay ahí?\\
217 AH: la estufa, la sala \# niega con la cabeza.\\
218 DT: ¿la? a ver acuérdese\\
219 AH: el espérame, el (?), la estufa, el ¿qué?\\
220 DT: ajá la estufa\\
221 AH: el el ay hombre\\
222 DT: ahorita se acuerda\\
223 AH: ajá a ver, la estufa y ¿qué es? \# voltea a ver a AC.\\
224 DT: a ver acuérde no haga trampas\\
225 AH: la estufa, ay muchas tra mu/sh/ trastes\\
226 DT: trastes\\
227 AH: muchos trastes, muchos /.trest trestes./\\
228 DT: ¿qué más, dónde guarda la comida?\\
229 AH: en el refrigerador\\
230 DT: en el refrigerador ¿y dónde guarda el aceite y las latas?\\
231 AH: el aceite \# extiende su brazo izquierdo señalando al aire--- en la, en la ¿qué es?\\
232 DT: ahí está ya sabe qué es\\
233 AH: en la, ay dios bendito ...\\
234 D: ¿cómo es?\\
235 AH: aquí en la \# extiende ambos brazos con los dedos índices levantados, mueve ambos brazos hacia arriba ligeramente.\\
236 DT: ¿y cómo, se abre así? \# junta ambas manos con los puños cerrados para posteriormente abrirlos con un movimiento.\\
237 AH: sí sí aquí \# hace el mismo movimiento que DT. ajá ay dios bendito\\
238 DT: ¿empieza con /a/? \# alarga ligeramente el sonido.\\
239 AH: /.alacera./ alacena alacena\\
240 DT: muy bien ¿y qué hace usted en la cocina?\\
241 AH: este, la sopa\\
242 DT: ¿qué acciones?\\
243 AH: la sopa, /.pi pi piscadilla pec picadilla./ picadillo, ayer, picadillo\\
244 DT: ¿pero qué hizo con el picadillo?\\
245 AH: este, estaba bien sabrosa, pero bien sabroso, bien sabroso\\
246 DT: ¿y qué hizo con el picadillo, lo aventó?\\
247 AH: no no, en la en la \# junta y separa los puños tres veces.\\
248 DT: ¿lo vendió?\\
249 AH: no hombre no \# se ríe.\\
250 DT: ¿qué hizo entonces? a ver platíqueme\\
251 AH: en la /s/ en las \# junta y separa los puños ligeramente--- en la, las cucharas y la cuchara, este la cuch ...\\
252 DT: a ver si hacemos esto \# baja la mano derecha con el puño cerrado para lentamente aproximarlo a su boca, haciendo el gesto de «comer con cuchara».\\
253 AH: la sopa\\
254 DT: pero qué esto haciendo, a ver usted hágalo así \# juntando las puntas de los dedos de la mano derecha, la aproxima a su boca, haciendo el gesto de «comer».\\
255 AH: \# junta las puntas de los dedos de la mano derecha, los coloca sobre su labio inferior, abre la boca, aleja rápidamente la mano y cierra la boca para inmediatamente acerca la mano de nuevo.\\
256 DT: así \# repite el gesto de «comer». con la tortillita\\
257 AH: comiendo, comiendo\\
258 DT: muy bien, esa es la acción\\
259 AH: ajá comiendo\\
260 DT: ¿qué más?\\
261 AH: y sopa\\
262 DT: ¿pero qué otra acción?\\
263 AH: comiendo\\
264 DT: ya me dijo comiendo, muy bien, vamos a anotarlo acá \# escribe «comiendo» en el pizarrón blanco.\\
265 D: ya dijo comer, trapear\\
266 AH: ajá\\
267 DT: comiendo, trapear \# sigue escribiendo las palabras en el pizarrón.\\
268 AH: ah mira \# se ríe.\\
269 DT: ya nos dijo dos ¿qué más hizo? acciones\\
270 AH: ¿acciones? estaba en la tarea, la tarea, la tarea\\
271 DT: a ver, ahora ya dejamos la cocina, ahora la sala ¿qué hay en la sala, en su sala qué tiene?\\
272 AH: la mus la música, la música\\
273 DT: ¿y qué hace con la música?\\
274 AH: \# con su dedo índice izquierdo señala su oreja izquierda.\\
275 DT: ¿qué es, cómo se llama eso?\\
276 AH: la música estaba bien /s/ ((fuera)) bien fuerte bien fuerte\\
277 DT: ¿pero usted qué estaba haciendo?\\
278 AH: ...\\
279 DT: ¿se come la música?\\
280 AH: no hombre\\
281 DT: ¿entonces qué se hace?\\
282 AH: bai bailando, bailando\\
283 DT: ah muy bien\\
284 AH: bailando sí\\
285 DT: otra acción, ya nos dijo tres \# escribe «bailando» en el pizarrón--- una y ya\\
286 AH: ah bueno\\
287 DT: ¿qué tiene en su recámara?\\
288 AH: el tocador, el tocador\\
289 DT: ¿y qué hace?\\
290 AH: estaba en el tocar, estaba en la televisión, en la televisión\\
291 DT: ¿pero qué hace con la televisión?\\
292 AH: estaba bien, bien, la televisión estaba, las noticias, las noticias\\
293 DT: pero usted ¿qué estaba haciendo?\\
294 AH: en la, aquí en la /m/, en la\\
295 DT: ¿estaba bailando con la televisión?\\
296 AH: no no no\\
297 DT: ¿entonces qué hizo?\\
298 AH: \# comienza a reír--- espérame, estaba, ay dios bendito\\
299 DT: estaba ¿cocinando la televisión?\\
300 AH: no no hombre\\
301 DT: no se cocina la televisión ¿entonces qué se hace con la televisión?\\
302 AH: estaba en la, ay dios bendito, en las noticias en las noticias\\
303 DT: ¿qué estaba haciendo usted?\\
304 AH: las noticias\\
305 DT: ¿con los ojos?\\
306 AH: no, los los \# se lleva la mano a los labios, con las puntas de los dedos juntas, haciendo el mismo gesto usado previamente para «comer»--- los oídos, los oídos, con los oídos\\
307 DT: ¿estaba oyendo?\\
308 AH: sí con los oídos\\
309 DT: a ver diga, oyendo\\
310 AH: oí\\
311 DT: ¿oí qué?\\
312 AH: oí\\
313 DT: ¿qué oyó?\\
314 AH: en la televisión, la televisión\\
315 DT: muy bien, vamos a ponerlo aquí \# escribe «oí» en el pizarrón--- muy bien, cuatro verbos\\
316 \\
317 \# Segmento: conversación sobre estado de ánimo.\\
318 DT: a ver, platíquenos ¿cómo se ha sentido?\\
319 AH: bien gracias, gracias a dios\\
320 DT: ¿ya mejor?\\
321 AH: uy sí, sí\\
322 DT: porque la semana pasada la veíamos\\
323 AH: no pero también con mi nieta \# señala a AC (no es su nieta).\\
324 DT: ¿sí?\\
325 AH: ay sí qué bonita, qué bonita qué bonito, qué bonito\\
326 DT: a ver ¿por qué? cuéntenos\\
327 AH: bien bonita, bien bonita \# señala a AC mi nieta (?).\\
328 DT: ¿ya se siente mejor?\\
329 AH: ay sí hombre\\
330 DT: sí se nota\\
331 AH: sí\\
332 DT: sí, es que luego la notábamos que venía\\
333 D: muy triste o decaída\\
334 AC: y es que más se levantó, es que bueno, nosotros no vivíamos con ella, y tiene al lado al nieto consentido, al favorito\\
335 DT: ¿sí, su nieto, cómo se llama?\\
336 AH: D\_\\
337 D: ah tu esposo es su nieto consentido \# dice a AH.\\
338 AH: sí\\
339 D: ah qué bonito\\
340 AC: y de ahí como que se levantó\\
341 D: ¿viven con ella o viven al lado?\\
342 AC: nos mudamos al lado y entonces ahí ya luego le ayudo, luego la veo\\
343 D: ah qué padre\\
344 AC: o luego D\_ está con ella, así como que estamos más ahí\\
345 AH: sí\\
346 D: claro, ahí presentes\\
347 AH: sí dios bendito\\
348 DT: sí se nota que ya viene más\\
349 D: más feliz\\
350 AH: sí hombre bien feliz hombre, con mi nieto mi nieta, mi nieta\\
351 D: ahora ella va a ser la consentida\\
352 AH: ay sí\\
353 DT: a ver vamos a ver, un ejercicio aquí rápido\\
354 \# Condición: habla espontánea.\\
355 AH: bien burlones mis hijos, bien burlones\\
356 DT: ¿siguen sus hijos?\\
357 AH: uy sí\\
358 ... \# se le comenta a AC sobre evitar comentarios inapropiados sobre la condición de AH.\\
359 \# Segmento: reconocimiento de emociones con alternancia semiótica.\\
360 \# Materiales: pizarrón blanco y marcador rojo.\\
361 \# Dibujo: en el pizarrón hay dos representaciones simples de caras, una representa «feliz» y la otra «triste».\\
362 DT: a ver ¿esta qué es? \# señala la cara «feliz» en el pizarrón.\\
363 AH: la boca\\
364 DT: sí pero ¿¿cuál es la emoción?\\
365 AH: emoción, aquí mira, bien feliz\\
366 DT: bien feliz ¿y acá? \# señala la cara «triste» en el pizarrón.\\
367 AH: bien gra, bien este, bien ay bien qué\\
368 DT: esta es feiz ¿y esta? \# señala de nuevo la cara «triste».\\
369 AH: bien bien ¿qué? bien apachur bien pachurrada, bien apachurrada, bien apachurrada\\
370 DT: ¿triste?\\
371 AH: bien triste, bien triste\\
372 \# Condición: copia de dibujos.\\
373 DT: ahora las vas a copiar aquí \# le da una hoja blanca y una pluma azul--- a ver ¿cuál va a dibujar primero?\\
374 AH: este \# no se aprecia cuál de los dibujos señala.\\
375 DT: ¿cuál es?\\
376 AH: la cara, la cara feliz\\
377 DT: ah muy bien\\
378 AH: \# copia el dibujo de la cara feliz, sin embargo la boca da dibuja de forma invertida, el dibujo a copiar muestra un tercio de círculo como sonrisa, al copiarlo AH invierte este tercio de círculo, similar de la cara «triste».\\
379 DT: a ver ahora póngale aquí cuál es ¿es la feliz? \# señala con el dedo debajo de la copia de AH.\\
380 AH: feliz sí\\
381 DT: a ver escriba por favor\\
382 AH: fe, fe /l/ liz, feliz \# Escritura: «FeLis».\\
383 DT: muy bien ¿y cuál nos falta?\\
384 AH: bien\\
385 DT: a ver ahora dibúje esta por favor \# señala la cara «triste».\\
386 AH: \# comienza a copiar el dibujo, al llegar a la boca se detiene--- ay no bien fea \# finaliza dibujando una línea curva.\\
387 D: le salió bien\\
388 DT: muy bien\\
389 AH: feliz triste, feliz triste\\
390 DT: ah faltó escribir\\
391 AH: triste, triste, tri /s/, tris te \# escritura: «TrsTE».\\
392 D: ¿qué nos falta ahí?\\
393 AH: tris, tris\\
394 DT: a ver lea la palabra\\
395 AH: ¿e?\\
396 DT: no no no\\
397 AH: tris\\
398 DT: ¿dónde está la /i/ por ejemplo?\\
399 AH: ¿i?\\
400 DT: ajá\\
401 AH: ah, tri, tris \# escritura: agrega la letra i: «TrisTE».\\
402 DT: ah muy bien muy bien\\
403 D: muy bien\\
404 AH: ah mira\\
405 \\
406 \# Segmento: lectura de un párrafo de un libro infantil, comprensión y recuerdo de la lectura.\\
407 \# Material: libro infantil.\\
408 \# Condición: lectura.\\
409 \# Lectura: «El rey Tulio se sentía muy enfermo y muy triste Y tenía razón de sentir tanta tristeza. Sus médicos le habían dicho que la única forma en que podría mejorar su salud era que dejara de ser rey».\\
410 DT: por favor léanos los que dice ahí\\
411 AH: el rey, el rey (?) el rey /.cu tusio./ el rey Tulio Tulio, /.sen se sienta./ muy enfermo, muy tristeza muy tre muy triste y tenía razón de sentir tanta tristeza por sus /.meca medidi./ médicos, médicos /.de habla abiar de abi de abiar./ dicho que la /.uni unidad confor conforme en una./ en que podría mejorar su salud, era /s/ que dejara de rey, de de, re reinar.\\
412 \# Lectura: «-Yo no puedo dejar de ser rey- decía Tulio Así nací. Pero sus médicos insistían».\\
413 AH: yo no puede no pude dejar de ser rey decir /.jud jutu ju ju./ ¿Julio, /.tudio./?\\
414 DT: a ver ¿dónde vamos, acá?\\
415 AH: ajá este, Julio\\
416 DT: a ver vea bien la letra\\
417 AH: decía /.tu Julio./\\
418 DT: a ver la había dicho bien\\
419 AH: Tulio, Tulio\\
420 DT: muy bien\\
421 AH: así así /.nada nada para./ pero sus médicos /.esta en estaban médicos instan instante instiante./\\
422 \# Condición: preguntas sobre la lectura\\
423 DT: bueno hasta ahí está bien, y bueno, para terminar estar partecita ¿cómo se llamaba el rey?\\
424 AH: el rey chulo \# comienza a reírse--- ¿Tulio, el Tulio?\\
425 DT: eso muy bien\\
426 AH: Tulio\\
427 DT: ¿y cómo se sentía?\\
428 AH: bien enfermo, bien enfermo\\
429 DT: ¿y qué carita era la de él? \# señala los dibujos en el pizarrón.\\
430 AH: bien enfermo\\
431 DT: ¿la feliz?\\
432 AH: feliz\\
433 DT: ¿o la triste?\\
434 AH: bien triste, ((pero)) bien triste\\
435 DT: ¿y habló con alguien, el rey habló con alguien?\\
436 AH: no\\
437 DT: ¿alguien le había dicho algo?\\
438 AH: no\\
439 DT: ¿había médicos?\\
440 AH: sí, los médicos los médicos\\
441 DT: ¿le dijeron algo los médicos al rey?\\
442 AH: estaba bien enfermo\\
443 DT: muy bien muy bien\\
444 \\
445 \# Segmento: acciones como procesos.\\
446 \# Materiales: láminas con dibujos que representan acciones.\\
447 D: a ver primero dígame en esta qué es lo que observa, en esta primera\\
448 \# Material: lámina de una niña inflando un globo: 1) el globo está desinflado, 2) el globo está ligeramente inflando, 3) el globo está completamente inflado.\\
449 AH: el cabello\\
450 D: ¿es una niña o es un niño?\\
451 AH: una niña una niña\\
452 D: ¿qué está agarrando con las manos\\
453 AH: globo el globo\\
454 D: muy bien ¿de esta imagen a esta imagen ¿qué cambia? \# señala el primer cuadro de la lámina y posteriormente el segundo.\\
455 AH: el globo\\
456 D: ¿qué tiene el globo?\\
457 AH: \# acerca su mano derecha a la boca y sopla ligeramente.\\
458 D: sí ¿es más grande o más pequeño?\\
459 AH: grande\\
460 D: sí, está más grande ¿qué tal en esta? \# señala el último segmento de la imagen.\\
461 AH: el globo está bien enfermo \# se lleva ambas manos a la boca--- estaba bien feo \# se ríe.\\
462 D: okey, ¿cómo de le decimos a eso?\\
463 AH: apachurrado, apachurrado\\
464 D: sí por ejemplo aquí está un poquito apachurrado ¿pero cómo le llamamos a tener un globo, ponerlo en la boca y soplar aire, y qué pasa con el globo?\\
465 AH: bien grande\\
466 D: sí ¿pero cómo le decimos? que el globo se\\
467 AH: se infec\\
468 D: se ¿i?\\
469 AH: se infla, se infla\\
470 D: okey muy bien, infar es un verbo, es un verbo porque las acciones necesitan un proceso ¿se acuerda que una vez de hablé de eso?\\
471 AH: a pos (pues) sí cierto\\
472 D: este es el proceso, el que pase el globo de esto, a esto y luego a esto ¿podemos ver cómo cambia en el tiempo verdad?\\
473 AH: sí sí bien inflado\\
474 D: y por eso es una acción o un verbo\\
475 AH: acciones, acción\\
476 D: ¿entonces cómo quedamos que se llama esta acción, la acción de?\\
477 AH: el globo\\
478 D: ¿pero cuál era la acción? el globo no es la acción, la acción es ¿i?\\
479 AH: inflado inflado\\
480 D: entonces \# escribe en el pizarrón «acción» y «cosa»--- ¿inflar dónde va?\\
481 AH: cosas\\
482 D: no\\
483 AH: acción\\
484 D: ¿por qué? porque conlleva un proceso, y entonces el globo dónde irá?\\
485 AH: acciones\\
486 D: no\\
487 AH: cosa\\
488 D: le vamos a escribir aquí hasta arriba a todo\\
489 AH: ah inflar\\
490 D: \# escribe en la parte superior de la lámina «inflar».\\
491 AH: ah mira\\
492 D: donde todo esto es inflar, todos estos tres\\
493 AH: ah sí cierto\\
494 D: y si bien se hace la acción con una cosa, que en este caso es el globo, el globo no es la acción ¿verdad?\\
495 AH: no no, cosa\\
496 D: entonces el globo ¿cosa o acción?\\
497 AH: cosa cosa\\
498 D: entonces aquí escríbame globo \# señala la zona del pizarrón donde está escrito «cosa».\\
499 \# Condición: escritura.\\
500 AH: ¿globo?\\
501 D: sí acá\\
502 AH: glo, glo \# Escitura: g.\\
503 D: /g/ /l/\\
504 AH: ¿glo?\\
505 D: ¿este cómo suena? \# señala la letra g.\\
506 AH: ga\\
507 D: ¿cuál es el siguiente sonido?\\
508 AH: glo\\
509 D: glo\\
510 AH: ¿o?\\
511 D: antes de la /o/ nos falta uno /l/, ¿qué sonido es?\\
512 AH: ¿ele?\\
513 D: sí\\
514 AH: ele, ele\\
515 D: a ver escriba aquí una ele \# señala la parte inferior del pizarrón.\\
516 AH: ele, ele \# escritura: F--- ele\\
517 D: esta suena /f/\\
518 AH: e ¿gato?\\
519 D: no, como en luz\\
520 AH: globo\\
521 D: a ver escríbame aquí luz \# señala una parte vacía del pizarrón.\\
522 AH: luz luz \# escitura: «Lus».\\
523 D: ¿ya vio que aquí sí pudo escribirla?\\
524 AH: ah\\
525 D: esa es la ele, globo\\
526 AH: \# Escitura: L.\\
527 D: ¿glo?\\
528 AH: ah \# Escitura: o--- ¿bo, bo?\\
529 D: sí, bo\\
530 AH: ¿bo?\\
531 D: /b/\\
532 AH: ¿be?\\
533 D: sí\\
534 AH: be, glo, bo, glo, bo\\
535 D: ahora léame qué dice\\
536 AH: globo\\
537 D: no, léalo con cuidado, hay un error y quiero que usted lo identifique\\
538 AH: a ver\\
539 D: ¿cuál es?\\
540 AH: be\\
541 D: ¿dónde está nuestra be?\\
542 AH: \# borra un elemento pero no se aprecia en el video.\\
543 D: /b/\\
544 AH: ¿R\_? \# menciona el nombre de un familiar, apoyo que usa para hallar fonemas y grafías.\\
545 D: esa suena /rr/ y nosotros queremos /b/\\
546 AH: ¿be?\\
547 D: como en barco, a ver escríbame aquí barco\\
548 AH: a mira, globo \# escitura final: «glovo».\\
549 D: okey con esa la vamos a dejar porque suenan igual, ahora el siguiente es este \\
550 \# Material: lámina de crecimiento de un árbol 1) un niño coloca semillas en un agujero, 2) un árbol pequeño, 3) un árbol grande.\\
551 D: este es otro verbo, todo esto es un verbo ¿qué está haciendo el niño?\\
552 AH: el árbol estaba, plantiando plantando plantando\\
553 D: muy bien, de aquí podemos sacar dos verbos, uno es este que ya me dijo, plantar ¿si es un verbo qué es, una acción o una cosa?\\
554 AH: cosa, acción acción\\
555 D: entonces ese lo vamos a poner porque usted lo dijo, plantar \# señala debajo de la palabra «acción» escrita en el pizarrón.\\
556 AH: plantar, pla, plan \# escitura: «PLaN».\\
557 D: /t/\\
558 AH: ¿te?\\
559 D: sí\\
560 AH: ah te, plan tar ¿erre? \# escitura: «Ta».\\
561 D: muy bien\\
562 AH: ah cierto \# escitura final: «PLaNTar».\\
563 D: muy bien, le salió muy rápido\\
564 AH: sí\\
565 D: ¿el iguiente cuál es, aquí el arbolito de qué tamaño es?\\
566 AH: pequeñita, pequeñito\\
567 D: ¿y aquí?\\
568 AH: bien grande, bien mediano, mediano mediano\\
569 D: okey aquí es mediano ¿y qué tal aquí?\\
570 AH: el árbol bien grande\\
571 D: entonces ¿a todo eso cómo le llamamos, que el árbol?\\
572 AH: ...\\
573 D: ¿qué pasa con el árbol? primero está así \# señala el primer dibujo de la lámina--- ¿luego?\\
574 AH: bien /f/ bien grande, la mitad\\
575 D: mediano ¿no?\\
576 AH: mediano, mediano\\
577 D: ¿y luego?\\
578 AH: bien grande\\
579 D: ¿cómo se le llama a eso, cuál es la acción?\\
580 AH: el árbol está bien fea, bien feo\\
581 D: ¿qué pasa con el árbol?\\
582 AH: bien grande\\
583 D: ¿pero qué pasa, cuál es la acción, el verbo?\\
584 AH: ..\\
585 D: ¿cre?\\
586 AH: crece, crece\\
587 D: muy bien, el árbol crece ¿cierto?\\
588 AH: sí\\
589 D: ¿y eso qué es? acuérdese que las acciones son todo esto\\
590 AH: crece\\
591 D: ¿pero qué es, es una acción o es una cosa?\\
592 AH: acción\\
593 D: muy bien muy bien\\
594 AH: acción\\
595 D: a ver escríbame aquí crecer \# señala una parte en blanco del pizarrón debajo de la palabra «plantar».\\
596 AH: cre cre /s/ \# escitura: «cr».\\
597 D: ¿cuál es la que sigue?\\
598 AH: sa ser ¿e?\\
599 D: primero cre ¿qué nos falta ahí?\\
600 AH: cre ser\\
601 D: ¿cuál es la siguiente para que diga cre?\\
602 AH: ¿e?\\
603 D: ajá ¿cuál es la e?\\
604 AH: ¿E\_? \# escitura: E, menciona el nombre de otro familiar como apoyo.\\
605 D: muy bien, crecer\\
606 AH: /s/ ¿ese?\\
607 D: bueno la vamos a dejar con esa\\
608 AH: crecer crecer \# escitura final: «crEser».\\
609 D: muy bien muy bien, ya tenemos otro ¿verdad?\\
610 AH: sí cierto\\
611 D: aquí quedamos que el árbol, tanto se planta como crece ¿verdad? a ver escríbame aquí árbol \# señala una zona en blanco del pizarrón debajo de la plabra «globo».\\
612 AH: árbol\\
613 D: esa es la cosa\\
614 AH: árbol, árbol \# escitura: «ArPOL».\\
615 D: ¿qué está mal aquí?\\
616 AH: pe\\
617 D: esa la puede convertir ¿no? \# señala la letra P.\\
618 AH: ah con be \# escritura: cambia la P por la B.\\
619 \# Material: lámina con dibujos que representan el proceso de ponerse un calcetín: 1) pie sin nada puesto, 2) pie con un calcetín a la mitad, 3) pie cubierto por el calcetín.\\
620 D: ¿aquí cuál será el verbo, qué vemos aquí?\\
621 AH: el pie\\
622 D: ¿y ahora qué tiene aquí el pie?\\
623 AH: la sal este, la, los calcetines\\
624 D: ¿y cuál es la diferencia de esta imagen a esta? \# señala primero la imagen donde el calcetín está a la mitad del pie y después señala la imagen en la que el calcetín está puesto en el pie completamente.\\
625 AH: los calcetines están bien feos \# se ríe.\\
626 D: sí puede que estén feos pero ¿cuál es la diferencia, dónde está aquí el calcetín y dónde esta aquí el calcetín? \# señala las imágenes en el mismo orden.\\
627 AH: en el ¿quién? en este, en el calcetín el calcetín, estaban bien feos \# señala la imagen con el calcetín completamente puesto en el pie.\\
628 D: ¿pero el calcetín dónde está todavía mal puesto?\\
629 AH: ah, ¿bien qué?\\
630 D: ¿dónde está mal puesto?\\
631 AH: mal puesto, aquí \# señala la imagen con el calcetin a medio poner.\\
632 D: ¿y dónde está bien puesto?\\
633 AH: bien feo bien benito bien bonito bien bonito \# señala la imagen con el calcetín completamente puesto.\\
634 D: muy bien ya está bien puesto aquí ¿verdad?\\
635 AH: sí\\
636 D: ¿y cómo se le llama a la acción meter el piesito en el calcetín, cómo le llamos a eso?\\
637 AH: los calcetines están bien bien\\
638 D: sí pero los calcetines los /p/\\
639 AH: ¿calcetines?\\
640 D: ¿los po?\\
641 AH: pone, me pone me pon\\
642 D: ajá ¿usted se qué?\\
643 AH: me /.poné el tel el caltes./ el calcetín\\
644 D: ¿pero alguien se los pone o usted se los pone?\\
645 AH: yo yo\\
646 D: ¿entonces, yo me?\\
647 AH: yo me\\
648 DT: a ver todo completo dígalo ¿qué hace usted con los calcetines?\\
649 AH: yo me yo me yo me voy\\
650 D: yo me ¿po?\\
651 AH: me pongo el calcetín\\
652 D: muy bien muy bien ¿ya vio?\\
653 \\
654 \# Segmento: reconocimiento de categorías de palabras: cosas y acciones.\\
655 \# Materiales: láminas que representan acciones (misma que en el segmento previo).\\
656 D: ¿aquí la cosa qué quedamos que era? \# señala la lámina con las imágenes del árbol--- ¿el qué?\\
657 AH: el árbol\\
658 D: ¿aquí cuál es la cosa? \# señala la imagen con el globo siendo inflado.\\
659 AH: el globo el globo\\
660 D: y esas son las cosas no lo verbos ¿aquí cuál es la cosa? si aquí era el globo ¿aquí cuál es la cosa? \# le muestra la lámina con las imágenes del calcetín.\\
661 AH: la\\
662 D: la cosa\\
663 AH: acción\\
664 D: no no pero la cosa\\
665 AH: ...\\
666 D: ¿qué es esto?\\
667 AH: el calcetín\\
668 D: esa es la cosa ¿entonces la acción cuál es?\\
669 AH: calcetín\\
670 D: la acción\\
671 AH: acción acción acción\\
672 D: sí pero en estas imágenes ¿cuál es la acción? \# señala la lámina con el proceso de «ponerse un calcetín».\\
673 AH: acción acción\\
674 D: ¿pero cuál es?\\
675 AH: acción\\
676 D: a ver vea esta ¿y cuál es la acción?\\
677 AH: calcetín\\
678 D: no esa es la cosa ¿cuál quedamos que era la acción\\
679 AH: calcetín\\
680 D: a ver ¿qué dice aquí?\\
681 AH: ponerse\\
682 D: esa es la acción\\
683 AH: ponerse\\
684 D: a ver aquí escríbalo \# señala la zona del pizarrón de acciones--- o poner, como usted quiera\\
685 \# Condición: escritura.\\
686 AH: poner poner /s/ se, ponerse \# escritura: «PoNERce».\\
687 D: ¿y entonces qué nos ponemos? \# señala la zona del pizarrón de cosas.\\
688 AH: cosas\\
689 D: ¿y en este caso cuál era la cosa, el?\\
690 AH: no ya no, este acción\\
691 D: ajá pero aquí en las cosas ¿qué nos ponemos un qué?\\
692 AH: acción\\
693 D: era un ¿cal?\\
694 AH: calcetín, calcetín\\
695 D: sí esa era la cosa, a ver escríbalo aquí \# señala la zona del pizarrón de cosas.\\
696 AH: cal, cal ce, cal ce, ti, calcetines \# escritura: «CaLcETiN».\\
697 D: así está bien\\
698 AH: ajá\\
699 \# Condición: conversación.\\
700 D: ¿le sigue costando mucho trabajo?\\
701 AH: no no\\
702 D: ¿crees que ya pude diferenciar las acciones de las cosas?\\
703 AH: sí hombre sí\\
704 D: ¿qué tienen las acciones que las cosas no?\\
705 AH: este cosas aquí \# señala la zonas del pizarrón donde se escribieron cosas en las actividades anteriores.\\
706 D: a ver dígame un ejemplo de una cosa, dígame algo que usted veo o se acuerde que sea una cosa\\
707 AH: aquí en la puerta ¿cómo? la puerta\\
708 D: la puerta es una cosa, muy bien\\
709 AH: sí la puerta\\
710 D: hora déme un ejemplo de una acción, una acción, una cosa no, acuérde que las acciones tienen cambios ¿verdad?\\
711 AH: la silla la silla\\
712 D: la silla no, es una cosa, estas están más difíciles ¿verdad? \# señala la zona del pizarrón donde se escribieron acciones.\\
713 AH: ah pues sí hombre\\
714 D: ¿pero qué podemos hacer con la silla? por ejemplo ¿cómo se le llama a pasar de esta posición a esta? \# se levanta de la silla y vuelve a sentarse.\\
715 DT: ¿qué hizo?\\
716 D: ¿yo me?\\
717 AH: ...\\
718 D: ¿cómo se le llama a pasar de esta posición a esta otra? \# se levanta y vuelve a sentarse en su asiento.\\
719 DT: ¿qué hizo?\\
720 D: ¿cómo se le llama a eso? yo me se\\
721 AH: yo me me ca me me ¿qué?\\
722 D: yo me se\\
723 AH: me senté me senté\\
724 D: muy bien esa sí es la acción porque vio que me moví y vio que hubo un cambio\\
725 AH: me senté me senté\\
726 D: pasé de estar parada a estar sentada ¿verdad?\\
727 AH: me senté me senté\\
