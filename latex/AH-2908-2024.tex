\section{Sesión del 29 de agosto del 2024}
\noindent
\textbf{ID}: AH-2908-2024\\
Participantes:\\
\textbf{AH}: Mujer de 65 años, informante\\
\textbf{D}: Daniela Salinas, entrevistadora\\
\textbf{DT}: Dante Nava, entrevistador\\
\textbf{AC}: Mujer, familiar de AH\\
\\
\noindent
001 \# Segmento: reporte de actividades.\\
002 \# Material: cuaderno de AH.\\
003 \# Condición: lectura.\\
004 D: muy bien ahora sí muéstreme qué trae de tarea\\
005 AH: \# toma su libreta--- a ver, este \# le da su libreta a D.\\
006 D: muy bien usted lo escribió ¿verdad?\\
007 AH: sí sí\\
008 D: muy bien ¿esto cuándo lo escribió?\\
009 AH: este \# con la mirada busca en su libreta--- veintitres.\\
010 D: muy bien \# lee el contenido de la libreta--- ¿elevadores?\\
011 AH: ajá, con elevadores estaban aquí en, en, estaba en ((el)) Chapultepec\\
012 D: ah ¿fueron a Chapultepec?\\
013 AH: no, este \# señala con su dedo índice izquierdo una línea de texto de su libreta (no se aprecia en el video).\\
014 D: ¿no?\\
015 AH: mira, este \# señala su libreta.\\
016 D: ¿pero eso cuándo fue?\\
017 AH: \# voltea a ver a AC y mueve la cabeza ligeramente hacía arriba.\\
018 AC: a donde ella trabajaba\\
019 AH: sí sí \# se ríe.\\
020 \# se muestra a cámara los escitos de AH.\\
021 \# Escrito: (hoja 1) 24 de agosto 2024- aciendo el qiaser- trpabagando- las comidas- elebadores trapiando- escaleras el baño- las oficinas chiles- reyellenos bien sabroso- especiales los dulces- trpabagando 12 años- pencionada Ror el ceguro- montes urales Chapultepec. (hoja 2) 26 de agosto fecha 2024- yo desayune atole y pan tostado- y frigoles una guesadilla- yo comi dos sopes y sopa- de bertura y agua y cafe- con leche- yo cene sopa de fideo i- longanisa con huebo- yo desayune sopes y atole- yo comi sopa de fideo sopes y agua. (hoja 4) fecha- 28 de agosto 2024- los intocables- yo trapie la sala las escaleras- yo boy a bañor- mucho umadero las tostadas- el agua estaba irbindo.\\
022 \# suspención momentánea por interrupción.\\
023 \\
024 \# Segmento: habla espontánea.\\
025 \# Condición: conversación.\\
026 D: cuénteme ¿qué hizo el lunes?\\
027 AH: el lunes trapié las escaleras\\
028 D: ¿de su casa?\\
029 AH: sí sí, trapié y duerme y duerme también\\
030 D: duerme mucho\\
031 AH: sí sí\\
032 D: ¿y qué más  hizo?\\
033 AH: y la tarea la tarea\\
034 D: sí la tarea ¿y qué más?\\
035 AH: este, cené, ay hombre no \# niega con la cabeza.\\
036 D: ¿por qué?\\
037 AH: cené café con leche\\
038 D: ay qué rico\\
039 AH: sí\\
040 D: ¿qué más hizo?\\
041 AH: y pes, y, sopa sopes, y fideo, de fideo\\
042 D: ¿eso lo cenó o lo comió?\\
043 AH: comimos cenamos y merendamos\\
044 D: muy bien ¿y en la tarde qué hizo?\\
045 AH: siempre durmiendo, las noticias por las noticias, duerme y duerme\\
046 D: ¿pone las noticias para dormirse?\\
047 AH: sí sí\\
048 D: ¿la arrullan?\\
049 AH: sí ajá cierto\\
050 D: ah muy bien ¿en dónde se duerme en su cuarto o en la sala?\\
051 AH: en mi cuarto, en mi cuarto\\
052 D: muy bien ¿hizo algo más ese día?\\
053 AH: este, ah, estaba durmiendo y, no sé no sé\\
054 D: ¿no se acuerda?\\
055 AH: no no\\
056 D: ¿usted sabe si todo eso es verdad? \# pregunta a AC.\\
057 AC: sí trapea, barre\\
058 AH: sí sí\\
059 D: ¿pero específicamente el lunes?\\
060 AC: pues prácticamente diario\\
061 AH: sí sí\\
062 AC: diario\\
063 D: ¿qué le gusta hacer a usted?\\
064 AH: este, pues no sé\\
065 D: le gustaría por ejemplo\\
066 AH: /.berdar./ bordar bordar \# interrumpe a D.\\
067 D: ah\\
068 AH: bordar bordar bordar\\
069 D: ah tejer, muy bien\\
070 AH: bordar siempre bordar, bordé, siempre bordé\\
071 D: qué bonito así con ganchitos\\
072 AH: sí sí\\
073 AC: (?) en aquel tiempo nuestros papás, bueno mi mamá ella acostumbraba a\\
074 AH: /.berdar./ bordar sí\\
075 AC: hacer sus sábanas pero de puro remiendo\\
076 D: de retazitos\\
077 AC: puros retazitos, y ella a puro coser y coser\\
078 AH: sí yo también, sí sí\\
079 D: ¿y eso le gusta?\\
080 AH: uy sí\\
081 D: un montón\\
082 AH: sí\\
083 D: sería muy bueno que tuviera una actividad que la mantenga ocupada a parte de las actividades de la casa\\
084 AH: ah mira\\
085 D: aun sí le cuesta trabajo moverse, caminar todos los días un poquito siempre es bueno\\
086 AH: ah pus (pues) sí\\
087 D: ¿camina?\\
088 AH: sí sí \# dirije discretamente la mirada a AC con la intención de que este no lo note.\\
089 D: muy bien \# se ríe.\\
090 AC: sí pues anda de aquí para allá y de allá para acá dentro de la casa, y luego la bajamos que a soriana, que a la tienda\\
091 D: claro\\
092 AC: y tiene que bajarse cinco\\
093 AH: escaleras, escaleras\\
094 AC: pisos\\
095 D: ah vive hasta arriba\\
096 AH: sí sí\\
097 D: órale\\
098 AC: pero es que ((luego le digo)) vamos allá, no ya no\\
099 D: ya se cansa, sí aunque sea un poquito, una media horita que camine diario, con eso\\
100 \\
101 \# Segmento: reconocimiento de categorías de palabras: acciones y cosas.\\
102 \# Material: pizarrón blanco, borrador y marcadores azul y rojo; el pizarrón se encuentra dividido en dos columnas, la de la derecha dice cosas y la de la izquierda dice acciones.\\
103 \# Condición: conversación.\\
104 D: a ver vamos a empezar ¿cortar es una acción o es una cosa?\\
105 AH: este...\\
106 D: a ver enséñeme cómo corta, por ejemplo si esta fuera una tabla de cortar y este su cuchillo \# le señala el pizarrón y le aproxima el borrador.\\
107 AH: \# antes de que D termine la explicación, usa ambas manos de forma alternada, con las palmas de las manos extendidas, golpea suavemente la mesa con los costados de las manos, toma el borrador y lo usa para golpear suavemente en repetidas ocasiones el borrador.\\
108 D: ajá, por ejemplo si este fuera una zanahoria ¿cómo la agarraría? \# le da el marcador.\\
109 AH: \# toma el marcador en posición de escritura.\\
110 D: una zanahoria\\
111 AH: \# hace varios trazos cortos verticales en el pizarrón.\\
112 D: sí pero imagínese que esta es la zanahoria\\
113 AH: sí\\
114 D: tiene aquí una puntita y un rabito ¿verdad?\\
115 AH: sí\\
116 D: a ver agárrelo como agarraría una zanahoria para cortarla\\
117 AH: \# de nuevo toma el marcador en posición de escritura, realiza un trazo vertical y posteriormente lo divide con tres trazos horizontales.\\
118 D: ¿no la agarraría así para cortarla? por ejemplo con el cuchillo \# toma el marcador y lo coloca sobre la mesa de forma horizontal, con el borrador le da pequeños golpes a lo largo, buscando representar la acción de «cortar con un cuchillo».\\
119 AH: sí \# se ríe.\\
120 D: eso qué es ¿es una acción o es una cosa?\\
121 AH: una acción.\\
122 D: muy bien, a ver escríbame cortar\\
123 \# Condición: escritura.\\
124 AH: co, cor, tar \# escritura: «corTAR» en la columna de acciones.\\
125 D: muy bien, y por ejemplo ¿la silla qué es una acción o una cosa?\\
126 AH: cosa, la cosa, cosa cosa\\
127 D: muy bien muy bien\\
128 AH: si, lla \# escritura: «ciLLA» en la columna de cosas.\\
129 \# Condición: conversación.\\
130 D: ahora si usted leyera ¿le gusta leer la biblia?\\
131 AH: sí sí\\
132 D: ¿le gusta leer las oraciones?\\
133 AH: sí sí\\
134 D: imagínese que esta es su biblia \# hojea el cuaderno de AH.\\
135 AH: sí\\
136 D: ¿cómo la lee? haga como que lee \# lee da el cuaderno a AH.\\
137 AH: leer \# toma el cuaderno y hojea un par de veces--- leer\\
138 D: ajá muy bien ¿eso qué es, una cosa o es una?\\
139 AH: acción acción \# interrumpe a D.\\
140 D: muy bien \# le da el marcador azul.\\
141 AH: ¿a?\\
142 D: sí pero ¿cuál es la acción?\\
143 AH: la ...\\
144 D: \# toma el cuaderno de AH y comienza a hojear.\\
145 AH: libro\\
146 D: no no, la acción, o sea esto \# continua hojeando el cueaderno.\\
147 AH: ...\\
148 D: leer ¿no?\\
149 \# Condición: escritura.\\
150 AH: leer sí leer, le, le, er, leer \# escritura: «LeeER» en la columna de acciones.\\
151 D: a ver lea lo que usted escribió\\
152 AH: le, er \# señala la palabra escrita.\\
153 D: ¿cuántas /e/ lleva leer?\\
154 AH: ¿e?\\
155 D: ¿cuántas de estas lleva leer? \# señala las letras «eeE» que escribió AH.\\
156 AH: le, e, ah, leer \# escritura: borra «eER» y agrega E, resultando en «LeE»--- le, er, e, e \# escritura: «LeErr»--- no \# borra la última y con la parte trasera del marcador realiza un trazo al final de la palabra--- ¿e?\\
157 D: no\\
158 AH: le, er\\
159 D: a ver léala\\
160 AH: le, er \# alarga el sonido del fragmento «er», voltea a ver a D.\\
161 D: ya está ¿no?\\
162 AH: ah cierto\\
163 D: por ejemplo, su chamarrita ¿qué es, una acción o es una cosa?\\
164 AH: una cosa, una cosa\\
165 D: muy bien, muy bien\\
166 AH: chamarra, cha \# escritura: «cha»--- ma, ma\\
167 D: ma\\
168 AH: ¿ma?\\
169 D: sí ¿qué sonido sigue?\\
170 AH: ah, cha, ma, marra, chamarra \# escritura: «chamar»--- chamarra\\
171 D: faltan algunas cositas\\
172 AH: \# borra la letra r.\\
173 D: no, no está mal, pero faltan cosas\\
174 AH: chama\\
175 D: marra \# hace énfasis en el sonido de la doble erre.\\
176 AH: chamarra \# pronuncia con mayor fuerza la doble erre. \# escritura: «chamara»--- ¿sí?\\
177 D: ¿falta algo aquí? \# señala la letra r.\\
178 AH: chama, rra \# pronuncia con fuerza el sonido de la doble erre--- erre\\
179 D: ajá ese está bien, pero aquí nos faltaría algo para que sea /rr/ y no /r/\\
180 AH: ¿R\_? \# menciona el nombre de un familiar como apoyo para reconocer el sonido de /rr/.\\
181 D: ajá nos falta una\\
182 AH: ¿erre?\\
183 D: voy a borrar esta \# borra la letra a.\\
184 AH: \# escritura final: agrega las dos letras faltantes resultando en «chamarRa».\\
185 \# Condición: conversación.\\
186 D: muy bien muy bien, por ejemplo si yo salto, imagínese que esta soy yo \# con la mano sobre la mesa usa los dedos índice y medio, sube y baja la mano en repetidas ocasiones tocando la superfici de la mesa únicamente con las puntas de los dedos.\\
187 AH: sí\\
188 D: y estoy saltando por ejemplo, no sé, la cuerda ¿eso es una acción o es una cosa?\\
189 AH: sal, cosa\\
190 D: estoy saltando\\
191 AH: ajá, acción \# lee la columna escrita en el pizarrón que dice «acciones».\\
192 D: sí muy bien\\
193 AH: acción\\
194 D: muy bien\\
195 \# Condición: escritura.\\
196 AH: acción\\
197 D: saltar\\
198 AH: saltar, sa \# escritura: c, en la columna de acciones.\\
199 D: /s/ ¿cómo es? /s/\\
200 AH: sa, sal \# escritura: «SAL».\\
201 D: pero hay que borrar la primera ¿no?\\
202 AH: ah cierto \# borra la letra c.\\
203 D: saltar\\
204 AH: sal ¿e?\\
205 D: /t/, /t/\\
206 AH: sal, tar, saltar \# escritura: «SALTAr»--- saltar\\
207 D: muy bien muy bien, gracias, y por ejemplo ¿caminar?\\
208 AH: caminar\\
209 D: ¿esto qué es? \# con los dedos índice y medio, de forma alternada los deplaza por la mesa, simulando la acción de «caminar».\\
210 AH: acción\\
211 D: sí muy bien\\
212 AH: acción, caminar \# escritura: c, en la columna de cosas.\\
213 D: ah pero este es el de la acciones \# señala la columna de acciones.\\
214 AH: ah sí, ca, mi, nar \# escritura: «camiNAr» en la columna de acciones.\\
215 D: a ver por ejemplo, ese espejo ¿es una acción o es una cosa?\\
216 AH: cosa, cosa\\
217 D: muy bien\\
218 AH: espejo, es, pe, jo \# escritura: «ESPego» en la columna de cosas.\\
219 D: ¿cómo quedamos que sonaba esta? \# señala la letra g.\\
220 AH: ah \# comienza a borrar la letra g.\\
221 D: ¿esta cómo suena? \# señala la letra g--- antes de que la borre\\
222 AH: eh, ga ga\\
223 D: y si la dejamos aquí diría espego ¿verdad?\\
224 AH: ah sí cierto\\
225 D: ¿por cuál la tenemos que cambiar?\\
226 AH: \# sustituye la letra g por la j.\\
227 D: muy bien, ahora por ejemplo, un bote de la crema, como este \# le muestra un bote de plástico--- ¿qué es, es una cosa o es una acción?\\
228 AH: cosa, cosa\\
229 D: muy bien, bote \# le da el marcador azul.\\
230 AH: bo, bo \# escritura: «BOTe»\\
231 D: muy bien y por ejemplo, el bote de basura ¿qué es?\\
232 AH: acción\\
233 D: ¿si segura?\\
234 AH: acción\\
235 D: ¿por qué?\\
236 AH: eh, acción, bote \# señala el bote de basura.\\
237 D: pero acuérdese que las acciones se llevan a cabo en el tiempo, generalmente alguien las realiza\\
238 AH: cosas, cosas, cosas\\
239 D: si nadie le hace nada al bote, si está nada más ahí, es una cosa\\
240 AH: ah sí\\
241 D: okey \# le aproxima el marcador azul.\\
242 AH: el bote ya \# señala la palabra «BOTe» que escribió recientemente.\\
243 \# Condición: conversación.\\
244 D: bueno está bien, ese es otro bote\\
245 AH: sí \# se ríe.\\
246 D: a ver ¿qué tal la computadora?\\
247 AH: computadora, acción\\
248 D: ¿por qué?\\
249 AH: acción\\
250 D: ¿por qué?\\
251 AH: computadora, computadora\\
252 D: no, es una cosa\\
253 AH: ah\\
254 D: si nada más existe así, es una cosa\\
255 AH: ah bueno\\
256 D: ¿qué tal el reloj?\\
257 AH: el reloj, acción\\
258 D: no\\
259 AH: ¿no? cosa\\
260 D: es una cosa, si sólo está ahí, sólo es una cosa\\
261 AH: sí\\
262 D: y por ejemplo ¿escribir? si yo agarro una hoja y hago esto \# comienza a escribir en una hoja blanca.\\
263 AH: escribir \# susurrando.\\
264 D: ¿esto qué es, es una acción o es una cosa?\\
265 AH: acción, no, cosa cosa\\
266 D: no\\
267 AH: acción\\
268 D: es una acción\\
269 AH: ajá\\
270 D: ¿le cuesta trabajo?\\
271 AH: no \# se ríe.\\
272 D: ¿segura?\\
273 AH: sí\\
274 D: bueno a ver, escibir sí es una acción \# le da el marcador azul.\\
275 \# Condición: escritura.\\
276 AH: es, cri, ah no \# escritura: «ESL».\\
277 D: escri\\
278 AH: escri\\
279 D: cri\\
280 AH: bi ¿bi?\\
281 D: sí pero nos faltan muchas cosas antes, escri, cri /k/\\
282 AH: ¿i?\\
283 D: /k/ ¿sabe qué sonido es este? /k/\\
284 AH: es, cri\\
285 D: como en casa\\
286 AH: ah, es, cri \# escritura: «ESci».\\
287 D: ah ah, aquí nos falta algo \# señala lo que Ah acaba de escribir.\\
288 AH: es \# borra la letra i.\\
289 D: cri\\
290 AH: cri, escri, escribir \# escritura: «EScridir»--- escibir\\
291 D: a ver lea lo que dice\\
292 AH: escri, bi \# borra la letra r--- escribir, bir, escibir, escribir \# escritura: agrega la letra N al final resultando en «EScridiN»--- no, escribir\\
293 D: ¿esta qué letra es? \# señala la letra N.\\
294 AH: ene\\
295 D: ¿y cuál es la última que queremos?\\
296 AH: ¿erre?\\
297 D: ajá\\
298 AH: \# escritura: sustituye la letra N por la r.\\
299 D: ajá, hay otro error ¿cuál es?\\
300 AH: es, cri, bir, ¿sí? \# no se percata del error.\\
301 D: no\\
302 AH: es, cri \# brevemente posiciona el marcador azul sobre la palma de su mano izquierda--- ah, be\\
303 D: ajá muy bien\\
304 AH: be, be \# escritura: sustituye la letra d por la B, resultando en «EScriBir».\\
305 D: muy bien muy bien\\
306 \\
307 \# Segmento:  reconocimiento de imágenes que representan acciones y cosas.\\
308 \# Material: imágenes mostradas en un teléfono celular.\\
309 \# Condición: conversación.\\
310 D: ¿esto qué es, acción o cosa? \# imagen de un pantalón.\\
311 AH: un pantalón\\
312 D: ¿acción o cosa?\\
313 AH: cosa cosa\\
314 D: muy bien muy bien, ¿qué tal esto, acción o cosa? \# imágen de un salero.\\
315 AH: ¿este? \# señala la imagen--- este, cosa\\
316 D: ¿qué es?\\
317 AH: sal, sal\\
318 D: muy bien ¿esto qué es cosa o acción? \# imagen de un jugador de futbol a punto de patear un balón.\\
319 AH: este, futbol, cosa\\
320 D: ¿cosa?\\
321 AH: no, este ... \# niega con la cabeza.\\
322 D: ¿qué está haciendo?\\
323 AH: con el futbol\\
324 D: a ver ¿y esta, qué observa? \# imagen de un niño escribiendo.\\
325 AH: el niñito, la niña el niño\\
326 D: ¿qué hace?\\
327 AH: este, estaba, la tarea, la tarea\\
328 D: ¿cómo se le dice a lo que está haciendo?\\
329 AH: trabajando, trabajando\\
330 D: está bien ¿eso qué es?\\
331 AH: el niño estaba trabajando\\
332 D: sí, pero ¿qué es, cosa o acción?\\
333 AH: acción, acción\\
334 D: muy bien, también podemos decir que está leyendo y eso también es una acción\\
335 AH: sí cierto\\
336 D: \# le muestra la siguiente imagen una bolsa de mano.\\
337 AH: una bolsa\\
338 D: muy bien ¿y qué es?\\
339 AH: acción\\
340 D: ¿por qué?\\
341 AH: una bolsa, este, cosa cosa\\
342 D: sí es una cosa, acuérdese si nada más está ahí es una cosa\\
343 AH: cosa, cosa\\
344 D: ¿qué es? \# imagen de una olla de cocina.\\
345 AH: cacerolas, una cacerola\\
346 D: muy bien ¿qué es?\\
347 AH: cosa, cosa\\
348 D: muy bien muy bien, ¿qué ve? \# imagen de una mujer cocinando.\\
349 AH: una cacerola\\
350 D: ¿qué más?\\
351 AH: una cacerola y fuego, fuegos\\
352 D: ¿qué más?\\
353 AH: este, jitomates\\
354 D: ¿hay alguna persona?\\
355 AH: la señora\\
356 D: ¿y qué está haciendo?\\
357 AH: este, en el /.escola en la esc en es, es./ en la estufa, en la estufa\\
358 D: ¿pero qué hace?\\
359 AH: en la estufa estaba, estaba, estufa estaba, la sal con sal, la sal\\
360 D: ajá ¿a todo eso cómo se le llama, usted lo hace en su casa, no?\\
361 AH: sí\\
362 D: ¿cómo se le llama a eso, yo estoy?\\
363 AH: yo estoy, yo estoy /. /s/ desullando estoy desua desullando desollunando./ yo comí comieron \# se ríe.\\
364 D: pero todavía no está comiendo ¿o sí? todavía no, esto viene antes \# señala la imagen.\\
365 AH: ajá\\
366 D: antes de comerlo ¿usted qué hace?\\
367 AH: ...\\
368 D: haga de cuenta que este es mi sartén, y tengo por aquí salecita, le echo las papitas que piqué \# toma una hoja blanca, con la mano izquierda finge colocar cosas encima y manipularlas--- ¿todo eso cómo se llama, estoy /k/?\\
369 AH: ...\\
370 D: estoy co\\
371 AH: comiendo\\
372 D: no\\
373 AH: estoy\\
374 D: antes, antes de comer ¿qué se le hace a la comida, usted la co?\\
375 AH: ...\\
376 D: la cocina\\
377 AH: la si, la /.cosida, cosida cosida, cosida./\\
378 D: ajá pero usted la cocina\\
379 AH: sí\\
380 D: okey ¿qué es eso?\\
381 AH: cocinando\\
382 D: sí muy bien\\
383 AH: cocinando\\
384 D: muy bien ¿eso qué es?\\
385 AH: este, cosas\\
386 D: ¿cosas?\\
387 AH: cosas, cosas\\
388 D: cocinar no es una cosa, es una acción\\
389 AH: sí\\
390 D: pero usted sí cocina con cosas ¿qué usa para cocinar?\\
391 AH: aceite\\
392 D: muy bien, eso es una cosa ¿qué más usa para cocinar?\\
393 AH: este, chiles\\
394 D: eso es una cosa ¿qué más?\\
395 AH: este, huevos\\
396 D: eso también es una cosa, pero cocinar, todo eso es una\\
397 AH: acción \# interrumpe a D.\\
398 D: ajá\\
399 AH: acción sí cierto\\
400 D: muy bien muy bien ¿esto qué es? \# imagen de una cebolla siendo cortada con un cuchillo.\\
401 AH: cebolla\\
402 D: bueno sí pero lo que se está haciendo\\
403 AH: cebolla el /.cucha el cucha./ el cuchillo, el cuchillo, el cuchillo\\
404 D: sí ¿pero cómo se le llama a eso, que la persona está?\\
405 AH: ...\\
406 D: ¿cómo se le llama a esto? \# toma el marcador azul y finge usarlo como cuchillo.\\
407 AH: estaba, estaba, estaba, ay \# accidentalmente toca el teléfono celular y quita la imagen--- ay hombre \# se rie.\\
408 D: no se preocupe \# vuelve a abrir la imagen.\\
409 AH: estaba, estaba\\
410 D: estaba ¿pi?\\
411 AH: picando, picando el cebo la cebolla\\
412 D: ajá\\
413 AH: picando cebolla\\
414 D: ¿picar qué es?\\
415 AH: este, picar, acción acción\\
416 D: muy bien, ¿cómo se le llama a esto? \# imagen de un sartén con aceite donde se fríe un alimento.\\
417 AH: ...\\
418 D: también se hace en la cocina, cierto?\\
419 AH: (?)\\
420 D: por ejemplo, cuando usted tiene un sartén y le pone mucho aceite para meter su comida ¿cómo se le llama a eso?\\
421 AH: ...\\
422 D: ¿al meter comida en el aceite?\\
423 AH: frío, frío\\
424 D: freír\\
425 AH: freír freír\\
426 D: muy muy bien muy bien\\
427 AH: freír freír\\
428 D: ¿eso qué es?\\
429 AH: este, freír, cosa\\
430 D: no\\
431 AH: acción acción\\
432 D: a ver, acuérdese, lo que usted fríe, lo que mete al aceite son cosas, pero todo eso es freír\\
433 AH: ah cierto\\
434 D: a ver ¿qué observa? \# imagen de unas manos que sostienen una zanahorias bajo el chorro de agua.\\
435 AH: lavando\\
436 D: sí ¿qué está lavando?\\
437 AH: este, lavando las, lavando la los, las, lavando los, lavando un zanahoria, las zanahorias\\
438 D: muy bien muy bien\\
439 AH: zanahorias\\
440 D: de momento sólo céntrese en lavar\\
441 AH: sí sí\\
442 D: ¿lavar qué es?\\
443 AH: lavar, las zanahorias, a ver, este un cosa\\
444 D: las zanahorias son cosa, pero lavar no es una cosa\\
445 AH: cosa, las cosas, una cosa\\
446 D: ¿pero qué es la cosa aquí, cuál es la cosa?\\
447 AH: acción acción \\
448 D: sí sí es una acción lavar, aquí sí hay cosas ¿cuáles son las cosas?\\
449 AH: esto\\
450 D: ¿cómo se llama?\\
451 AH: las zanahorias\\
452 D: sí, eso es una cosa, a ver póngamelo aquí en cosas \# pone sobre la mesa el pizarrón blanco usado previamente.\\
453 \# Condición: escritura.\\
454 AH: cosas \# escritura: «co» en la columna de cosas.\\
455 D: ah, zanahorias ¿no?\\
456 AH: ah zanahorias \# borra lo que acaba de escribir--- /s/ sa \# voltea a ver a D con duda.\\
457 D: sa /s/\\
458 AH: sa ¿a? \# escritura: A.\\
459 D: antes de la /a/ nos falta algo\\
460 AH: se, sa, na o, sana, o, oria, zanahoria \# escritura: agrega la letra S antes de la A y concluye la escritura resultando en «SANAoria».\\
461 \# Condición: conversación.\\
462 D: muy bien ¿y cuál es la acción?\\
463 AH: cosas\\
464 D: no no, aquí \# señala la imagen de las zanahorias siendo lavadas--- ¿cuál es la acción, qué se está haciendo con las zanahorias?\\
465 AH: zanahoria, lavando la zanahoria\\
466 D: ¿cuál es la acción?\\
467 AH: acción aquí \# señala la columna de acciones en el pizarrón.\\
468 D: ¿pero cuál es? usted ya me lo dijo\\
469 AH: zanahoria\\
470 D: ¿cuál es la acción? ya me lo dijo\\
471 AH: zanaho\\
472 D: y no es zanahoria\\
473 AH: ¿no?\\
474 D: zanahoria no es ¿cuál es la acción? la acción\\
475 AH: agua, el agua\\
476 D: no eso también es una cosa\\
477 AH: \# se ríe.\\
478 D: ya me lo dijo ¿cuál era?\\
479 AH: zanahoria\\
480 D: no, no es ni zanahoria ni agua ¿qué es?\\
481 AH: acción\\
482 D: ajá ¿cuál es la acción? \\
483 AH: lavando\\
484 D: sí muy bien, esa es la acción\\
485 AH: lavando lavando\\
486 D: a ver escríba lavar \# señala la columna de acciones en el pizarrín blanco.\\
487 \# Condición: escritura.\\
488 AH: lavando, la, van, lavan, do \# escritura: «LaVANdo».\\
489 D: okey muy bien\\
