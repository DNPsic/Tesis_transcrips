\section{Sesión del de del 2024}
\noindent
Datos de la sesión:
ID:
Participantes:
AH:
D:
DT:
AC:

\noindent
001 # Segmento: inicio de sesión.\\
002 # Condición: conversación.\\
003 D: cuénteme qué hizo en las mañanas, en las tardes y en las noches\\
004 AH: yo desayuné\\
005 D: ¿qué más hizo qué desayunó?\\
006 AH: sopa de verduras\\
007 D: ¿qué más?\\
008 AH: y atole\\
009 D: ¿eso qué día fue?\\
010 AH: siempre siempre\\
011 D: ¿todos los días desayuna supa de verduras?\\
012 AH: sí sí\\
013 D: muy bien qué saludable ¿qué más?\\
014 AH: y pescado\\
015 D: ¿qué más hizo en la semana?\\
016 AH: este la tarea este, trapié, la tarea\\
017 D: ¿qué más?\\
018 AH: y duerme y duerme\\
019 D: ay qué bonito es dormir\\
020 AH: sí verdad\\
021 D: a ver cuénteme qué hizo por las tardes\\
022 AH: este duerme y duerme, duerme y duerme # asiente.\\
023 D: ¿qué más qué otra cosa?\\
024 AH: este /s/ duerme y duerme con mis nietos con mis nietos y con mi su esposa de D_\\
025 D: ¿D_ quién es?\\
026 AH: este mi nieto\\
027 D: ah muy bien ¿lo fueron a visitar?\\
028 AH: aquí está en el departamento ¿vea?(¿verdad?) # voltea a ver a AC.\\
029 D: ah viven con usted\\
030 AH: no el (?) a la mitad # voltea a ver a AC.\\
031 AC: viven en el otro ahí somos vecinos\\
032 AH: sí\\
033 D: viven al lado\\
034 AC: llegaron a rentar ahí\\
035 D: ah muy bien muy bien\\
036 AC: A_ mi hijo les consiguió ahí el departamento y pues ya se cambiaron de donde estaban\\
037 AH: sí\\
038 D: es el hijo de A_ entonces\\
039 AH: sí\\
040 AC: el hijo de A_\\
041 D: ¿hizo alguna otra cosa en las noches por ejemplo?\\
042 AH: café con leche y pan tostado\\
043 D: muy bien muy bien ¿algo más?\\
044 AH: el seguro también en la man ayer me fui al seguro\\
045 D: ah muy bien ¿escuchó que usted dijo me fui?\\
046 DT: me fui al seguro me fui al seguro # asiente.\\
047 D: muy bien ahí usó un verbo\\
048 AH: ¿sí? ah sí mira\\
049 D: ¿alguna otra cosa?\\
050 AH: este ya no no sé\\
051 D: muy bien\\
052 AH: el tianguis también el miércoles\\
053 D: usted fue al tianguis\\
054 AH: no porque ya no ya luego está muy feo muy feo\\
055 D: ¿a porque es peligroso?\\
056 AH: pus(pues) sí\\
057 # Segmento: actividad con ilustraciones de objetos.\\
058 # Material: tarjetas de imágenes y las preguntas en el pizarrón ¿qué es? ¿qué se hace?.\\
059 ...\\
060 DT: con este de arriba ya no vamos a trabajar, ya no es la cosa, ya lo tachamos, ahora con ¿qué se hace? ¿lo puede leer otra vez?\\
061 AH: ¿qué se hace?\\
062 DT: yo le voy a poner un ejemplo\\
063 # se le muestra una ilustración con una taza de café\\
064 DT: ¿qué cosa es?\\
065 AH: café\\
066 DT: ¿qué se hace?\\
067 AH: leche con café\\
068 DT: ahora para contestar, este es un ejemplo, ¿qué se ahce con el café? me lo tomo ¿ya vio?\\
069 AH: sí sí\\
070 DT: entonces acá arriba es café y con esta ¿la puede leer otra vez?\\
071 AH: qué hace\\
072 DT: me lo tomo\\
073 AH: me lo tomo me lo tomo me lo tomo\\
074 DT: exacto este fue un ejemplo, entonces vamos a empezar\\
075 AH: sí\\
076 DT: a ver este qué me había dicho\\
077 # se le muestra una ilustración de una estufa\\
078 AH: la estufa\\
079 DT: ¿y qué se hace?\\
080 AH: en la estufa se hace el ca estaba el café la estufa, en la estufa\\
081 DT: sí muy bien ¿y qué se hace con la estufa?\\
082 AH: este esperame, la lumbre la lumbre\\
083 DT: ¿y qué hace usted ahí en la estufa?\\
084 AH: en la lumbre, el café con leche, café el café\\
085 DT: ¿qué más?\\
086 AH: el café,este atole\\
087 DT: ¿y eso cómo se dice? le voy a ayudar, cuando usted pone cosas aquí es cocinar\\
088 AH: cocinar ah pues sí cierto\\
089 DT: entonce vamoa a intentar decirlo todo completo ¿qué se hace en la estufa?\\
090 AH: cocinar\\
091 DT: o puede decir yo cocino\\
092 AH: yo deci(?) yo\\
093 DT: yo cocino\\
094 AH: yo cociné\\
095 DT: a ver señálese así\\
096 AH: yo cociné\\
097 DT: muy bien a ver otra vez, en la estufa\\
098 AH: yo cociné \\
099 DT: muy bien\\
100 ...\\
101 DT: ahora vamos con este ¿qué dijimos que era?\\
102 # se le muestra una ilustración de un teléfono fijo, no celular\\
103 AH: el teléfono\\
104 DT: ¿y qué se hace?\\
105 AH: ¿qué hace? # el teléfono señala la tarjeta con la imagen del teléfono. /rr/ # intenta imitar el sonido de un teléfono.\\
106 DT: ¿y eso cómo se llama? sí está bien\\
107 AH: el teléfono estaba bien /. rue ru ./ # se ríe.\\
108 DT: muy bien pero tiene otra forma de decirlo, hace\\
109 AH: hace ruido\\
110 DT: eso muy bien\\
111 AH: hace ruido \\
112 DT: ¿y usted qué hace con el teléfono? acá me dijo cocino\\
113 # se le muestra la ilustración de la estufa\\
114 DT: y aquí con este\\
115 # se le señala la ilustración del teléfono\\
116 AH: el teléfono\\
117 DT: ¿y qué hace con el teléfono\\
118 AH: este con mis hijos, el teléfono con mis hijos\\
119 DT: ¿pero qué hace con sus hijos?\\
120 AH: con mis hijos en el refri(refigerador) estaba en # comeinza a reirse y se cubre la boca con la mano. el teléfono\\
121 DT: muy bien, le voy a ayudar otra vez, acá en la estufa cocinamos\\
122 # se le señala la imagen de la estufa\\
123 AH: cocinamos\\
124 DT: y por el teléfono o en el teléfono\\
125 # se le hace la seña con la mano de «llamar por teléfono»\\
126 DT: ¿cómo se dice esto?\\
127 AH: ...\\
128 DT: hablamos\\
129 AH: hablando hablamos\\
130 DT: o llamamos\\
131 AH: hablamos ah sí cierto hablamos # no dijo «llamamos» repitió varias veces «hablamos».\\
132 DT: a ver con esta qué hacemos\\
133 # se le señala la ilustración de la estufa\\
134 AH: estufa\\
135 DT: ¿qué hacemos con la estufa?\\
136 AH: la estufa\\
137 DT: ¿qué hacemos?\\
138 AH: eh la estufa\\
139 DT: ¿qué habíamos dicho que hacía?\\
140 AH: estaba en la estufa ...\\
141 DT: cocinamos\\
142 AH: cocinamos\\
143 AT: ¿y aquí?\\
144 # se le señala la ilustración del teléfono\\
145 AH: el teléfono está bien sordo # se ríe y voltea a ver a AC.\\
146 DT: muy bien vamos a ir al siguiente, a ver por ejemplo este qué dijimos\\
147 # se le muestra la ilustración de una canasta con diferentes tipos de pan\\
148 AH: las galletas\\
149 DT: ¿qué se hace con las galletas?\\
150 AH: las galletas ((estaban)) bien sabrosas\\
151 DT: ¿qué hace usted con las galletas?\\
152 ... # interrupción por llamada telefónica\\
153 # Segmento: continuación posterior a la interrupción.\\
154 AH: están bien sabrosas\\
155 DT: ¿y qué hace con las galletas usted?\\
156 AH: a ver, eh pérame(espérame) ay, sí, ay pérame(espérame) ¿qué es? las galletas estaban bien sabrosas\\
157 DT: bien, vamos a pasar a la siguiente, ¿aquí qué dijimos que era?\\
158 # se le muestra una ilustración de una playa\\
159 AH: en pérame(espérame) ... a ver\\
160 DT: ya nos había dicho hace rato\\
161 AH: en ... no no me acuerdo\\
162 DT: empieza con esta letra\\
163 # se da la vuelta a la tarjeta donde esta escrita la letra «M»\\
164 AH: Mar mar\\
165 D: muy bien\\
166 DT: ¿y aquí qué hace?\\
167 AH: el mar estaba bien fea bien feo # se rie\\
168 DT: ¿no le gusta ir al mar?\\
169 DH: no no\\
170 DT: ¿no le gusta?\\
171 D: ¿por qué?\\
172 AH: # voltea a ver a D y asiente sin responder\\
173 DT: ¿y qué puede hacer en el mar?\\
174 AH: nadar, nadando nadando\\
175 D: muy bien\\
176 DT: a ver dígame yo nado\\
177 AH: yo nadé nad(?) ya nadé yo nadé\\
178 DT: muy bien muy bien ahí sí nos salió el verbo, a ver y esto qué era\\
179 # se le muestra una ilustración de una taza vacía\\
180 AH: la ...\\
181 D: ¿con qué letra empieza, se acuerda?\\
182 AH: ay pérame(espérame) no\\
183 DT: vamos a hacer trampa\\
184 # se da la vuelta a la tarjeta donde esta escrito «TA»\\
185 AH: la la talla la taj ay # comeinza a reirse y se cubre la boca.\\
186 DT: la talla\\
187 D: más o menos pero no\\
188 AH: la tas la cosa la cosa # continua riéndose.\\
189 DT: sí es una cosa\\
190 D: sí muy bien\\
191 AH: ¿qué? la\\
192 D: ¿cómo se llama?\\
193 AH: la # deja de reirse. ay mucho tristeza mucha tristeza # se lleva la mano izquierda a la garganta. mucha tristeza mi lengua\\
194 DT: ¿por eso le está costando trabajo?\\
195 AH: sí\\
196 ... # pausa para descansar de la actividad.\\
197 # Segmento: continuación posterior a la pausa.\\
198 # se le muestra la ilustración de la taza vacía\\
199 DT: la taza\\
200 AH: la taza, la taza la taza estaba bien sabrosas\\
201 DT: ¿la taza, se come?\\
202 AH: la el café el café\\
203 DT: muy bien muy bien, es como esta ¿no?\\
204 # se le muestra la ilustración de la taza con café\\
205 AH: sí ajá\\
206 DT: vamos a cambiarla\\
207 # se deja la ilustración de la taza con café y se retira la de la taza vacía\\
208 DT: aquí está el café\\
209 AH: el café\\
210 DT: ¿y qué hace con el café?\\
211 AH: este, bien sabroso\\
212 D: ¿pero qué hace?\\
213 DT: ¿qué hace? ya estamos acá en esta pregunta\\
214 # se le señala la pregunta «¿qué se hace?» escrita en el pizarrón\\
215 DT: ¿qué se hace con el café?\\
216 AH: se hace café, el café ((se hace)) el café\\
217 DT: se toma # hace el gesto con la mano izquierda de «tomar».\\
218 AH: se tomó se tomá el café\\
219 DT: yo me tomé el café\\
220 AH: yo tomé café\\
221 DT: eso muy bien\\
222 ...\\
223 DT: a ver vamos a ver, por ejemplo con este\\
224 # se le muestra una ilustración de un cuchillo de cocina\\
225 AH: el cuchillo\\
226 DT: ¿qué se hace con un cuchillo?\\
227 AH: el cuchillo el cuchillo está bien rasposo # comienza a reírse.\\
228 DT: vamos a borrar esto y dejar el hace\\
229 # se borra lo escrito en el pizarrón blanco dejando únicamente la palabra «hace»\\
230 AH: hace\\
231 DT: ¿qué hacemos con el cuchillo?\\
232 AH: el cuchillo estaba bien filoso\\
233 D: muy bien\\
234 AH: bien filoso\\
235 DT: ¿y usted qué hace con el cuchillo?\\
236 AH: el cuchillo está\\
237 DT: ¿para qué lo usa?\\
238 AH: /mr/ # hace un sonido suave de /r/ no se entiende si es una palabra. ia(mira) aquí # muestra el dedo medio de la mano izquierda y lo señala con el índice derecho.\\
239 D: ajá muy bien\\
240 DT: ¿qué hizo usted?\\
241 AH: se /m/ estaba bien /s/ # sonido de /s/ aspirada. ay, bien filoso\\
242 DT: ¿se cortó?\\
243 AH: sí\\
244 DT: a ver dígalo\\
245 AH: me corté me corté\\
246 DT: eso, con el cuchillo\\
247 AH: cuchillo, me corté cuchillo\\
248 DT: ahí está muy bien, vamos a intentar unos más, a ver ¿con este?\\
249 # se le muestra una ilustración de un tenedor con «TE» escrito al reverso\\
250 AH: no me acuerdo\\
251 DT: ¿cómo se llama? vamos a usar la tramapa\\
252 # se le da la vuelta a la tarjeta para que pueda leer la pista\\
253 DT: ¿cómo se llama?\\
254 AH: ¿qué es, qué es hijo?\\
255 DT: ¿cómo lo usa?\\
256 AH: ...\\
257 DT: ¿cómo lo usa este?\\
258 AH: este qué es\\
259 DT: ¿se come?\\
260 AH: no no hombre no\\
261 DT: pero sí se usa para\\
262 AH: sí sí\\
263 DT: ¿para qué se usa?\\
264 AH: este así # se lleva la mano derecha juntando las puntas de los dedos a la boca en repetidas cosasiones haciendo el gesto de «comer» o «llevarse un bocado a la boca».\\
265 DT: ¿cómo se llama eso?\\
266 AH: no me acuerdo # mueve la mano izquierda cerca de la boca juntando las puntas de los dedos.\\
267 DT: comer\\
268 AH: yo comí este qué # niega levemente con la cabeza y se rie.\\
269 DT: este no se come\\
270 AH: ajá\\
271 DT: pero lo usamos para comer\\
272 AH: ah sí\\
273 DT: ¿no se acuerda cómo se llama?\\
274 AH: no\\
275 DT: a ver\\
276 # se vuelve a mostrar el «TE» escrito al reverso de la tarjeta\\
277 DT: te\\
278 AH: te\\
279 DT: ten # haciendo énfasis en el sonido de la /n/\\
280 AH: tendedor ¿cómo?\\
281 D: muy bien\\
282 DT: más o menos, ten\\
283 AH: ten\\
284 DT: tene\\
285 AH: tenedor tenedor\\
286 DT: eso\\
287 AH: ah tenedor tenedor \\
288 DT: a ver ¿para qué se usa el tenedor?\\
289 AH: el tenedor estaba hace # se rie. el tenedor estaba /fs/ # se lleva la mano izquierda enfrente de la boca y comienza a reírse.\\
290 DT: a ver vamos a decir comió con el tenedor\\
291 AH: el tenedor comió comí con el tenedor\\
292 ... # por conversación de D y DT con AC.\\
293 DT: a ver ahora este ¿qué dijimos?\\
294 # se le muestra una ilustración de un vagon del metro de la ciudad de México.\\
295 AH: el metro\\
296 DT: ¿y qué hace?\\
297 AH: /pi/ # alargando el último sonido.\\
298 DT: sí ¿y eso cómo se llama?\\
299 AH: este en el metro estaba ((catitlan)) acatitla # se rie.\\
300 DT: en acatitla\\
301 AH: sí\\
302 DT: bueno vamos a decir que ¿hace ruido?\\
303 AH: sí mucho ruido\\
304 DT: ¿va lento o rápido?\\
305 AH: pus(pues) len bien /f/ feo bien \\
306 DT: ¿bien?\\
307 AH: bien\\
308 DT: ¿se mueve?\\
309 AH: sí hombre\\
310 DT: a ver vamos a decir se mueve, usted diga, todo completo, el metro\\
311 AH: el metro estaba bien estaba /fs/ # con las dos manos hace puños y los sube y baja en un par de ocasiones. ¿cómo? # comienza a reírse.\\
312 DT: ¿lento?\\
313 AH: no bien\\
314 DT: ¿bien qué?\\
315 AH: bien ... no\\
316 DT: ¿rápido?\\
317 AH: rápio bien rápido rápido rápido\\
318 DT: ¿se mueve rápido?\\
319 AH: sí bien rápido\\
320 DT: a ver diga así, se mueve rápido\\
321 AH: ¿cómo?\\
322 DT: se mueve\\
323 AH: se bue\\
324 DT: mueve\\
325 AH: se mueve el metro\\
326 DT: ajá\\
327 AH: el metro est el metro estaba bien bien \\
328 DT: rápido\\
329 AH: bien\\
330 DT: rápido\\
331 AH: bien rápido\\
332 DT: a ver este\\
333 # se le muestra una ilustración de un tambor con el escrito «TAM» al reverso\\
334 AH: a ver, /m/ ay no me acuerdo\\
335 DT: a ver voltéelo atrás qué dice\\
336 AH: tambor\\
337 DT: ¿qué hace?\\
338 AH: estaba /psh/ # con ambas manos con puños cerrados sube y baja de forma alternada simulando tocar un tambor. \\
339 DT: ¿eso cómo se dice?\\
340 AH: el tambor estaba bien /s/ # repite el gesto con las manos de tocar un tambor.\\
341 DT: ¿usted qué hace con el tambor?\\
342 AH: # repite el gesto de tocar el tambor, esta vez la distancia vertical al alternar las manos es mucho más pronunciada y no agrega sonido.\\
343 DT: ¿eso cómo se llama, cómo se llama hacerle así? # imita el gesto con las manos.\\
344 AH: ay\\
345 DT: así como si yo le estuviera haciendo así ¿no? # toma los marcadores para pizarrón y simula tocar un tambor. ¿cómo se dice esto?\\
346 AH: ¿mucho ruido?\\
347 DT: muy bien\\
348 AH: mucho ruido\\
349 DT: tocar el tambor\\
350 AH: tocar\\
351 DT: a ver diga usted\\
352 AH: tocar el ba el tocador # comienza a reir. estaba el ¿qué es?\\
353 DT: el tambor\\
354 AH: el tambor\\
355 DT: ¿qué hace?\\
356 AH: estaba /m/ # vuelve a hacer el gesto con ambas manos de tocar un tambor.\\
357 DT: tocando el tambor\\
358 AH: tocar\\
359 DT: y a ver ya el último\\
360 # se le muestra una ilustración de un plato vacío\\
361 AH: plato\\
362 DT: ¿y qué se puede hacer con un plato?\\
363 AH: la sopa la sopa\\
364 DT: ¿este se come?\\
365 AH: no hombre\\
366 DT: ¿se usa para la comida?\\
367 AH: sí\\
368 DT: ¿cómo se usa?\\
369 AH: este # señala la ilustración. un plato y con las cucharas y sopa\\
370 DT: entonces servimos la comida\\
371 AH: sí sí\\
372 DT: puede ser\\
373 AH: sí\\
374 DT: a ver dígalo usted\\
375 AH: /. sevimos ./ servimos servimos\\
376 DT: o yo sirvo\\
377 AH: yo sirvo el sopa\\
378 DT: muy bien\\
379 # Segmento: clasificación de palabras en categorías de acción (verbos) o cosas.\\
380 # Material: pizarrón blanco con un tabla de dos columnas, una  que dice «cosa» y otra «acción».\\
381 DT: esto ya lo hemos hecho ¿me ayuda a leer?\\
382 # se le señalan las columnas en el pizarrón\\
383 AH: cosa acción\\
384 DT: ¿se acuerda que los estábamos separando? que si esto es una cosa que si esto una acción\\
385 AH: sí sí cierto\\
386 DT: por ejemplo vamos a usar su tarea, ¿el lápiz dónde iría, acá o acá? lápiz\\
387 AH: a ver # señala la columna de acción.\\
388 DT: ¿y escribir?\\
389 AH: aquí # señala la columna de cosa.\\
390 DT: en realidad van al revés pero vamos a empezar a trabajar\\
391 AH: ah sí sí\\
392 DT: por ejemplo, ¿esto qué es?\\
393 # se le muestra una ilustración de una taza de café\\
394 AH: café\\
395 DT: ¿es cosa o es acción?\\
396 AH: este acción # señala la columna de acción.\\
397 DT: no, va acá # coloca la ilustración en la columna de cosa.\\
398 AH: ah sí cierto\\
399 DT: es una cosa, pero a ver ¿qué hace con el café?\\
400 AH: este el café\\
401 DT: ¿pero qué hace usted? así si le doy su tacita de café # junta ambas manos acercándolas a AH simulando el gesto de «dar».\\
402 AH: este bebí bebí\\
403 DT: eso muy bien\\
404 D: muy bien\\
405 AH: bebí\\
406 DT: a ver escriba aquí bebí # señala la columna de acción.\\
407 AH: bebí # toma el marcador de pizarrón.\\
408 # Condición: escritura.\\
409 AH: be # escribe «Be». be # escribe la letra B. bí # escribe la letra i, resultando en «BeBi».\\
410 # Condición: conversación.\\
411 DT: muy bien, esta es la acción # señala la palabra «bebí» que AH acaba de escribir. a ver hágale así de que tomamos # hace el gesto con la mano izquierda de «beber».\\
412 AH: comí es # señala la ilustración de la taza de café. este el café, café /. bibi ./ bebí café\\
413 DT: eso ¿ya vio? el café es la cosa es diferente\\
414 AH: sí cierto\\
415 DT: a ver vamos a hacerlo otra vez\\
416 AH: órale\\
417 DT: estuvo bien eh, así como lo hizo está bien\\
418 AH: sí\\
419 DT: ¿qué dijimos que es?\\
420 AH: café\\
421 DT: ¿es cosa o es acción?\\
422 AH: cosa\\
423 DT: a ver acomódelo\\
424 # se le entrega la ilustración con la taza de café\\
425 AH: # coloca la ilustración en la columna de cosa.\\
426 DT: muy bien, y ¿qué hace conel café? # le entrega el marcador para pizarrón.\\
427 AH: cosa\\
428 DT: a ver est ya está acá en cosa # acomoda la ilustración. ¿qué hace con el café usted?\\
429 AH: este café\\
430 D: ¿pero qué quedamos que hace con el café?\\
431 DT: ¿qué hace?\\
432 AH: bebí\\
433 DT: muy bien\\
434 AH: bebí\\
435 # Condición: escritura.\\
436 AH: be # escribe «Be». bí # escribe «Bi» resultando en «BeBi».\\
437 # Condición: conversación.\\
438 DT: por ejemplo ¿cuál estaría bien?\\
439 D: el de los tamales\\
440 DT: a ver si le gusta nuestro dibujito ¿esto qué es?\\
441 # se le muestra una ilustración de un plato con 2 tamales.\\
442 AH: tamal\\
443 DT: my bien ¿y es cosa o es acción?\\
444 AH: este cosa\\
445 DT: acomódelo # le entrega la ilustración.\\
446 AH: cosa # intenta alcanzar el marcador de pizarrón.\\
447 DT: a ver primero acomode ese aquí encima # señala la columna de cosa.\\
448 AH: así # coloca la ilustración en la columna señalada.\\
449 DT: entonce el tamal cosa ¿y qué hace con el tamal?\\
450 AH: cosa, tamal # intenta escribir en la columna de cosa, acción que no se pide en la actividad.\\
451 DT: ¿pero qué hace con el tamal, usted qué hace?\\
452 AH: cosa\\
453 DT: sí es cosa ¿pero qué hace usted? así le sirvo aquí sus tamales # le hacerca ambas manos. ¿qué hace?\\
454 AH: tamales, ay pérame(espérame)\\
455 DT: los tamales ¿los avienta?\\
456 AH: no no\\
457 DT: ¿los vende?\\
458 AH: sí sí\\
459 DT: los puede vender\\
460 D: ¿qué otra cosa puede hacer con los tamales? si lo tiene aquí y usted tiene hambre ¿qué hace?\\
461 AH: acción\\
462 D: sí muy bien ¿pero cómo se le llama? # hace el gesto con la mano derecha de llevarse algo a la boca. ¿se acuerda? \\
463 AH: ay no es que me da miedo no sé me \\
464 D: ¿por qué miedo?\\
465 AH: # mueve los ojos a la izquierda donde se encuentra AC sin que este se percate, mueve los labios pero no dice nada y sube y baja las cejas.\\
466 D: no pasa nada\\
467 DT: a ver le podemos ayudar\\
468 AH: a ver\\
469 DT: ¿es comida? # señala la ilustración de los tamales.\\
470 AH: comida\\
471 DT: los tamales son comida\\
472 AH: sí\\
473 DT: ¿qué hace con la comida? se # hace el gesto de llevarse algo a la boca con la mano derecha.\\
474 AH: se # hace el gesto de llevarse algo a la boca con la mano izquierda. be be\\
475 DT: cerca\\
476 D: más o menos pero no\\
477 DT: beber es así ¿no? tengo mi vacito o mi tacita # hace el gesto de beber con la mano derecha. pero cuando tengo mi comida y hago esto # repite el gesto de llevarse algo a la boca con la mano derecha.\\
478 AH: # frunce el seño.\\
479 DT: ¿cómo se come sus tamales? así abre la hoja\\
480 AH: sí\\
481 DT: y ¿qué, con un tenedor?\\
482 AH: ah un tenedor\\
483 DT: así lo parte # hace el gesto de llevarse algo a la boca.\\
484 AH: sí sí ¿y luego? # se empieza a reir.\\
485 DT: ¿y luego? ajá eso le toca a usted\\
486 AH: /m/\\
487 DT: nos lo co\\
488 AH: comemos\\
489 DT: eso\\
490 AH: comemos\\
491 DT: esa es la acción\\
492 AH: comemos\\
493 DT: a ver póngala aquí # señala la columna de acción.\\
494 AH: comemos\\
495 # Condición: escritura\\
496 AH: comemos, co # escribe «co». me # escribe «mi». mos # escribe «mos» resultando en «comimos».\\
497 # Condición: conversación.\\
498 DT: los tamales # señala la ilustración y después la palabra que acaba de escribir AH.\\
499 AH: comemos comimos comemos\\
500 DT: bueno está bien aquí le cambió una letra, a ver léalo\\
501 AH: comimos\\
502 DT: está bien no pasa nada, los tamales ¿qué hacemos con los tamales? # señala nuevamente la palabra escrita.\\
503 AH: comimos\\
504 DT: vamos a hacer uno último, para que no se nos sature\\
505 AH: ajá\\
506 DT: a ver este ya lo habíamos usado\\
507 # se le muestra una ilustración de un cuchillo de cocina.\\
508 AH: cuchara cuchillo\\
509 DT: ¿y es cosa o es acción?\\
510 AH: este cosa\\
511 DT: acomódelo # asiente y le entrega la ilustración.\\
512 AH: # coloca la ilustración en la columna de cosa.\\
513 DT: eso # le entrega el marcador de pizarrón. ¿y qué hace con el cuchillo?\\
514 AH: este cu ...\\
515 DT: mire to también me # le muestra una cortada pequeña que tiene el brazo. como usted\\
516 AH: ah mira sí\\
517 AC: aquí tiene otra # señala otra cortada en el brazo de DT.\\
518 DT: ah sí aquí tengo otra\\
519 AH: ah mira\\
520 DT: nada más que estas no me las hice con un cuchillo, me las hizo mi gato\\
521 AH: ay ¿a poco? # muestra sorpresa. íjole(expresión de sorpresa)\\
522 DT: pero usted con el cuchillo ¿qué se hizo?\\
523 AH: este con ira(mira) # señala su dedo medio de la mano izquierda. con /s/\\
524 DT: sí ¿cuál es la acción?\\
525 AH: este cuchillo está bien filoso\\
526 DT: ¿y qué pasó porque estaba filoso? se # con la mano derecha extendida hace el gesto de «cortar» sobre su mano izquierda.\\
527 AH: machu/k/ machuqué\\
528 DT: más o menos\\
529 D: más o menos\\
530 AH: machuqué\\
531 DT: va por ahí es otro verbo # continúa haciendo el gesto de «cortar» con la mano derecha\\
532 AH: me machuqué\\
533 DT: me cor\\
534 AH: me corté me corté\\
535 DT: eso\\
536 AH: me corté\\
537 DT: a ver escriba eso # señala la columna de acción.\\
538 # Condición: escritura.\\
539 AH: me # escribe «me». cor # escribe «cor». té # escribe «te» resultando en «me corte».\\
540 DT: ¿con el cuchillo?\\
541 AH: cuchillo # asiente pero no produce la oración completa, sólo repite la frase. con el cuchillo\\
542 DT: muy bien ahora vamos a decirlotodo completo\\
543 AH: ajá\\
544 DT: ¿qué pasó con el cuchillo?\\
545 AH: me corté # lee la palabra escrita el pizarrón.\\
546 DT: a ver dígalo todo completo\\
547 AH: me corté\\
548 D: muy bien\\
549 DT: con\\
550 AH: con el cuchillo\\
551 DT: otra vez todo completo\\
552 AH: me corté con el cuchillo\\
553 DT: eso muy bien, a ver ¿este qué habíamos dicho? # le muestra la ilustración  de los tamales nuevamente.\\
554 AH: el tamal el tamal\\
555 DT: ajá muy bien # le señala la palabra escrita en el pizarrón «comimos».\\
556 AH: comimos tamal\\
557 DT: a ver otra vez\\
558 AH: comimos tamal\\
559 DT: eso, ¿y el último? # le muestra la ilustración de la taza con café nuevamente.\\
560 AH: el café # lee la palabra «BeBi» escrita en el pizarrón. bebí café\\
561 DT: eso muy bien ¿ya vio la diferencia?\\
562 AH: ajá sí cierto\\
563 ...\\
564 DT: vamos a hacer un último ejercicio\\
565 AH: sí\\
566 DT: va a ser un poquito más difícil\\
567 AH: sí cierto\\
568 DT: pero lo hizo muy bien\\
569 ...\\
570 DT: entonces vamos a ver mire con azul las cosas y con rojo las acciones # se le muestra una hoja blanca de forma horizontal dividida en dos columnas, la izquierda dice «Acción» en color rojo y la derecha «Cosa» en azul. usted ahorita va a escojer, vamos a usar su tarea\\
571 AH: sí\\
572 DT: cepillar\\
573 AH: se\\
574 DT: cepillar, a ver yo voy a hacer la acción # con la mano derecha hace el gesto de «cepillar los dientes». ¿usted lo hace?\\
575 AH: sí sí\\
576 DT: a ver ¿cómo se lava los dientes?\\
577 AH: # lleva la mano derecha con las puntas de los dedos juntas a la comisura izquierda de sus labios, luego a la derecha y repite esta acción dos veces. sí sí sí\\
578 DT: esta es la acción\\
579 AH: acción sí\\
580 DT: cepillar\\
581 AH: sí\\
582 DT: a ver diga cepillar\\
583 AH: cepillar cepillar\\
584 DT: ¿dónde va? # señala la hoja.\\
585 AH: aquí # señala la columna que dice acción. acción\\
586 DT: muy bien # le da el marcador rojo.\\
587 AH: ¿cepillar?\\
588 DT: sí\\
589 # Condición: escritura.\\
590 AH: # escribe la letra c. ce # escribe «cep». pi # escribe la letra i segido de «ll». llar # escribe «ar» resultando en «cepillar».\\
591 DT: muy bien esa es la acción\\
592 AH: sí ajá\\
593 DT: y ¿dientes?\\
594 AH: los dientes aquí # señala la columna de cosa. cosa\\
595 DT: eso muy bien a ver\\
596 AH: cosa\\
597 DT: dientes\\
598 AH: dientes\\
599 # Condición: escritura.\\
600 AH: # escribe «diE». dien # escribe «Nte». tes # escribe la letra s resultando en «diENtes».\\
601 # Condición: conversación.\\
602 DT: muy bien a ver ahora dígalo # le señala la palabra cepillar y luego dientes.\\
603 AH: cepillar dientes\\
604 DT: ¿cómo lo puede decir? yo\\
605 AH: yo yo cepillé los dientes\\
606 DT: perfecto, a ver vamos a intetar ahora con agua\\
607 AH: /a/ agua # señala primero la columna cosa y luego la de acción, comienza a hacer un movimient con el dedo índice de la mano derecha sobre la hoja en la columna de cosa simulando la escritura de la palabra requerida, parece escribir en el aire «agi».\\
608 DT: a ver ¿es una cosa el agua?\\
609 AH: acción # señala la columna de cosa. agua agua\\
610 DT: estaba bien aquí # señala al mismo tiempo que AH la columna de cosa.\\
611 # Condición: escritura.\\
612 AH: # sin decir la palabra escribe «agui» similar a la acción anterior de escribir en el aire con el dedo índice. /a/ agua # nota el error y corrige escribiendo sobre la letra i una A, resultando en «aguA».\\
613 DT: muy bien ¿y servir?\\
614 AH: /. serville serv ./ \\
615 DT: servir\\
616 AH: servir aquí # señala la columna de acción. acción\\
617 DT: sí muy bien a ver\\
618 AH: ((vio))\\
619 DT: servir\\
620 AH: servir\\
621 # Condición: escritura.\\
622 AH: ser # escribe «CE». /s/ ser # escribe la letra N resultando en «CEN». ¿no?\\
623 DT: servir\\
624 AH: servir # escribe la letra d. ser # escribe la letra i. vir # escribe la letra N, resultando en «CENdiN».\\
625 DT: no pasa nada ¿cómo sería completo? # señala la palabra escrita por AH «CENdiN» y después\\
626 AH: servir\\
627 DT: servir # señala la palabra «aguA».\\
628 AH: agua\\
629 DT: a ver ahora ¿cómo lo diría completo?\\
630 AH: servir agua coca este cosa\\
631 DT: ah ya se le antojó la coca\\
632 AH: cosa # comienza a reirse.\\
633 DT: muy bien, y ahora vamos a ponerle uno que no ha hecho usted, por ejemplo ver\\
634 AH: ah\\
635 DT: ¿qué es cosa o acción?\\
636 AH: ver ver\\
637 DT: ver ¿cuál es?\\
638 AH: a ver # señala la columna de cosa.\\
639 DT: ¿segura?\\
640 AH: acción # continúa señalando la misma columna. ver\\
641 DT: a ver # le da el marcador rojo.\\
642 AH: ver\\
643 # Condición: escritura.\\
644 AH: ver # escribe «ER». ver ver \\
645 DT: a ver # cubre con un objeto lo que AH acaba de escribir. aquí abajito escríbalo, ver\\
646 AH: ver # escribe «Per»\\
647 DT: y a ver, tele(televisión)\\
648 AH: ¿tele?\\
649 DT: la tele\\
650 AH: coca cosa cosa cosa\\
651 DT: muy bien, sí se le antojó la coca\\
652 AH: # se rie pero no escribe.\\
653 DT: tele, la tele\\
654 AH: # escribe «te». tele # escribe «Le». vi # escribe «b». vi # escribe la letra i. /s/ # escribe «ci». ción # escribe oN, resultando en «teLebicioN».\\
655 DT: bien ¿usted ve la televisión?\\
656 AH: sí\\
657 DT: a ver diga así veo la televisión\\
658 AH: ayer ayer estaba en la televisión\\
659 DT: ¿usted estaba en la televisión?\\
660 AH: sí sí\\
661 DT: ¿la grabaron?\\
662 AH: no hombre # se comienza a reir.\\
663 DT: estaba viendo la televisión\\
664 AH: vien\\
665 DT: viendo la televisión\\
666 AH: viendo la televisión # lo dice al mismo tiempo que DT.\\
667 # Segmento: escritura al dictado con dibujos.\\
668 # Material: una nueva hoja en blanco y un lápiz.\\
669 DT: a ver va a escribir ¿qué hace con un lápiz?\\
670 AH: lápiz\\
671 DT: con el lápiz ¿qué hacemos? # señala el cuaderno de AH.\\
672 AH: escribir\\
673 D: muy bien\\
674 DT: ¿puede escribir aquí una oración usted?\\
675 # Condición: escritura.\\
676 AH: escribir escribir es # escribe «ES». cri a ver # voltea a ver su cuaderno para copiar la palabra. \\
677 DT: ah no sin trampas sin trampas\\
678 AH: escribir es cri bir # dice la palabra separando por sílabas y concluye la palabra resultando en «ESiBiN», el cambio de la grafía de N es sustición de la R ha sido constante.\\
679 # Condición: conversación.\\
680 DT: a ver ¿se parece? vamos a comparar # pone la hoja donde acaba de escribir AH al lado de las palabras escritas en su cuaderno. ¿se parece lo que usted puso?\\
681 AH: no\\
682 DT: ¿no se parece?\\
683 AH: no\\
684 DT: a ver escríbalo aquía bajo\\
685 # Condición: escritura.\\
686 AH: a ver # copia la palabra «escribir» anotada en su cuaderno.\\
687 DT: había faltado ¿verdad?\\
688 AH: sí\\
689 DT: ¿con qué escribimos?\\
690 AH: el lápiz\\
691 DT: a ver escríbalo\\
692 AH: lápiz # copia la palabra lápiz escrita en su cuaderno.\\
693 DT: a ver ¿puede dibujar un lápiz?\\
694 # Condición: dibujo.\\
695 AH: # dibuja al final de la hoja un óvalo pequeño de un centímetro aproximadamente, de la parte inferior del óvalo dibuja una línea recta vertical de un centímetro.\\
696 DT: a ver ¿qué es esto? # señala el dibujo que AH acaba de ralizar.\\
697 AH: pis # le provoca risa y comienza a hacer trazos con el lápiz sobre la palma de su mano izquierda. la ah no\\
698 DT: ¿qué dibujó? platíqueme usted qué dibujó\\
699 AH: a ver # vuelve a hacer trazos sobre la palma de su mano izquierda. la a ver lápiz # comienza a reirse. \\
700 DT: a ver para que se guíe # dibuja un rectángulo debajo de la palabra lápiz que AH escribió. aquí adentro dibuje un lápiz # le entrega el lápiz a AH. dibujado\\
701 AH: # sin habla comienza a dibujar una línea recta vertical, en el extremo superior dibuja un círculo pequeño, dibujo similar al anterior.\\
702 DT: ¿esto qué es? # señala el dibujo reciente.\\
703 AH: lápiz\\
704 DT: ¿esta parte de aquí qué es? # señala el círculo.\\
705 AH: la goma la goma\\
706 DT: ¿y todo esto que dibujó? # señala la línea vertical.\\
707 AH: ...\\
708 DT: a ver ¿esto qué es? # señala la punta del lápiz con el que hizo el dibujo.\\
709 AH: la punta\\
710 D: muy bien\\
711 DT: a ver # dibuja un rectángulo debajo de la palabra «escribir» que AH escribió. ¿con qué escribe?\\
712 AH: una pluma\\
713 DT: ¿con qué partes del cuerpo escribe?\\
714 AH: aquí con la pluma # señala el lápiz con su mano izquierda.\\
715 DT: ¿y qué mueve el lápiz, se mueve sólo?\\
716 AH: no\\
717 DT: ¿cómo se usa? con\\
718 AH: con la mano con la mano\\
719 DT: a ver dibújeme una manita aquí # señala el rectángulo debajo de la palabra «escribir».\\
720 AH: con la mano # dibuja una líea con cinco ángulos como «^^^^^» (podría tratarse de una representación de los dedos de la mano).\\
721 DT: los cinco dedos\\
722 AH: sí\\
723 DT: muy bien entonces escribimos # señala el dibujo de la mano.\\
724 AH: escribir la mano la mano con la mano\\
725 DT: a ver todo completo\\
726 AH: escribir con la cap con la /k/ escribir con # voltea a ver su mano derecha donde sostiene el lápiz. con la mano\\
727 DT: ¿o? # señala el dibujo del lápiz.\\
728 AH: o la o con lápiz\\
729 # Segmento: inicio de sesión.\\
730 # Condición: conversación.\\
731 D: cuénteme qué hizo en las mañanas, en las tardes y en las noches\\
732 AH: yo desayuné\\
733 D: ¿qué más hizo qué desayunó?\\
734 AH: sopa de verduras\\
735 D: ¿qué más?\\
736 AH: y atole\\
737 D: ¿eso qué día fue?\\
738 AH: siempre siempre\\
739 D: ¿todos los días desayuna supa de verduras?\\
740 AH: sí sí\\
741 D: muy bien qué saludable ¿qué más?\\
742 AH: y pescado\\
743 D: ¿qué más hizo en la semana?\\
744 AH: este la tarea este, trapié, la tarea\\
745 D: ¿qué más?\\
746 AH: y duerme y duerme\\
747 D: ay qué bonito es dormir\\
748 AH: sí verdad\\
749 D: a ver cuénteme qué hizo por las tardes\\
750 AH: este duerme y duerme, duerme y duerme # asiente.\\
751 D: ¿qué más qué otra cosa?\\
752 AH: este /s/ duerme y duerme con mis nietos con mis nietos y con mi su esposa de D_\\
753 D: ¿D_ quién es?\\
754 AH: este mi nieto\\
755 D: ah muy bien ¿lo fueron a visitar?\\
756 AH: aquí está en el departamento ¿vea?(¿verdad?) # voltea a ver a AC.\\
757 D: ah viven con usted\\
758 AH: no el (?) a la mitad # voltea a ver a AC.\\
759 AC: viven en el otro ahí somos vecinos\\
760 AH: sí\\
761 D: viven al lado\\
762 AC: llegaron a rentar ahí\\
763 D: ah muy bien muy bien\\
764 AC: A_ mi hijo les consiguió ahí el departamento y pues ya se cambiaron de donde estaban\\
765 AH: sí\\
766 D: es el hijo de A_ entonces\\
767 AH: sí\\
768 AC: el hijo de A_\\
769 D: ¿hizo alguna otra cosa en las noches por ejemplo?\\
770 AH: café con leche y pan tostado\\
771 D: muy bien muy bien ¿algo más?\\
772 AH: el seguro también en la man ayer me fui al seguro\\
773 D: ah muy bien ¿escuchó que usted dijo me fui?\\
774 DT: me fui al seguro me fui al seguro # asiente.\\
775 D: muy bien ahí usó un verbo\\
776 AH: ¿sí? ah sí mira\\
777 D: ¿alguna otra cosa?\\
778 AH: este ya no no sé\\
779 D: muy bien\\
780 AH: el tianguis también el miércoles\\
781 D: usted fue al tianguis\\
782 AH: no porque ya no ya luego está muy feo muy feo\\
783 D: ¿a porque es peligroso?\\
784 AH: pus(pues) sí\\
785 # Segmento: actividad con ilustraciones de objetos.\\
786 # Material: tarjetas de imágenes y las preguntas en el pizarrón ¿qué es? ¿qué se hace?.\\
787 ...\\
788 DT: con este de arriba ya no vamos a trabajar, ya no es la cosa, ya lo tachamos, ahora con ¿qué se hace? ¿lo puede leer otra vez?\\
789 AH: ¿qué se hace?\\
790 DT: yo le voy a poner un ejemplo\\
791 # se le muestra una ilustración con una taza de café\\
792 DT: ¿qué cosa es?\\
793 AH: café\\
794 DT: ¿qué se hace?\\
795 AH: leche con café\\
796 DT: ahora para contestar, este es un ejemplo, ¿qué se ahce con el café? me lo tomo ¿ya vio?\\
797 AH: sí sí\\
798 DT: entonces acá arriba es café y con esta ¿la puede leer otra vez?\\
799 AH: qué hace\\
800 DT: me lo tomo\\
801 AH: me lo tomo me lo tomo me lo tomo\\
802 DT: exacto este fue un ejemplo, entonces vamos a empezar\\
803 AH: sí\\
804 DT: a ver este qué me había dicho\\
805 # se le muestra una ilustración de una estufa\\
806 AH: la estufa\\
807 DT: ¿y qué se hace?\\
808 AH: en la estufa se hace el ca estaba el café la estufa, en la estufa\\
809 DT: sí muy bien ¿y qué se hace con la estufa?\\
810 AH: este esperame, la lumbre la lumbre\\
811 DT: ¿y qué hace usted ahí en la estufa?\\
812 AH: en la lumbre, el café con leche, café el café\\
813 DT: ¿qué más?\\
814 AH: el café,este atole\\
815 DT: ¿y eso cómo se dice? le voy a ayudar, cuando usted pone cosas aquí es cocinar\\
816 AH: cocinar ah pues sí cierto\\
817 DT: entonce vamoa a intentar decirlo todo completo ¿qué se hace en la estufa?\\
818 AH: cocinar\\
819 DT: o puede decir yo cocino\\
820 AH: yo deci(?) yo\\
821 DT: yo cocino\\
822 AH: yo cociné\\
823 DT: a ver señálese así\\
824 AH: yo cociné\\
825 DT: muy bien a ver otra vez, en la estufa\\
826 AH: yo cociné \\
827 DT: muy bien\\
828 ...\\
829 DT: ahora vamos con este ¿qué dijimos que era?\\
830 # se le muestra una ilustración de un teléfono fijo, no celular\\
831 AH: el teléfono\\
832 DT: ¿y qué se hace?\\
833 AH: ¿qué hace? # el teléfono señala la tarjeta con la imagen del teléfono. /rr/ # intenta imitar el sonido de un teléfono.\\
834 DT: ¿y eso cómo se llama? sí está bien\\
835 AH: el teléfono estaba bien /. rue ru ./ # se ríe.\\
836 DT: muy bien pero tiene otra forma de decirlo, hace\\
837 AH: hace ruido\\
838 DT: eso muy bien\\
839 AH: hace ruido \\
840 DT: ¿y usted qué hace con el teléfono? acá me dijo cocino\\
841 # se le muestra la ilustración de la estufa\\
842 DT: y aquí con este\\
843 # se le señala la ilustración del teléfono\\
844 AH: el teléfono\\
845 DT: ¿y qué hace con el teléfono\\
846 AH: este con mis hijos, el teléfono con mis hijos\\
847 DT: ¿pero qué hace con sus hijos?\\
848 AH: con mis hijos en el refri(refigerador) estaba en # comeinza a reirse y se cubre la boca con la mano. el teléfono\\
849 DT: muy bien, le voy a ayudar otra vez, acá en la estufa cocinamos\\
850 # se le señala la imagen de la estufa\\
851 AH: cocinamos\\
852 DT: y por el teléfono o en el teléfono\\
853 # se le hace la seña con la mano de «llamar por teléfono»\\
854 DT: ¿cómo se dice esto?\\
855 AH: ...\\
856 DT: hablamos\\
857 AH: hablando hablamos\\
858 DT: o llamamos\\
859 AH: hablamos ah sí cierto hablamos # no dijo «llamamos» repitió varias veces «hablamos».\\
860 DT: a ver con esta qué hacemos\\
861 # se le señala la ilustración de la estufa\\
862 AH: estufa\\
863 DT: ¿qué hacemos con la estufa?\\
864 AH: la estufa\\
865 DT: ¿qué hacemos?\\
866 AH: eh la estufa\\
867 DT: ¿qué habíamos dicho que hacía?\\
868 AH: estaba en la estufa ...\\
869 DT: cocinamos\\
870 AH: cocinamos\\
871 AT: ¿y aquí?\\
872 # se le señala la ilustración del teléfono\\
873 AH: el teléfono está bien sordo # se ríe y voltea a ver a AC.\\
874 DT: muy bien vamos a ir al siguiente, a ver por ejemplo este qué dijimos\\
875 # se le muestra la ilustración de una canasta con diferentes tipos de pan\\
876 AH: las galletas\\
877 DT: ¿qué se hace con las galletas?\\
878 AH: las galletas ((estaban)) bien sabrosas\\
879 DT: ¿qué hace usted con las galletas?\\
880 ... # interrupción por llamada telefónica\\
881 # Segmento: continuación posterior a la interrupción.\\
882 AH: están bien sabrosas\\
883 DT: ¿y qué hace con las galletas usted?\\
884 AH: a ver, eh pérame(espérame) ay, sí, ay pérame(espérame) ¿qué es? las galletas estaban bien sabrosas\\
885 DT: bien, vamos a pasar a la siguiente, ¿aquí qué dijimos que era?\\
886 # se le muestra una ilustración de una playa\\
887 AH: en pérame(espérame) ... a ver\\
888 DT: ya nos había dicho hace rato\\
889 AH: en ... no no me acuerdo\\
890 DT: empieza con esta letra\\
891 # se da la vuelta a la tarjeta donde esta escrita la letra «M»\\
892 AH: Mar mar\\
893 D: muy bien\\
894 DT: ¿y aquí qué hace?\\
895 AH: el mar estaba bien fea bien feo # se rie\\
896 DT: ¿no le gusta ir al mar?\\
897 DH: no no\\
898 DT: ¿no le gusta?\\
899 D: ¿por qué?\\
900 AH: # voltea a ver a D y asiente sin responder\\
901 DT: ¿y qué puede hacer en el mar?\\
902 AH: nadar, nadando nadando\\
903 D: muy bien\\
904 DT: a ver dígame yo nado\\
905 AH: yo nadé nad(?) ya nadé yo nadé\\
906 DT: muy bien muy bien ahí sí nos salió el verbo, a ver y esto qué era\\
907 # se le muestra una ilustración de una taza vacía\\
908 AH: la ...\\
909 D: ¿con qué letra empieza, se acuerda?\\
910 AH: ay pérame(espérame) no\\
911 DT: vamos a hacer trampa\\
912 # se da la vuelta a la tarjeta donde esta escrito «TA»\\
913 AH: la la talla la taj ay # comeinza a reirse y se cubre la boca.\\
914 DT: la talla\\
915 D: más o menos pero no\\
916 AH: la tas la cosa la cosa # continua riéndose.\\
917 DT: sí es una cosa\\
918 D: sí muy bien\\
919 AH: ¿qué? la\\
920 D: ¿cómo se llama?\\
921 AH: la # deja de reirse. ay mucho tristeza mucha tristeza # se lleva la mano izquierda a la garganta. mucha tristeza mi lengua\\
922 DT: ¿por eso le está costando trabajo?\\
923 AH: sí\\
924 ... # pausa para descansar de la actividad.\\
925 # Segmento: continuación posterior a la pausa.\\
926 # se le muestra la ilustración de la taza vacía\\
927 DT: la taza\\
928 AH: la taza, la taza la taza estaba bien sabrosas\\
929 DT: ¿la taza, se come?\\
930 AH: la el café el café\\
931 DT: muy bien muy bien, es como esta ¿no?\\
932 # se le muestra la ilustración de la taza con café\\
933 AH: sí ajá\\
934 DT: vamos a cambiarla\\
935 # se deja la ilustración de la taza con café y se retira la de la taza vacía\\
936 DT: aquí está el café\\
937 AH: el café\\
938 DT: ¿y qué hace con el café?\\
939 AH: este, bien sabroso\\
940 D: ¿pero qué hace?\\
941 DT: ¿qué hace? ya estamos acá en esta pregunta\\
942 # se le señala la pregunta «¿qué se hace?» escrita en el pizarrón\\
943 DT: ¿qué se hace con el café?\\
944 AH: se hace café, el café ((se hace)) el café\\
945 DT: se toma # hace el gesto con la mano izquierda de «tomar».\\
946 AH: se tomó se tomá el café\\
947 DT: yo me tomé el café\\
948 AH: yo tomé café\\
949 DT: eso muy bien\\
950 ...\\
951 DT: a ver vamos a ver, por ejemplo con este\\
952 # se le muestra una ilustración de un cuchillo de cocina\\
953 AH: el cuchillo\\
954 DT: ¿qué se hace con un cuchillo?\\
955 AH: el cuchillo el cuchillo está bien rasposo # comienza a reírse.\\
956 DT: vamos a borrar esto y dejar el hace\\
957 # se borra lo escrito en el pizarrón blanco dejando únicamente la palabra «hace»\\
958 AH: hace\\
959 DT: ¿qué hacemos con el cuchillo?\\
960 AH: el cuchillo estaba bien filoso\\
961 D: muy bien\\
962 AH: bien filoso\\
963 DT: ¿y usted qué hace con el cuchillo?\\
964 AH: el cuchillo está\\
965 DT: ¿para qué lo usa?\\
966 AH: /mr/ # hace un sonido suave de /r/ no se entiende si es una palabra. ia(mira) aquí # muestra el dedo medio de la mano izquierda y lo señala con el índice derecho.\\
967 D: ajá muy bien\\
968 DT: ¿qué hizo usted?\\
969 AH: se /m/ estaba bien /s/ # sonido de /s/ aspirada. ay, bien filoso\\
970 DT: ¿se cortó?\\
971 AH: sí\\
972 DT: a ver dígalo\\
973 AH: me corté me corté\\
974 DT: eso, con el cuchillo\\
975 AH: cuchillo, me corté cuchillo\\
976 DT: ahí está muy bien, vamos a intentar unos más, a ver ¿con este?\\
977 # se le muestra una ilustración de un tenedor con «TE» escrito al reverso\\
978 AH: no me acuerdo\\
979 DT: ¿cómo se llama? vamos a usar la tramapa\\
980 # se le da la vuelta a la tarjeta para que pueda leer la pista\\
981 DT: ¿cómo se llama?\\
982 AH: ¿qué es, qué es hijo?\\
983 DT: ¿cómo lo usa?\\
984 AH: ...\\
985 DT: ¿cómo lo usa este?\\
986 AH: este qué es\\
987 DT: ¿se come?\\
988 AH: no no hombre no\\
989 DT: pero sí se usa para\\
990 AH: sí sí\\
991 DT: ¿para qué se usa?\\
992 AH: este así # se lleva la mano derecha juntando las puntas de los dedos a la boca en repetidas cosasiones haciendo el gesto de «comer» o «llevarse un bocado a la boca».\\
993 DT: ¿cómo se llama eso?\\
994 AH: no me acuerdo # mueve la mano izquierda cerca de la boca juntando las puntas de los dedos.\\
995 DT: comer\\
996 AH: yo comí este qué # niega levemente con la cabeza y se rie.\\
997 DT: este no se come\\
998 AH: ajá\\
999 DT: pero lo usamos para comer\\
1000 AH: ah sí\\
1001 DT: ¿no se acuerda cómo se llama?\\
1002 AH: no\\
1003 DT: a ver\\
1004 # se vuelve a mostrar el «TE» escrito al reverso de la tarjeta\\
1005 DT: te\\
1006 AH: te\\
1007 DT: ten # haciendo énfasis en el sonido de la /n/\\
1008 AH: tendedor ¿cómo?\\
1009 D: muy bien\\
1010 DT: más o menos, ten\\
1011 AH: ten\\
1012 DT: tene\\
1013 AH: tenedor tenedor\\
1014 DT: eso\\
1015 AH: ah tenedor tenedor \\
1016 DT: a ver ¿para qué se usa el tenedor?\\
1017 AH: el tenedor estaba hace # se rie. el tenedor estaba /fs/ # se lleva la mano izquierda enfrente de la boca y comienza a reírse.\\
1018 DT: a ver vamos a decir comió con el tenedor\\
1019 AH: el tenedor comió comí con el tenedor\\
1020 ... # por conversación de D y DT con AC.\\
1021 DT: a ver ahora este ¿qué dijimos?\\
1022 # se le muestra una ilustración de un vagon del metro de la ciudad de México.\\
1023 AH: el metro\\
1024 DT: ¿y qué hace?\\
1025 AH: /pi/ # alargando el último sonido.\\
1026 DT: sí ¿y eso cómo se llama?\\
1027 AH: este en el metro estaba ((catitlan)) acatitla # se rie.\\
1028 DT: en acatitla\\
1029 AH: sí\\
1030 DT: bueno vamos a decir que ¿hace ruido?\\
1031 AH: sí mucho ruido\\
1032 DT: ¿va lento o rápido?\\
1033 AH: pus(pues) len bien /f/ feo bien \\
1034 DT: ¿bien?\\
1035 AH: bien\\
1036 DT: ¿se mueve?\\
1037 AH: sí hombre\\
1038 DT: a ver vamos a decir se mueve, usted diga, todo completo, el metro\\
1039 AH: el metro estaba bien estaba /fs/ # con las dos manos hace puños y los sube y baja en un par de ocasiones. ¿cómo? # comienza a reírse.\\
1040 DT: ¿lento?\\
1041 AH: no bien\\
1042 DT: ¿bien qué?\\
1043 AH: bien ... no\\
1044 DT: ¿rápido?\\
1045 AH: rápio bien rápido rápido rápido\\
1046 DT: ¿se mueve rápido?\\
1047 AH: sí bien rápido\\
1048 DT: a ver diga así, se mueve rápido\\
1049 AH: ¿cómo?\\
1050 DT: se mueve\\
1051 AH: se bue\\
1052 DT: mueve\\
1053 AH: se mueve el metro\\
1054 DT: ajá\\
1055 AH: el metro est el metro estaba bien bien \\
1056 DT: rápido\\
1057 AH: bien\\
1058 DT: rápido\\
1059 AH: bien rápido\\
1060 DT: a ver este\\
1061 # se le muestra una ilustración de un tambor con el escrito «TAM» al reverso\\
1062 AH: a ver, /m/ ay no me acuerdo\\
1063 DT: a ver voltéelo atrás qué dice\\
1064 AH: tambor\\
1065 DT: ¿qué hace?\\
1066 AH: estaba /psh/ # con ambas manos con puños cerrados sube y baja de forma alternada simulando tocar un tambor. \\
1067 DT: ¿eso cómo se dice?\\
1068 AH: el tambor estaba bien /s/ # repite el gesto con las manos de tocar un tambor.\\
1069 DT: ¿usted qué hace con el tambor?\\
1070 AH: # repite el gesto de tocar el tambor, esta vez la distancia vertical al alternar las manos es mucho más pronunciada y no agrega sonido.\\
1071 DT: ¿eso cómo se llama, cómo se llama hacerle así? # imita el gesto con las manos.\\
1072 AH: ay\\
1073 DT: así como si yo le estuviera haciendo así ¿no? # toma los marcadores para pizarrón y simula tocar un tambor. ¿cómo se dice esto?\\
1074 AH: ¿mucho ruido?\\
1075 DT: muy bien\\
1076 AH: mucho ruido\\
1077 DT: tocar el tambor\\
1078 AH: tocar\\
1079 DT: a ver diga usted\\
1080 AH: tocar el ba el tocador # comienza a reir. estaba el ¿qué es?\\
1081 DT: el tambor\\
1082 AH: el tambor\\
1083 DT: ¿qué hace?\\
1084 AH: estaba /m/ # vuelve a hacer el gesto con ambas manos de tocar un tambor.\\
1085 DT: tocando el tambor\\
1086 AH: tocar\\
1087 DT: y a ver ya el último\\
1088 # se le muestra una ilustración de un plato vacío\\
1089 AH: plato\\
1090 DT: ¿y qué se puede hacer con un plato?\\
1091 AH: la sopa la sopa\\
1092 DT: ¿este se come?\\
1093 AH: no hombre\\
1094 DT: ¿se usa para la comida?\\
1095 AH: sí\\
1096 DT: ¿cómo se usa?\\
1097 AH: este # señala la ilustración. un plato y con las cucharas y sopa\\
1098 DT: entonces servimos la comida\\
1099 AH: sí sí\\
1100 DT: puede ser\\
1101 AH: sí\\
1102 DT: a ver dígalo usted\\
1103 AH: /. sevimos ./ servimos servimos\\
1104 DT: o yo sirvo\\
1105 AH: yo sirvo el sopa\\
1106 DT: muy bien\\
1107 # Segmento: clasificación de palabras en categorías de acción (verbos) o cosas.\\
1108 # Material: pizarrón blanco con un tabla de dos columnas, una  que dice «cosa» y otra «acción».\\
1109 DT: esto ya lo hemos hecho ¿me ayuda a leer?\\
1110 # se le señalan las columnas en el pizarrón\\
1111 AH: cosa acción\\
1112 DT: ¿se acuerda que los estábamos separando? que si esto es una cosa que si esto una acción\\
1113 AH: sí sí cierto\\
1114 DT: por ejemplo vamos a usar su tarea, ¿el lápiz dónde iría, acá o acá? lápiz\\
1115 AH: a ver # señala la columna de acción.\\
1116 DT: ¿y escribir?\\
1117 AH: aquí # señala la columna de cosa.\\
1118 DT: en realidad van al revés pero vamos a empezar a trabajar\\
1119 AH: ah sí sí\\
1120 DT: por ejemplo, ¿esto qué es?\\
1121 # se le muestra una ilustración de una taza de café\\
1122 AH: café\\
1123 DT: ¿es cosa o es acción?\\
1124 AH: este acción # señala la columna de acción.\\
1125 DT: no, va acá # coloca la ilustración en la columna de cosa.\\
1126 AH: ah sí cierto\\
1127 DT: es una cosa, pero a ver ¿qué hace con el café?\\
1128 AH: este el café\\
1129 DT: ¿pero qué hace usted? así si le doy su tacita de café # junta ambas manos acercándolas a AH simulando el gesto de «dar».\\
1130 AH: este bebí bebí\\
1131 DT: eso muy bien\\
1132 D: muy bien\\
1133 AH: bebí\\
1134 DT: a ver escriba aquí bebí # señala la columna de acción.\\
1135 AH: bebí # toma el marcador de pizarrón.\\
1136 # Condición: escritura.\\
1137 AH: be # escribe «Be». be # escribe la letra B. bí # escribe la letra i, resultando en «BeBi».\\
1138 # Condición: conversación.\\
1139 DT: muy bien, esta es la acción # señala la palabra «bebí» que AH acaba de escribir. a ver hágale así de que tomamos # hace el gesto con la mano izquierda de «beber».\\
1140 AH: comí es # señala la ilustración de la taza de café. este el café, café /. bibi ./ bebí café\\
1141 DT: eso ¿ya vio? el café es la cosa es diferente\\
1142 AH: sí cierto\\
1143 DT: a ver vamos a hacerlo otra vez\\
1144 AH: órale\\
1145 DT: estuvo bien eh, así como lo hizo está bien\\
1146 AH: sí\\
1147 DT: ¿qué dijimos que es?\\
1148 AH: café\\
1149 DT: ¿es cosa o es acción?\\
1150 AH: cosa\\
1151 DT: a ver acomódelo\\
1152 # se le entrega la ilustración con la taza de café\\
1153 AH: # coloca la ilustración en la columna de cosa.\\
1154 DT: muy bien, y ¿qué hace conel café? # le entrega el marcador para pizarrón.\\
1155 AH: cosa\\
1156 DT: a ver est ya está acá en cosa # acomoda la ilustración. ¿qué hace con el café usted?\\
1157 AH: este café\\
1158 D: ¿pero qué quedamos que hace con el café?\\
1159 DT: ¿qué hace?\\
1160 AH: bebí\\
1161 DT: muy bien\\
1162 AH: bebí\\
1163 # Condición: escritura.\\
1164 AH: be # escribe «Be». bí # escribe «Bi» resultando en «BeBi».\\
1165 # Condición: conversación.\\
1166 DT: por ejemplo ¿cuál estaría bien?\\
1167 D: el de los tamales\\
1168 DT: a ver si le gusta nuestro dibujito ¿esto qué es?\\
1169 # se le muestra una ilustración de un plato con 2 tamales.\\
1170 AH: tamal\\
1171 DT: my bien ¿y es cosa o es acción?\\
1172 AH: este cosa\\
1173 DT: acomódelo # le entrega la ilustración.\\
1174 AH: cosa # intenta alcanzar el marcador de pizarrón.\\
1175 DT: a ver primero acomode ese aquí encima # señala la columna de cosa.\\
1176 AH: así # coloca la ilustración en la columna señalada.\\
1177 DT: entonce el tamal cosa ¿y qué hace con el tamal?\\
1178 AH: cosa, tamal # intenta escribir en la columna de cosa, acción que no se pide en la actividad.\\
1179 DT: ¿pero qué hace con el tamal, usted qué hace?\\
1180 AH: cosa\\
1181 DT: sí es cosa ¿pero qué hace usted? así le sirvo aquí sus tamales # le hacerca ambas manos. ¿qué hace?\\
1182 AH: tamales, ay pérame(espérame)\\
1183 DT: los tamales ¿los avienta?\\
1184 AH: no no\\
1185 DT: ¿los vende?\\
1186 AH: sí sí\\
1187 DT: los puede vender\\
1188 D: ¿qué otra cosa puede hacer con los tamales? si lo tiene aquí y usted tiene hambre ¿qué hace?\\
1189 AH: acción\\
1190 D: sí muy bien ¿pero cómo se le llama? # hace el gesto con la mano derecha de llevarse algo a la boca. ¿se acuerda? \\
1191 AH: ay no es que me da miedo no sé me \\
1192 D: ¿por qué miedo?\\
1193 AH: # mueve los ojos a la izquierda donde se encuentra AC sin que este se percate, mueve los labios pero no dice nada y sube y baja las cejas.\\
1194 D: no pasa nada\\
1195 DT: a ver le podemos ayudar\\
1196 AH: a ver\\
1197 DT: ¿es comida? # señala la ilustración de los tamales.\\
1198 AH: comida\\
1199 DT: los tamales son comida\\
1200 AH: sí\\
1201 DT: ¿qué hace con la comida? se # hace el gesto de llevarse algo a la boca con la mano derecha.\\
1202 AH: se # hace el gesto de llevarse algo a la boca con la mano izquierda. be be\\
1203 DT: cerca\\
1204 D: más o menos pero no\\
1205 DT: beber es así ¿no? tengo mi vacito o mi tacita # hace el gesto de beber con la mano derecha. pero cuando tengo mi comida y hago esto # repite el gesto de llevarse algo a la boca con la mano derecha.\\
1206 AH: # frunce el seño.\\
1207 DT: ¿cómo se come sus tamales? así abre la hoja\\
1208 AH: sí\\
1209 DT: y ¿qué, con un tenedor?\\
1210 AH: ah un tenedor\\
1211 DT: así lo parte # hace el gesto de llevarse algo a la boca.\\
1212 AH: sí sí ¿y luego? # se empieza a reir.\\
1213 DT: ¿y luego? ajá eso le toca a usted\\
1214 AH: /m/\\
1215 DT: nos lo co\\
1216 AH: comemos\\
1217 DT: eso\\
1218 AH: comemos\\
1219 DT: esa es la acción\\
1220 AH: comemos\\
1221 DT: a ver póngala aquí # señala la columna de acción.\\
1222 AH: comemos\\
1223 # Condición: escritura\\
1224 AH: comemos, co # escribe «co». me # escribe «mi». mos # escribe «mos» resultando en «comimos».\\
1225 # Condición: conversación.\\
1226 DT: los tamales # señala la ilustración y después la palabra que acaba de escribir AH.\\
1227 AH: comemos comimos comemos\\
1228 DT: bueno está bien aquí le cambió una letra, a ver léalo\\
1229 AH: comimos\\
1230 DT: está bien no pasa nada, los tamales ¿qué hacemos con los tamales? # señala nuevamente la palabra escrita.\\
1231 AH: comimos\\
1232 DT: vamos a hacer uno último, para que no se nos sature\\
1233 AH: ajá\\
1234 DT: a ver este ya lo habíamos usado\\
1235 # se le muestra una ilustración de un cuchillo de cocina.\\
1236 AH: cuchara cuchillo\\
1237 DT: ¿y es cosa o es acción?\\
1238 AH: este cosa\\
1239 DT: acomódelo # asiente y le entrega la ilustración.\\
1240 AH: # coloca la ilustración en la columna de cosa.\\
1241 DT: eso # le entrega el marcador de pizarrón. ¿y qué hace con el cuchillo?\\
1242 AH: este cu ...\\
1243 DT: mire to también me # le muestra una cortada pequeña que tiene el brazo. como usted\\
1244 AH: ah mira sí\\
1245 AC: aquí tiene otra # señala otra cortada en el brazo de DT.\\
1246 DT: ah sí aquí tengo otra\\
1247 AH: ah mira\\
1248 DT: nada más que estas no me las hice con un cuchillo, me las hizo mi gato\\
1249 AH: ay ¿a poco? # muestra sorpresa. íjole(expresión de sorpresa)\\
1250 DT: pero usted con el cuchillo ¿qué se hizo?\\
1251 AH: este con ira(mira) # señala su dedo medio de la mano izquierda. con /s/\\
1252 DT: sí ¿cuál es la acción?\\
1253 AH: este cuchillo está bien filoso\\
1254 DT: ¿y qué pasó porque estaba filoso? se # con la mano derecha extendida hace el gesto de «cortar» sobre su mano izquierda.\\
1255 AH: machu/k/ machuqué\\
1256 DT: más o menos\\
1257 D: más o menos\\
1258 AH: machuqué\\
1259 DT: va por ahí es otro verbo # continúa haciendo el gesto de «cortar» con la mano derecha\\
1260 AH: me machuqué\\
1261 DT: me cor\\
1262 AH: me corté me corté\\
1263 DT: eso\\
1264 AH: me corté\\
1265 DT: a ver escriba eso # señala la columna de acción.\\
1266 # Condición: escritura.\\
1267 AH: me # escribe «me». cor # escribe «cor». té # escribe «te» resultando en «me corte».\\
1268 DT: ¿con el cuchillo?\\
1269 AH: cuchillo # asiente pero no produce la oración completa, sólo repite la frase. con el cuchillo\\
1270 DT: muy bien ahora vamos a decirlotodo completo\\
1271 AH: ajá\\
1272 DT: ¿qué pasó con el cuchillo?\\
1273 AH: me corté # lee la palabra escrita el pizarrón.\\
1274 DT: a ver dígalo todo completo\\
1275 AH: me corté\\
1276 D: muy bien\\
1277 DT: con\\
1278 AH: con el cuchillo\\
1279 DT: otra vez todo completo\\
1280 AH: me corté con el cuchillo\\
1281 DT: eso muy bien, a ver ¿este qué habíamos dicho? # le muestra la ilustración  de los tamales nuevamente.\\
1282 AH: el tamal el tamal\\
1283 DT: ajá muy bien # le señala la palabra escrita en el pizarrón «comimos».\\
1284 AH: comimos tamal\\
1285 DT: a ver otra vez\\
1286 AH: comimos tamal\\
1287 DT: eso, ¿y el último? # le muestra la ilustración de la taza con café nuevamente.\\
1288 AH: el café # lee la palabra «BeBi» escrita en el pizarrón. bebí café\\
1289 DT: eso muy bien ¿ya vio la diferencia?\\
1290 AH: ajá sí cierto\\
1291 ...\\
1292 DT: vamos a hacer un último ejercicio\\
1293 AH: sí\\
1294 DT: va a ser un poquito más difícil\\
1295 AH: sí cierto\\
1296 DT: pero lo hizo muy bien\\
1297 ...\\
1298 DT: entonces vamos a ver mire con azul las cosas y con rojo las acciones # se le muestra una hoja blanca de forma horizontal dividida en dos columnas, la izquierda dice «Acción» en color rojo y la derecha «Cosa» en azul. usted ahorita va a escojer, vamos a usar su tarea\\
1299 AH: sí\\
1300 DT: cepillar\\
1301 AH: se\\
1302 DT: cepillar, a ver yo voy a hacer la acción # con la mano derecha hace el gesto de «cepillar los dientes». ¿usted lo hace?\\
1303 AH: sí sí\\
1304 DT: a ver ¿cómo se lava los dientes?\\
1305 AH: # lleva la mano derecha con las puntas de los dedos juntas a la comisura izquierda de sus labios, luego a la derecha y repite esta acción dos veces. sí sí sí\\
1306 DT: esta es la acción\\
1307 AH: acción sí\\
1308 DT: cepillar\\
1309 AH: sí\\
1310 DT: a ver diga cepillar\\
1311 AH: cepillar cepillar\\
1312 DT: ¿dónde va? # señala la hoja.\\
1313 AH: aquí # señala la columna que dice acción. acción\\
1314 DT: muy bien # le da el marcador rojo.\\
1315 AH: ¿cepillar?\\
1316 DT: sí\\
1317 # Condición: escritura.\\
1318 AH: # escribe la letra c. ce # escribe «cep». pi # escribe la letra i segido de «ll». llar # escribe «ar» resultando en «cepillar».\\
1319 DT: muy bien esa es la acción\\
1320 AH: sí ajá\\
1321 DT: y ¿dientes?\\
1322 AH: los dientes aquí # señala la columna de cosa. cosa\\
1323 DT: eso muy bien a ver\\
1324 AH: cosa\\
1325 DT: dientes\\
1326 AH: dientes\\
1327 # Condición: escritura.\\
1328 AH: # escribe «diE». dien # escribe «Nte». tes # escribe la letra s resultando en «diENtes».\\
1329 # Condición: conversación.\\
1330 DT: muy bien a ver ahora dígalo # le señala la palabra cepillar y luego dientes.\\
1331 AH: cepillar dientes\\
1332 DT: ¿cómo lo puede decir? yo\\
1333 AH: yo yo cepillé los dientes\\
1334 DT: perfecto, a ver vamos a intetar ahora con agua\\
1335 AH: /a/ agua # señala primero la columna cosa y luego la de acción, comienza a hacer un movimient con el dedo índice de la mano derecha sobre la hoja en la columna de cosa simulando la escritura de la palabra requerida, parece escribir en el aire «agi».\\
1336 DT: a ver ¿es una cosa el agua?\\
1337 AH: acción # señala la columna de cosa. agua agua\\
1338 DT: estaba bien aquí # señala al mismo tiempo que AH la columna de cosa.\\
1339 # Condición: escritura.\\
1340 AH: # sin decir la palabra escribe «agui» similar a la acción anterior de escribir en el aire con el dedo índice. /a/ agua # nota el error y corrige escribiendo sobre la letra i una A, resultando en «aguA».\\
1341 DT: muy bien ¿y servir?\\
1342 AH: /. serville serv ./ \\
1343 DT: servir\\
1344 AH: servir aquí # señala la columna de acción. acción\\
1345 DT: sí muy bien a ver\\
1346 AH: ((vio))\\
1347 DT: servir\\
1348 AH: servir\\
1349 # Condición: escritura.\\
1350 AH: ser # escribe «CE». /s/ ser # escribe la letra N resultando en «CEN». ¿no?\\
1351 DT: servir\\
1352 AH: servir # escribe la letra d. ser # escribe la letra i. vir # escribe la letra N, resultando en «CENdiN».\\
1353 DT: no pasa nada ¿cómo sería completo? # señala la palabra escrita por AH «CENdiN» y después\\
1354 AH: servir\\
1355 DT: servir # señala la palabra «aguA».\\
1356 AH: agua\\
1357 DT: a ver ahora ¿cómo lo diría completo?\\
1358 AH: servir agua coca este cosa\\
1359 DT: ah ya se le antojó la coca\\
1360 AH: cosa # comienza a reirse.\\
1361 DT: muy bien, y ahora vamos a ponerle uno que no ha hecho usted, por ejemplo ver\\
1362 AH: ah\\
1363 DT: ¿qué es cosa o acción?\\
1364 AH: ver ver\\
1365 DT: ver ¿cuál es?\\
1366 AH: a ver # señala la columna de cosa.\\
1367 DT: ¿segura?\\
1368 AH: acción # continúa señalando la misma columna. ver\\
1369 DT: a ver # le da el marcador rojo.\\
1370 AH: ver\\
1371 # Condición: escritura.\\
1372 AH: ver # escribe «ER». ver ver \\
1373 DT: a ver # cubre con un objeto lo que AH acaba de escribir. aquí abajito escríbalo, ver\\
1374 AH: ver # escribe «Per»\\
1375 DT: y a ver, tele(televisión)\\
1376 AH: ¿tele?\\
1377 DT: la tele\\
1378 AH: coca cosa cosa cosa\\
1379 DT: muy bien, sí se le antojó la coca\\
1380 AH: # se rie pero no escribe.\\
1381 DT: tele, la tele\\
1382 AH: # escribe «te». tele # escribe «Le». vi # escribe «b». vi # escribe la letra i. /s/ # escribe «ci». ción # escribe oN, resultando en «teLebicioN».\\
1383 DT: bien ¿usted ve la televisión?\\
1384 AH: sí\\
1385 DT: a ver diga así veo la televisión\\
1386 AH: ayer ayer estaba en la televisión\\
1387 DT: ¿usted estaba en la televisión?\\
1388 AH: sí sí\\
1389 DT: ¿la grabaron?\\
1390 AH: no hombre # se comienza a reir.\\
1391 DT: estaba viendo la televisión\\
1392 AH: vien\\
1393 DT: viendo la televisión\\
1394 AH: viendo la televisión # lo dice al mismo tiempo que DT.\\
1395 # Segmento: escritura al dictado con dibujos.\\
1396 # Material: una nueva hoja en blanco y un lápiz.\\
1397 DT: a ver va a escribir ¿qué hace con un lápiz?\\
1398 AH: lápiz\\
1399 DT: con el lápiz ¿qué hacemos? # señala el cuaderno de AH.\\
1400 AH: escribir\\
1401 D: muy bien\\
1402 DT: ¿puede escribir aquí una oración usted?\\
1403 # Condición: escritura.\\
1404 AH: escribir escribir es # escribe «ES». cri a ver # voltea a ver su cuaderno para copiar la palabra. \\
1405 DT: ah no sin trampas sin trampas\\
1406 AH: escribir es cri bir # dice la palabra separando por sílabas y concluye la palabra resultando en «ESiBiN», el cambio de la grafía de N es sustición de la R ha sido constante.\\
1407 # Condición: conversación.\\
1408 DT: a ver ¿se parece? vamos a comparar # pone la hoja donde acaba de escribir AH al lado de las palabras escritas en su cuaderno. ¿se parece lo que usted puso?\\
1409 AH: no\\
1410 DT: ¿no se parece?\\
1411 AH: no\\
1412 DT: a ver escríbalo aquía bajo\\
1413 # Condición: escritura.\\
1414 AH: a ver # copia la palabra «escribir» anotada en su cuaderno.\\
1415 DT: había faltado ¿verdad?\\
1416 AH: sí\\
1417 DT: ¿con qué escribimos?\\
1418 AH: el lápiz\\
1419 DT: a ver escríbalo\\
1420 AH: lápiz # copia la palabra lápiz escrita en su cuaderno.\\
1421 DT: a ver ¿puede dibujar un lápiz?\\
1422 # Condición: dibujo.\\
1423 AH: # dibuja al final de la hoja un óvalo pequeño de un centímetro aproximadamente, de la parte inferior del óvalo dibuja una línea recta vertical de un centímetro.\\
1424 DT: a ver ¿qué es esto? # señala el dibujo que AH acaba de ralizar.\\
1425 AH: pis # le provoca risa y comienza a hacer trazos con el lápiz sobre la palma de su mano izquierda. la ah no\\
1426 DT: ¿qué dibujó? platíqueme usted qué dibujó\\
1427 AH: a ver # vuelve a hacer trazos sobre la palma de su mano izquierda. la a ver lápiz # comienza a reirse. \\
1428 DT: a ver para que se guíe # dibuja un rectángulo debajo de la palabra lápiz que AH escribió. aquí adentro dibuje un lápiz # le entrega el lápiz a AH. dibujado\\
1429 AH: # sin habla comienza a dibujar una línea recta vertical, en el extremo superior dibuja un círculo pequeño, dibujo similar al anterior.\\
1430 DT: ¿esto qué es? # señala el dibujo reciente.\\
1431 AH: lápiz\\
1432 DT: ¿esta parte de aquí qué es? # señala el círculo.\\
1433 AH: la goma la goma\\
1434 DT: ¿y todo esto que dibujó? # señala la línea vertical.\\
1435 AH: ...\\
1436 DT: a ver ¿esto qué es? # señala la punta del lápiz con el que hizo el dibujo.\\
1437 AH: la punta\\
1438 D: muy bien\\
1439 DT: a ver # dibuja un rectángulo debajo de la palabra «escribir» que AH escribió. ¿con qué escribe?\\
1440 AH: una pluma\\
1441 DT: ¿con qué partes del cuerpo escribe?\\
1442 AH: aquí con la pluma # señala el lápiz con su mano izquierda.\\
1443 DT: ¿y qué mueve el lápiz, se mueve sólo?\\
1444 AH: no\\
1445 DT: ¿cómo se usa? con\\
1446 AH: con la mano con la mano\\
1447 DT: a ver dibújeme una manita aquí # señala el rectángulo debajo de la palabra «escribir».\\
1448 AH: con la mano # dibuja una líea con cinco ángulos como «^^^^^» (podría tratarse de una representación de los dedos de la mano).\\
1449 DT: los cinco dedos\\
1450 AH: sí\\
1451 DT: muy bien entonces escribimos # señala el dibujo de la mano.\\
1452 AH: escribir la mano la mano con la mano\\
1453 DT: a ver todo completo\\
1454 AH: escribir con la cap con la /k/ escribir con # voltea a ver su mano derecha donde sostiene el lápiz. con la mano\\
1455 DT: ¿o? # señala el dibujo del lápiz.\\
1456 AH: o la o con lápiz\\
1457 \\
1458 001 # Segmento: inicio de sesión.\\
1459 002 # Condición: conversación.\\
1460 003 D: cuénteme qué hizo en las mañanas, en las tardes y en las noches\\
1461 004 AH: yo desayuné\\
1462 005 D: ¿qué más hizo qué desayunó?\\
1463 006 AH: sopa de verduras\\
1464 007 D: ¿qué más?\\
1465 008 AH: y atole\\
1466 009 D: ¿eso qué día fue?\\
1467 010 AH: siempre siempre\\
1468 011 D: ¿todos los días desayuna supa de verduras?\\
1469 012 AH: sí sí\\
1470 013 D: muy bien qué saludable ¿qué más?\\
1471 014 AH: y pescado\\
1472 015 D: ¿qué más hizo en la semana?\\
1473 016 AH: este la tarea este, trapié, la tarea\\
1474 017 D: ¿qué más?\\
1475 018 AH: y duerme y duerme\\
1476 019 D: ay qué bonito es dormir\\
1477 020 AH: sí verdad\\
1478 021 D: a ver cuénteme qué hizo por las tardes\\
1479 022 AH: este duerme y duerme, duerme y duerme # asiente.\\
1480 023 D: ¿qué más qué otra cosa?\\
1481 024 AH: este /s/ duerme y duerme con mis nietos con mis nietos y con mi su esposa de D_\\
1482 025 D: ¿D_ quién es?\\
1483 026 AH: este mi nieto\\
1484 027 D: ah muy bien ¿lo fueron a visitar?\\
1485 028 AH: aquí está en el departamento ¿vea?(¿verdad?) # voltea a ver a AC.\\
1486 029 D: ah viven con usted\\
1487 030 AH: no el (?) a la mitad # voltea a ver a AC.\\
1488 031 AC: viven en el otro ahí somos vecinos\\
1489 032 AH: sí\\
1490 033 D: viven al lado\\
1491 034 AC: llegaron a rentar ahí\\
1492 035 D: ah muy bien muy bien\\
1493 036 AC: A_ mi hijo les consiguió ahí el departamento y pues ya se cambiaron de donde estaban\\
1494 037 AH: sí\\
1495 038 D: es el hijo de A_ entonces\\
1496 039 AH: sí\\
1497 040 AC: el hijo de A_\\
1498 041 D: ¿hizo alguna otra cosa en las noches por ejemplo?\\
1499 042 AH: café con leche y pan tostado\\
1500 043 D: muy bien muy bien ¿algo más?\\
1501 044 AH: el seguro también en la man ayer me fui al seguro\\
1502 045 D: ah muy bien ¿escuchó que usted dijo me fui?\\
1503 046 DT: me fui al seguro me fui al seguro # asiente.\\
1504 047 D: muy bien ahí usó un verbo\\
1505 048 AH: ¿sí? ah sí mira\\
1506 049 D: ¿alguna otra cosa?\\
1507 050 AH: este ya no no sé\\
1508 051 D: muy bien\\
1509 052 AH: el tianguis también el miércoles\\
1510 053 D: usted fue al tianguis\\
1511 054 AH: no porque ya no ya luego está muy feo muy feo\\
1512 055 D: ¿a porque es peligroso?\\
1513 056 AH: pus(pues) sí\\
1514 057 # Segmento: actividad con ilustraciones de objetos.\\
1515 058 # Material: tarjetas de imágenes y las preguntas en el pizarrón ¿qué es? ¿qué se hace?.\\
1516 059 ...\\
1517 060 DT: con este de arriba ya no vamos a trabajar, ya no es la cosa, ya lo tachamos, ahora con ¿qué se hace? ¿lo puede leer otra vez?\\
1518 061 AH: ¿qué se hace?\\
1519 062 DT: yo le voy a poner un ejemplo\\
1520 063 # se le muestra una ilustración con una taza de café\\
1521 064 DT: ¿qué cosa es?\\
1522 065 AH: café\\
1523 066 DT: ¿qué se hace?\\
1524 067 AH: leche con café\\
1525 068 DT: ahora para contestar, este es un ejemplo, ¿qué se ahce con el café? me lo tomo ¿ya vio?\\
1526 069 AH: sí sí\\
1527 070 DT: entonces acá arriba es café y con esta ¿la puede leer otra vez?\\
1528 071 AH: qué hace\\
1529 072 DT: me lo tomo\\
1530 073 AH: me lo tomo me lo tomo me lo tomo\\
1531 074 DT: exacto este fue un ejemplo, entonces vamos a empezar\\
1532 075 AH: sí\\
1533 076 DT: a ver este qué me había dicho\\
1534 077 # se le muestra una ilustración de una estufa\\
1535 078 AH: la estufa\\
1536 079 DT: ¿y qué se hace?\\
1537 080 AH: en la estufa se hace el ca estaba el café la estufa, en la estufa\\
1538 081 DT: sí muy bien ¿y qué se hace con la estufa?\\
1539 082 AH: este esperame, la lumbre la lumbre\\
1540 083 DT: ¿y qué hace usted ahí en la estufa?\\
1541 084 AH: en la lumbre, el café con leche, café el café\\
1542 085 DT: ¿qué más?\\
1543 086 AH: el café,este atole\\
1544 087 DT: ¿y eso cómo se dice? le voy a ayudar, cuando usted pone cosas aquí es cocinar\\
1545 088 AH: cocinar ah pues sí cierto\\
1546 089 DT: entonce vamoa a intentar decirlo todo completo ¿qué se hace en la estufa?\\
1547 090 AH: cocinar\\
1548 091 DT: o puede decir yo cocino\\
1549 092 AH: yo deci(?) yo\\
1550 093 DT: yo cocino\\
1551 094 AH: yo cociné\\
1552 095 DT: a ver señálese así\\
1553 096 AH: yo cociné\\
1554 097 DT: muy bien a ver otra vez, en la estufa\\
1555 098 AH: yo cociné \\
1556 099 DT: muy bien\\
1557 100 ...\\
1558 101 DT: ahora vamos con este ¿qué dijimos que era?\\
1559 102 # se le muestra una ilustración de un teléfono fijo, no celular\\
1560 103 AH: el teléfono\\
1561 104 DT: ¿y qué se hace?\\
1562 105 AH: ¿qué hace? # el teléfono señala la tarjeta con la imagen del teléfono. /rr/ # intenta imitar el sonido de un teléfono.\\
1563 106 DT: ¿y eso cómo se llama? sí está bien\\
1564 107 AH: el teléfono estaba bien /. rue ru ./ # se ríe.\\
1565 108 DT: muy bien pero tiene otra forma de decirlo, hace\\
1566 109 AH: hace ruido\\
1567 110 DT: eso muy bien\\
1568 111 AH: hace ruido \\
1569 112 DT: ¿y usted qué hace con el teléfono? acá me dijo cocino\\
1570 113 # se le muestra la ilustración de la estufa\\
1571 114 DT: y aquí con este\\
1572 115 # se le señala la ilustración del teléfono\\
1573 116 AH: el teléfono\\
1574 117 DT: ¿y qué hace con el teléfono\\
1575 118 AH: este con mis hijos, el teléfono con mis hijos\\
1576 119 DT: ¿pero qué hace con sus hijos?\\
1577 120 AH: con mis hijos en el refri(refigerador) estaba en # comeinza a reirse y se cubre la boca con la mano. el teléfono\\
1578 121 DT: muy bien, le voy a ayudar otra vez, acá en la estufa cocinamos\\
1579 122 # se le señala la imagen de la estufa\\
1580 123 AH: cocinamos\\
1581 124 DT: y por el teléfono o en el teléfono\\
1582 125 # se le hace la seña con la mano de «llamar por teléfono»\\
1583 126 DT: ¿cómo se dice esto?\\
1584 127 AH: ...\\
1585 128 DT: hablamos\\
1586 129 AH: hablando hablamos\\
1587 130 DT: o llamamos\\
1588 131 AH: hablamos ah sí cierto hablamos # no dijo «llamamos» repitió varias veces «hablamos».\\
1589 132 DT: a ver con esta qué hacemos\\
1590 133 # se le señala la ilustración de la estufa\\
1591 134 AH: estufa\\
1592 135 DT: ¿qué hacemos con la estufa?\\
1593 136 AH: la estufa\\
1594 137 DT: ¿qué hacemos?\\
1595 138 AH: eh la estufa\\
1596 139 DT: ¿qué habíamos dicho que hacía?\\
1597 140 AH: estaba en la estufa ...\\
1598 141 DT: cocinamos\\
1599 142 AH: cocinamos\\
1600 143 AT: ¿y aquí?\\
1601 144 # se le señala la ilustración del teléfono\\
1602 145 AH: el teléfono está bien sordo # se ríe y voltea a ver a AC.\\
1603 146 DT: muy bien vamos a ir al siguiente, a ver por ejemplo este qué dijimos\\
1604 147 # se le muestra la ilustración de una canasta con diferentes tipos de pan\\
1605 148 AH: las galletas\\
1606 149 DT: ¿qué se hace con las galletas?\\
1607 150 AH: las galletas ((estaban)) bien sabrosas\\
1608 151 DT: ¿qué hace usted con las galletas?\\
1609 152 ... # interrupción por llamada telefónica\\
1610 153 # Segmento: continuación posterior a la interrupción.\\
1611 154 AH: están bien sabrosas\\
1612 155 DT: ¿y qué hace con las galletas usted?\\
1613 156 AH: a ver, eh pérame(espérame) ay, sí, ay pérame(espérame) ¿qué es? las galletas estaban bien sabrosas\\
1614 157 DT: bien, vamos a pasar a la siguiente, ¿aquí qué dijimos que era?\\
1615 158 # se le muestra una ilustración de una playa\\
1616 159 AH: en pérame(espérame) ... a ver\\
1617 160 DT: ya nos había dicho hace rato\\
1618 161 AH: en ... no no me acuerdo\\
1619 162 DT: empieza con esta letra\\
1620 163 # se da la vuelta a la tarjeta donde esta escrita la letra «M»\\
1621 164 AH: Mar mar\\
1622 165 D: muy bien\\
1623 166 DT: ¿y aquí qué hace?\\
1624 167 AH: el mar estaba bien fea bien feo # se rie\\
1625 168 DT: ¿no le gusta ir al mar?\\
1626 169 DH: no no\\
1627 170 DT: ¿no le gusta?\\
1628 171 D: ¿por qué?\\
1629 172 AH: # voltea a ver a D y asiente sin responder\\
1630 173 DT: ¿y qué puede hacer en el mar?\\
1631 174 AH: nadar, nadando nadando\\
1632 175 D: muy bien\\
1633 176 DT: a ver dígame yo nado\\
1634 177 AH: yo nadé nad(?) ya nadé yo nadé\\
1635 178 DT: muy bien muy bien ahí sí nos salió el verbo, a ver y esto qué era\\
1636 179 # se le muestra una ilustración de una taza vacía\\
1637 180 AH: la ...\\
1638 181 D: ¿con qué letra empieza, se acuerda?\\
1639 182 AH: ay pérame(espérame) no\\
1640 183 DT: vamos a hacer trampa\\
1641 184 # se da la vuelta a la tarjeta donde esta escrito «TA»\\
1642 185 AH: la la talla la taj ay # comeinza a reirse y se cubre la boca.\\
1643 186 DT: la talla\\
1644 187 D: más o menos pero no\\
1645 188 AH: la tas la cosa la cosa # continua riéndose.\\
1646 189 DT: sí es una cosa\\
1647 190 D: sí muy bien\\
1648 191 AH: ¿qué? la\\
1649 192 D: ¿cómo se llama?\\
1650 193 AH: la # deja de reirse. ay mucho tristeza mucha tristeza # se lleva la mano izquierda a la garganta. mucha tristeza mi lengua\\
1651 194 DT: ¿por eso le está costando trabajo?\\
1652 195 AH: sí\\
1653 196 ... # pausa para descansar de la actividad.\\
1654 197 # Segmento: continuación posterior a la pausa.\\
1655 198 # se le muestra la ilustración de la taza vacía\\
1656 199 DT: la taza\\
1657 200 AH: la taza, la taza la taza estaba bien sabrosas\\
1658 201 DT: ¿la taza, se come?\\
1659 202 AH: la el café el café\\
1660 203 DT: muy bien muy bien, es como esta ¿no?\\
1661 204 # se le muestra la ilustración de la taza con café\\
1662 205 AH: sí ajá\\
1663 206 DT: vamos a cambiarla\\
1664 207 # se deja la ilustración de la taza con café y se retira la de la taza vacía\\
1665 208 DT: aquí está el café\\
1666 209 AH: el café\\
1667 210 DT: ¿y qué hace con el café?\\
1668 211 AH: este, bien sabroso\\
1669 212 D: ¿pero qué hace?\\
1670 213 DT: ¿qué hace? ya estamos acá en esta pregunta\\
1671 214 # se le señala la pregunta «¿qué se hace?» escrita en el pizarrón\\
1672 215 DT: ¿qué se hace con el café?\\
1673 216 AH: se hace café, el café ((se hace)) el café\\
1674 217 DT: se toma # hace el gesto con la mano izquierda de «tomar».\\
1675 218 AH: se tomó se tomá el café\\
1676 219 DT: yo me tomé el café\\
1677 220 AH: yo tomé café\\
1678 221 DT: eso muy bien\\
1679 222 ...\\
1680 223 DT: a ver vamos a ver, por ejemplo con este\\
1681 224 # se le muestra una ilustración de un cuchillo de cocina\\
1682 225 AH: el cuchillo\\
1683 226 DT: ¿qué se hace con un cuchillo?\\
1684 227 AH: el cuchillo el cuchillo está bien rasposo # comienza a reírse.\\
1685 228 DT: vamos a borrar esto y dejar el hace\\
1686 229 # se borra lo escrito en el pizarrón blanco dejando únicamente la palabra «hace»\\
1687 230 AH: hace\\
1688 231 DT: ¿qué hacemos con el cuchillo?\\
1689 232 AH: el cuchillo estaba bien filoso\\
1690 233 D: muy bien\\
1691 234 AH: bien filoso\\
1692 235 DT: ¿y usted qué hace con el cuchillo?\\
1693 236 AH: el cuchillo está\\
1694 237 DT: ¿para qué lo usa?\\
1695 238 AH: /mr/ # hace un sonido suave de /r/ no se entiende si es una palabra. ia(mira) aquí # muestra el dedo medio de la mano izquierda y lo señala con el índice derecho.\\
1696 239 D: ajá muy bien\\
1697 240 DT: ¿qué hizo usted?\\
1698 241 AH: se /m/ estaba bien /s/ # sonido de /s/ aspirada. ay, bien filoso\\
1699 242 DT: ¿se cortó?\\
1700 243 AH: sí\\
1701 244 DT: a ver dígalo\\
1702 245 AH: me corté me corté\\
1703 246 DT: eso, con el cuchillo\\
1704 247 AH: cuchillo, me corté cuchillo\\
1705 248 DT: ahí está muy bien, vamos a intentar unos más, a ver ¿con este?\\
1706 249 # se le muestra una ilustración de un tenedor con «TE» escrito al reverso\\
1707 250 AH: no me acuerdo\\
1708 251 DT: ¿cómo se llama? vamos a usar la tramapa\\
1709 252 # se le da la vuelta a la tarjeta para que pueda leer la pista\\
1710 253 DT: ¿cómo se llama?\\
1711 254 AH: ¿qué es, qué es hijo?\\
1712 255 DT: ¿cómo lo usa?\\
1713 256 AH: ...\\
1714 257 DT: ¿cómo lo usa este?\\
1715 258 AH: este qué es\\
1716 259 DT: ¿se come?\\
1717 260 AH: no no hombre no\\
1718 261 DT: pero sí se usa para\\
1719 262 AH: sí sí\\
1720 263 DT: ¿para qué se usa?\\
1721 264 AH: este así # se lleva la mano derecha juntando las puntas de los dedos a la boca en repetidas cosasiones haciendo el gesto de «comer» o «llevarse un bocado a la boca».\\
1722 265 DT: ¿cómo se llama eso?\\
1723 266 AH: no me acuerdo # mueve la mano izquierda cerca de la boca juntando las puntas de los dedos.\\
1724 267 DT: comer\\
1725 268 AH: yo comí este qué # niega levemente con la cabeza y se rie.\\
1726 269 DT: este no se come\\
1727 270 AH: ajá\\
1728 271 DT: pero lo usamos para comer\\
1729 272 AH: ah sí\\
1730 273 DT: ¿no se acuerda cómo se llama?\\
1731 274 AH: no\\
1732 275 DT: a ver\\
1733 276 # se vuelve a mostrar el «TE» escrito al reverso de la tarjeta\\
1734 277 DT: te\\
1735 278 AH: te\\
1736 279 DT: ten # haciendo énfasis en el sonido de la /n/\\
1737 280 AH: tendedor ¿cómo?\\
1738 281 D: muy bien\\
1739 282 DT: más o menos, ten\\
1740 283 AH: ten\\
1741 284 DT: tene\\
1742 285 AH: tenedor tenedor\\
1743 286 DT: eso\\
1744 287 AH: ah tenedor tenedor \\
1745 288 DT: a ver ¿para qué se usa el tenedor?\\
1746 289 AH: el tenedor estaba hace # se rie. el tenedor estaba /fs/ # se lleva la mano izquierda enfrente de la boca y comienza a reírse.\\
1747 290 DT: a ver vamos a decir comió con el tenedor\\
1748 291 AH: el tenedor comió comí con el tenedor\\
1749 292 ... # por conversación de D y DT con AC.\\
1750 293 DT: a ver ahora este ¿qué dijimos?\\
1751 294 # se le muestra una ilustración de un vagon del metro de la ciudad de México.\\
1752 295 AH: el metro\\
1753 296 DT: ¿y qué hace?\\
1754 297 AH: /pi/ # alargando el último sonido.\\
1755 298 DT: sí ¿y eso cómo se llama?\\
1756 299 AH: este en el metro estaba ((catitlan)) acatitla # se rie.\\
1757 300 DT: en acatitla\\
1758 301 AH: sí\\
1759 302 DT: bueno vamos a decir que ¿hace ruido?\\
1760 303 AH: sí mucho ruido\\
1761 304 DT: ¿va lento o rápido?\\
1762 305 AH: pus(pues) len bien /f/ feo bien \\
1763 306 DT: ¿bien?\\
1764 307 AH: bien\\
1765 308 DT: ¿se mueve?\\
1766 309 AH: sí hombre\\
1767 310 DT: a ver vamos a decir se mueve, usted diga, todo completo, el metro\\
1768 311 AH: el metro estaba bien estaba /fs/ # con las dos manos hace puños y los sube y baja en un par de ocasiones. ¿cómo? # comienza a reírse.\\
1769 312 DT: ¿lento?\\
1770 313 AH: no bien\\
1771 314 DT: ¿bien qué?\\
1772 315 AH: bien ... no\\
1773 316 DT: ¿rápido?\\
1774 317 AH: rápio bien rápido rápido rápido\\
1775 318 DT: ¿se mueve rápido?\\
1776 319 AH: sí bien rápido\\
1777 320 DT: a ver diga así, se mueve rápido\\
1778 321 AH: ¿cómo?\\
1779 322 DT: se mueve\\
1780 323 AH: se bue\\
1781 324 DT: mueve\\
1782 325 AH: se mueve el metro\\
1783 326 DT: ajá\\
1784 327 AH: el metro est el metro estaba bien bien \\
1785 328 DT: rápido\\
1786 329 AH: bien\\
1787 330 DT: rápido\\
1788 331 AH: bien rápido\\
1789 332 DT: a ver este\\
1790 333 # se le muestra una ilustración de un tambor con el escrito «TAM» al reverso\\
1791 334 AH: a ver, /m/ ay no me acuerdo\\
1792 335 DT: a ver voltéelo atrás qué dice\\
1793 336 AH: tambor\\
1794 337 DT: ¿qué hace?\\
1795 338 AH: estaba /psh/ # con ambas manos con puños cerrados sube y baja de forma alternada simulando tocar un tambor. \\
1796 339 DT: ¿eso cómo se dice?\\
1797 340 AH: el tambor estaba bien /s/ # repite el gesto con las manos de tocar un tambor.\\
1798 341 DT: ¿usted qué hace con el tambor?\\
1799 342 AH: # repite el gesto de tocar el tambor, esta vez la distancia vertical al alternar las manos es mucho más pronunciada y no agrega sonido.\\
1800 343 DT: ¿eso cómo se llama, cómo se llama hacerle así? # imita el gesto con las manos.\\
1801 344 AH: ay\\
1802 345 DT: así como si yo le estuviera haciendo así ¿no? # toma los marcadores para pizarrón y simula tocar un tambor. ¿cómo se dice esto?\\
1803 346 AH: ¿mucho ruido?\\
1804 347 DT: muy bien\\
1805 348 AH: mucho ruido\\
1806 349 DT: tocar el tambor\\
1807 350 AH: tocar\\
1808 351 DT: a ver diga usted\\
1809 352 AH: tocar el ba el tocador # comienza a reir. estaba el ¿qué es?\\
1810 353 DT: el tambor\\
1811 354 AH: el tambor\\
1812 355 DT: ¿qué hace?\\
1813 356 AH: estaba /m/ # vuelve a hacer el gesto con ambas manos de tocar un tambor.\\
1814 357 DT: tocando el tambor\\
1815 358 AH: tocar\\
1816 359 DT: y a ver ya el último\\
1817 360 # se le muestra una ilustración de un plato vacío\\
1818 361 AH: plato\\
1819 362 DT: ¿y qué se puede hacer con un plato?\\
1820 363 AH: la sopa la sopa\\
1821 364 DT: ¿este se come?\\
1822 365 AH: no hombre\\
1823 366 DT: ¿se usa para la comida?\\
1824 367 AH: sí\\
1825 368 DT: ¿cómo se usa?\\
1826 369 AH: este # señala la ilustración. un plato y con las cucharas y sopa\\
1827 370 DT: entonces servimos la comida\\
1828 371 AH: sí sí\\
1829 372 DT: puede ser\\
1830 373 AH: sí\\
1831 374 DT: a ver dígalo usted\\
1832 375 AH: /. sevimos ./ servimos servimos\\
1833 376 DT: o yo sirvo\\
1834 377 AH: yo sirvo el sopa\\
1835 378 DT: muy bien\\
1836 379 # Segmento: clasificación de palabras en categorías de acción (verbos) o cosas.\\
1837 380 # Material: pizarrón blanco con un tabla de dos columnas, una  que dice «cosa» y otra «acción».\\
1838 381 DT: esto ya lo hemos hecho ¿me ayuda a leer?\\
1839 382 # se le señalan las columnas en el pizarrón\\
1840 383 AH: cosa acción\\
1841 384 DT: ¿se acuerda que los estábamos separando? que si esto es una cosa que si esto una acción\\
1842 385 AH: sí sí cierto\\
1843 386 DT: por ejemplo vamos a usar su tarea, ¿el lápiz dónde iría, acá o acá? lápiz\\
1844 387 AH: a ver # señala la columna de acción.\\
1845 388 DT: ¿y escribir?\\
1846 389 AH: aquí # señala la columna de cosa.\\
1847 390 DT: en realidad van al revés pero vamos a empezar a trabajar\\
1848 391 AH: ah sí sí\\
1849 392 DT: por ejemplo, ¿esto qué es?\\
1850 393 # se le muestra una ilustración de una taza de café\\
1851 394 AH: café\\
1852 395 DT: ¿es cosa o es acción?\\
1853 396 AH: este acción # señala la columna de acción.\\
1854 397 DT: no, va acá # coloca la ilustración en la columna de cosa.\\
1855 398 AH: ah sí cierto\\
1856 399 DT: es una cosa, pero a ver ¿qué hace con el café?\\
1857 400 AH: este el café\\
1858 401 DT: ¿pero qué hace usted? así si le doy su tacita de café # junta ambas manos acercándolas a AH simulando el gesto de «dar».\\
1859 402 AH: este bebí bebí\\
1860 403 DT: eso muy bien\\
1861 404 D: muy bien\\
1862 405 AH: bebí\\
1863 406 DT: a ver escriba aquí bebí # señala la columna de acción.\\
1864 407 AH: bebí # toma el marcador de pizarrón.\\
1865 408 # Condición: escritura.\\
1866 409 AH: be # escribe «Be». be # escribe la letra B. bí # escribe la letra i, resultando en «BeBi».\\
1867 410 # Condición: conversación.\\
1868 411 DT: muy bien, esta es la acción # señala la palabra «bebí» que AH acaba de escribir. a ver hágale así de que tomamos # hace el gesto con la mano izquierda de «beber».\\
1869 412 AH: comí es # señala la ilustración de la taza de café. este el café, café /. bibi ./ bebí café\\
1870 413 DT: eso ¿ya vio? el café es la cosa es diferente\\
1871 414 AH: sí cierto\\
1872 415 DT: a ver vamos a hacerlo otra vez\\
1873 416 AH: órale\\
1874 417 DT: estuvo bien eh, así como lo hizo está bien\\
1875 418 AH: sí\\
1876 419 DT: ¿qué dijimos que es?\\
1877 420 AH: café\\
1878 421 DT: ¿es cosa o es acción?\\
1879 422 AH: cosa\\
1880 423 DT: a ver acomódelo\\
1881 424 # se le entrega la ilustración con la taza de café\\
1882 425 AH: # coloca la ilustración en la columna de cosa.\\
1883 426 DT: muy bien, y ¿qué hace conel café? # le entrega el marcador para pizarrón.\\
1884 427 AH: cosa\\
1885 428 DT: a ver est ya está acá en cosa # acomoda la ilustración. ¿qué hace con el café usted?\\
1886 429 AH: este café\\
1887 430 D: ¿pero qué quedamos que hace con el café?\\
1888 431 DT: ¿qué hace?\\
1889 432 AH: bebí\\
1890 433 DT: muy bien\\
1891 434 AH: bebí\\
1892 435 # Condición: escritura.\\
1893 436 AH: be # escribe «Be». bí # escribe «Bi» resultando en «BeBi».\\
1894 437 # Condición: conversación.\\
1895 438 DT: por ejemplo ¿cuál estaría bien?\\
1896 439 D: el de los tamales\\
1897 440 DT: a ver si le gusta nuestro dibujito ¿esto qué es?\\
1898 441 # se le muestra una ilustración de un plato con 2 tamales.\\
1899 442 AH: tamal\\
1900 443 DT: my bien ¿y es cosa o es acción?\\
1901 444 AH: este cosa\\
1902 445 DT: acomódelo # le entrega la ilustración.\\
1903 446 AH: cosa # intenta alcanzar el marcador de pizarrón.\\
1904 447 DT: a ver primero acomode ese aquí encima # señala la columna de cosa.\\
1905 448 AH: así # coloca la ilustración en la columna señalada.\\
1906 449 DT: entonce el tamal cosa ¿y qué hace con el tamal?\\
1907 450 AH: cosa, tamal # intenta escribir en la columna de cosa, acción que no se pide en la actividad.\\
1908 451 DT: ¿pero qué hace con el tamal, usted qué hace?\\
1909 452 AH: cosa\\
1910 453 DT: sí es cosa ¿pero qué hace usted? así le sirvo aquí sus tamales # le hacerca ambas manos. ¿qué hace?\\
1911 454 AH: tamales, ay pérame(espérame)\\
1912 455 DT: los tamales ¿los avienta?\\
1913 456 AH: no no\\
1914 457 DT: ¿los vende?\\
1915 458 AH: sí sí\\
1916 459 DT: los puede vender\\
1917 460 D: ¿qué otra cosa puede hacer con los tamales? si lo tiene aquí y usted tiene hambre ¿qué hace?\\
1918 461 AH: acción\\
1919 462 D: sí muy bien ¿pero cómo se le llama? # hace el gesto con la mano derecha de llevarse algo a la boca. ¿se acuerda? \\
1920 463 AH: ay no es que me da miedo no sé me \\
1921 464 D: ¿por qué miedo?\\
1922 465 AH: # mueve los ojos a la izquierda donde se encuentra AC sin que este se percate, mueve los labios pero no dice nada y sube y baja las cejas.\\
1923 466 D: no pasa nada\\
1924 467 DT: a ver le podemos ayudar\\
1925 468 AH: a ver\\
1926 469 DT: ¿es comida? # señala la ilustración de los tamales.\\
1927 470 AH: comida\\
1928 471 DT: los tamales son comida\\
1929 472 AH: sí\\
1930 473 DT: ¿qué hace con la comida? se # hace el gesto de llevarse algo a la boca con la mano derecha.\\
1931 474 AH: se # hace el gesto de llevarse algo a la boca con la mano izquierda. be be\\
1932 475 DT: cerca\\
1933 476 D: más o menos pero no\\
1934 477 DT: beber es así ¿no? tengo mi vacito o mi tacita # hace el gesto de beber con la mano derecha. pero cuando tengo mi comida y hago esto # repite el gesto de llevarse algo a la boca con la mano derecha.\\
1935 478 AH: # frunce el seño.\\
1936 479 DT: ¿cómo se come sus tamales? así abre la hoja\\
1937 480 AH: sí\\
1938 481 DT: y ¿qué, con un tenedor?\\
1939 482 AH: ah un tenedor\\
1940 483 DT: así lo parte # hace el gesto de llevarse algo a la boca.\\
1941 484 AH: sí sí ¿y luego? # se empieza a reir.\\
1942 485 DT: ¿y luego? ajá eso le toca a usted\\
1943 486 AH: /m/\\
1944 487 DT: nos lo co\\
1945 488 AH: comemos\\
1946 489 DT: eso\\
1947 490 AH: comemos\\
1948 491 DT: esa es la acción\\
1949 492 AH: comemos\\
1950 493 DT: a ver póngala aquí # señala la columna de acción.\\
1951 494 AH: comemos\\
1952 495 # Condición: escritura\\
1953 496 AH: comemos, co # escribe «co». me # escribe «mi». mos # escribe «mos» resultando en «comimos».\\
1954 497 # Condición: conversación.\\
1955 498 DT: los tamales # señala la ilustración y después la palabra que acaba de escribir AH.\\
1956 499 AH: comemos comimos comemos\\
1957 500 DT: bueno está bien aquí le cambió una letra, a ver léalo\\
1958 501 AH: comimos\\
1959 502 DT: está bien no pasa nada, los tamales ¿qué hacemos con los tamales? # señala nuevamente la palabra escrita.\\
1960 503 AH: comimos\\
1961 504 DT: vamos a hacer uno último, para que no se nos sature\\
1962 505 AH: ajá\\
1963 506 DT: a ver este ya lo habíamos usado\\
1964 507 # se le muestra una ilustración de un cuchillo de cocina.\\
1965 508 AH: cuchara cuchillo\\
1966 509 DT: ¿y es cosa o es acción?\\
1967 510 AH: este cosa\\
1968 511 DT: acomódelo # asiente y le entrega la ilustración.\\
1969 512 AH: # coloca la ilustración en la columna de cosa.\\
1970 513 DT: eso # le entrega el marcador de pizarrón. ¿y qué hace con el cuchillo?\\
1971 514 AH: este cu ...\\
1972 515 DT: mire to también me # le muestra una cortada pequeña que tiene el brazo. como usted\\
1973 516 AH: ah mira sí\\
1974 517 AC: aquí tiene otra # señala otra cortada en el brazo de DT.\\
1975 518 DT: ah sí aquí tengo otra\\
1976 519 AH: ah mira\\
1977 520 DT: nada más que estas no me las hice con un cuchillo, me las hizo mi gato\\
1978 521 AH: ay ¿a poco? # muestra sorpresa. íjole(expresión de sorpresa)\\
1979 522 DT: pero usted con el cuchillo ¿qué se hizo?\\
1980 523 AH: este con ira(mira) # señala su dedo medio de la mano izquierda. con /s/\\
1981 524 DT: sí ¿cuál es la acción?\\
1982 525 AH: este cuchillo está bien filoso\\
1983 526 DT: ¿y qué pasó porque estaba filoso? se # con la mano derecha extendida hace el gesto de «cortar» sobre su mano izquierda.\\
1984 527 AH: machu/k/ machuqué\\
1985 528 DT: más o menos\\
1986 529 D: más o menos\\
1987 530 AH: machuqué\\
1988 531 DT: va por ahí es otro verbo # continúa haciendo el gesto de «cortar» con la mano derecha\\
1989 532 AH: me machuqué\\
1990 533 DT: me cor\\
1991 534 AH: me corté me corté\\
1992 535 DT: eso\\
1993 536 AH: me corté\\
1994 537 DT: a ver escriba eso # señala la columna de acción.\\
1995 538 # Condición: escritura.\\
1996 539 AH: me # escribe «me». cor # escribe «cor». té # escribe «te» resultando en «me corte».\\
1997 540 DT: ¿con el cuchillo?\\
1998 541 AH: cuchillo # asiente pero no produce la oración completa, sólo repite la frase. con el cuchillo\\
1999 542 DT: muy bien ahora vamos a decirlotodo completo\\
2000 543 AH: ajá\\
2001 544 DT: ¿qué pasó con el cuchillo?\\
2002 545 AH: me corté # lee la palabra escrita el pizarrón.\\
2003 546 DT: a ver dígalo todo completo\\
2004 547 AH: me corté\\
2005 548 D: muy bien\\
2006 549 DT: con\\
2007 550 AH: con el cuchillo\\
2008 551 DT: otra vez todo completo\\
2009 552 AH: me corté con el cuchillo\\
2010 553 DT: eso muy bien, a ver ¿este qué habíamos dicho? # le muestra la ilustración  de los tamales nuevamente.\\
2011 554 AH: el tamal el tamal\\
2012 555 DT: ajá muy bien # le señala la palabra escrita en el pizarrón «comimos».\\
2013 556 AH: comimos tamal\\
2014 557 DT: a ver otra vez\\
2015 558 AH: comimos tamal\\
2016 559 DT: eso, ¿y el último? # le muestra la ilustración de la taza con café nuevamente.\\
2017 560 AH: el café # lee la palabra «BeBi» escrita en el pizarrón. bebí café\\
2018 561 DT: eso muy bien ¿ya vio la diferencia?\\
2019 562 AH: ajá sí cierto\\
2020 563 ...\\
2021 564 DT: vamos a hacer un último ejercicio\\
2022 565 AH: sí\\
2023 566 DT: va a ser un poquito más difícil\\
2024 567 AH: sí cierto\\
2025 568 DT: pero lo hizo muy bien\\
2026 569 ...\\
2027 570 DT: entonces vamos a ver mire con azul las cosas y con rojo las acciones # se le muestra una hoja blanca de forma horizontal dividida en dos columnas, la izquierda dice «Acción» en color rojo y la derecha «Cosa» en azul. usted ahorita va a escojer, vamos a usar su tarea\\
2028 571 AH: sí\\
2029 572 DT: cepillar\\
2030 573 AH: se\\
2031 574 DT: cepillar, a ver yo voy a hacer la acción # con la mano derecha hace el gesto de «cepillar los dientes». ¿usted lo hace?\\
2032 575 AH: sí sí\\
2033 576 DT: a ver ¿cómo se lava los dientes?\\
2034 577 AH: # lleva la mano derecha con las puntas de los dedos juntas a la comisura izquierda de sus labios, luego a la derecha y repite esta acción dos veces. sí sí sí\\
2035 578 DT: esta es la acción\\
2036 579 AH: acción sí\\
2037 580 DT: cepillar\\
2038 581 AH: sí\\
2039 582 DT: a ver diga cepillar\\
2040 583 AH: cepillar cepillar\\
2041 584 DT: ¿dónde va? # señala la hoja.\\
2042 585 AH: aquí # señala la columna que dice acción. acción\\
2043 586 DT: muy bien # le da el marcador rojo.\\
2044 587 AH: ¿cepillar?\\
2045 588 DT: sí\\
2046 589 # Condición: escritura.\\
2047 590 AH: # escribe la letra c. ce # escribe «cep». pi # escribe la letra i segido de «ll». llar # escribe «ar» resultando en «cepillar».\\
2048 591 DT: muy bien esa es la acción\\
2049 592 AH: sí ajá\\
2050 593 DT: y ¿dientes?\\
2051 594 AH: los dientes aquí # señala la columna de cosa. cosa\\
2052 595 DT: eso muy bien a ver\\
2053 596 AH: cosa\\
2054 597 DT: dientes\\
2055 598 AH: dientes\\
2056 599 # Condición: escritura.\\
2057 600 AH: # escribe «diE». dien # escribe «Nte». tes # escribe la letra s resultando en «diENtes».\\
2058 601 # Condición: conversación.\\
2059 602 DT: muy bien a ver ahora dígalo # le señala la palabra cepillar y luego dientes.\\
2060 603 AH: cepillar dientes\\
2061 604 DT: ¿cómo lo puede decir? yo\\
2062 605 AH: yo yo cepillé los dientes\\
2063 606 DT: perfecto, a ver vamos a intetar ahora con agua\\
2064 607 AH: /a/ agua # señala primero la columna cosa y luego la de acción, comienza a hacer un movimient con el dedo índice de la mano derecha sobre la hoja en la columna de cosa simulando la escritura de la palabra requerida, parece escribir en el aire «agi».\\
2065 608 DT: a ver ¿es una cosa el agua?\\
2066 609 AH: acción # señala la columna de cosa. agua agua\\
2067 610 DT: estaba bien aquí # señala al mismo tiempo que AH la columna de cosa.\\
2068 611 # Condición: escritura.\\
2069 612 AH: # sin decir la palabra escribe «agui» similar a la acción anterior de escribir en el aire con el dedo índice. /a/ agua # nota el error y corrige escribiendo sobre la letra i una A, resultando en «aguA».\\
2070 613 DT: muy bien ¿y servir?\\
2071 614 AH: /. serville serv ./ \\
2072 615 DT: servir\\
2073 616 AH: servir aquí # señala la columna de acción. acción\\
2074 617 DT: sí muy bien a ver\\
2075 618 AH: ((vio))\\
2076 619 DT: servir\\
2077 620 AH: servir\\
2078 621 # Condición: escritura.\\
2079 622 AH: ser # escribe «CE». /s/ ser # escribe la letra N resultando en «CEN». ¿no?\\
2080 623 DT: servir\\
2081 624 AH: servir # escribe la letra d. ser # escribe la letra i. vir # escribe la letra N, resultando en «CENdiN».\\
2082 625 DT: no pasa nada ¿cómo sería completo? # señala la palabra escrita por AH «CENdiN» y después\\
2083 626 AH: servir\\
2084 627 DT: servir # señala la palabra «aguA».\\
2085 628 AH: agua\\
2086 629 DT: a ver ahora ¿cómo lo diría completo?\\
2087 630 AH: servir agua coca este cosa\\
2088 631 DT: ah ya se le antojó la coca\\
2089 632 AH: cosa # comienza a reirse.\\
2090 633 DT: muy bien, y ahora vamos a ponerle uno que no ha hecho usted, por ejemplo ver\\
2091 634 AH: ah\\
2092 635 DT: ¿qué es cosa o acción?\\
2093 636 AH: ver ver\\
2094 637 DT: ver ¿cuál es?\\
2095 638 AH: a ver # señala la columna de cosa.\\
2096 639 DT: ¿segura?\\
2097 640 AH: acción # continúa señalando la misma columna. ver\\
2098 641 DT: a ver # le da el marcador rojo.\\
2099 642 AH: ver\\
2100 643 # Condición: escritura.\\
2101 644 AH: ver # escribe «ER». ver ver \\
2102 645 DT: a ver # cubre con un objeto lo que AH acaba de escribir. aquí abajito escríbalo, ver\\
2103 646 AH: ver # escribe «Per»\\
2104 647 DT: y a ver, tele(televisión)\\
2105 648 AH: ¿tele?\\
2106 649 DT: la tele\\
2107 650 AH: coca cosa cosa cosa\\
2108 651 DT: muy bien, sí se le antojó la coca\\
2109 652 AH: # se rie pero no escribe.\\
2110 653 DT: tele, la tele\\
2111 654 AH: # escribe «te». tele # escribe «Le». vi # escribe «b». vi # escribe la letra i. /s/ # escribe «ci». ción # escribe oN, resultando en «teLebicioN».\\
2112 655 DT: bien ¿usted ve la televisión?\\
2113 656 AH: sí\\
2114 657 DT: a ver diga así veo la televisión\\
2115 658 AH: ayer ayer estaba en la televisión\\
2116 659 DT: ¿usted estaba en la televisión?\\
2117 660 AH: sí sí\\
2118 661 DT: ¿la grabaron?\\
2119 662 AH: no hombre # se comienza a reir.\\
2120 663 DT: estaba viendo la televisión\\
2121 664 AH: vien\\
2122 665 DT: viendo la televisión\\
2123 666 AH: viendo la televisión # lo dice al mismo tiempo que DT.\\
2124 667 # Segmento: escritura al dictado con dibujos.\\
2125 668 # Material: una nueva hoja en blanco y un lápiz.\\
2126 669 DT: a ver va a escribir ¿qué hace con un lápiz?\\
2127 670 AH: lápiz\\
2128 671 DT: con el lápiz ¿qué hacemos? # señala el cuaderno de AH.\\
2129 672 AH: escribir\\
2130 673 D: muy bien\\
2131 674 DT: ¿puede escribir aquí una oración usted?\\
2132 675 # Condición: escritura.\\
2133 676 AH: escribir escribir es # escribe «ES». cri a ver # voltea a ver su cuaderno para copiar la palabra. \\
2134 677 DT: ah no sin trampas sin trampas\\
2135 678 AH: escribir es cri bir # dice la palabra separando por sílabas y concluye la palabra resultando en «ESiBiN», el cambio de la grafía de N es sustición de la R ha sido constante.\\
2136 679 # Condición: conversación.\\
2137 680 DT: a ver ¿se parece? vamos a comparar # pone la hoja donde acaba de escribir AH al lado de las palabras escritas en su cuaderno. ¿se parece lo que usted puso?\\
2138 681 AH: no\\
2139 682 DT: ¿no se parece?\\
2140 683 AH: no\\
2141 684 DT: a ver escríbalo aquía bajo\\
2142 685 # Condición: escritura.\\
2143 686 AH: a ver # copia la palabra «escribir» anotada en su cuaderno.\\
2144 687 DT: había faltado ¿verdad?\\
2145 688 AH: sí\\
2146 689 DT: ¿con qué escribimos?\\
2147 690 AH: el lápiz\\
2148 691 DT: a ver escríbalo\\
2149 692 AH: lápiz # copia la palabra lápiz escrita en su cuaderno.\\
2150 693 DT: a ver ¿puede dibujar un lápiz?\\
2151 694 # Condición: dibujo.\\
2152 695 AH: # dibuja al final de la hoja un óvalo pequeño de un centímetro aproximadamente, de la parte inferior del óvalo dibuja una línea recta vertical de un centímetro.\\
2153 696 DT: a ver ¿qué es esto? # señala el dibujo que AH acaba de ralizar.\\
2154 697 AH: pis # le provoca risa y comienza a hacer trazos con el lápiz sobre la palma de su mano izquierda. la ah no\\
2155 698 DT: ¿qué dibujó? platíqueme usted qué dibujó\\
2156 699 AH: a ver # vuelve a hacer trazos sobre la palma de su mano izquierda. la a ver lápiz # comienza a reirse. \\
2157 700 DT: a ver para que se guíe # dibuja un rectángulo debajo de la palabra lápiz que AH escribió. aquí adentro dibuje un lápiz # le entrega el lápiz a AH. dibujado\\
2158 701 AH: # sin habla comienza a dibujar una línea recta vertical, en el extremo superior dibuja un círculo pequeño, dibujo similar al anterior.\\
2159 702 DT: ¿esto qué es? # señala el dibujo reciente.\\
2160 703 AH: lápiz\\
2161 704 DT: ¿esta parte de aquí qué es? # señala el círculo.\\
2162 705 AH: la goma la goma\\
2163 706 DT: ¿y todo esto que dibujó? # señala la línea vertical.\\
2164 707 AH: ...\\
2165 708 DT: a ver ¿esto qué es? # señala la punta del lápiz con el que hizo el dibujo.\\
2166 709 AH: la punta\\
2167 710 D: muy bien\\
2168 711 DT: a ver # dibuja un rectángulo debajo de la palabra «escribir» que AH escribió. ¿con qué escribe?\\
2169 712 AH: una pluma\\
2170 713 DT: ¿con qué partes del cuerpo escribe?\\
2171 714 AH: aquí con la pluma # señala el lápiz con su mano izquierda.\\
2172 715 DT: ¿y qué mueve el lápiz, se mueve sólo?\\
2173 716 AH: no\\
2174 717 DT: ¿cómo se usa? con\\
2175 718 AH: con la mano con la mano\\
2176 719 DT: a ver dibújeme una manita aquí # señala el rectángulo debajo de la palabra «escribir».\\
2177 720 AH: con la mano # dibuja una líea con cinco ángulos como «^^^^^» (podría tratarse de una representación de los dedos de la mano).\\
2178 721 DT: los cinco dedos\\
2179 722 AH: sí\\
2180 723 DT: muy bien entonces escribimos # señala el dibujo de la mano.\\
2181 724 AH: escribir la mano la mano con la mano\\
2182 725 DT: a ver todo completo\\
2183 726 AH: escribir con la cap con la /k/ escribir con # voltea a ver su mano derecha donde sostiene el lápiz. con la mano\\
2184 727 DT: ¿o? # señala el dibujo del lápiz.\\
2185 728 AH: o la o con lápiz\\
2186 \\
